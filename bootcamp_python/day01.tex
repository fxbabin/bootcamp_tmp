\PassOptionsToPackage{unicode=true}{hyperref} % options for packages loaded elsewhere
\PassOptionsToPackage{hyphens}{url}
\PassOptionsToPackage{dvipsnames,svgnames*,x11names*}{xcolor}
%
\documentclass[]{article}
\usepackage{lmodern}
\usepackage{amssymb,amsmath}
\usepackage{ifxetex,ifluatex}
\usepackage{fixltx2e} % provides \textsubscript
\ifnum 0\ifxetex 1\fi\ifluatex 1\fi=0 % if pdftex
  \usepackage[T1]{fontenc}
  \usepackage[utf8x]{inputenc}
  \usepackage{textcomp} % provides euro and other symbols
\else % if luatex or xelatex
  \usepackage{unicode-math}
  \defaultfontfeatures{Ligatures=TeX,Scale=MatchLowercase}
\fi
% use upquote if available, for straight quotes in verbatim environments
\IfFileExists{upquote.sty}{\usepackage{upquote}}{}
% use microtype if available
\IfFileExists{microtype.sty}{%
\usepackage[]{microtype}
\UseMicrotypeSet[protrusion]{basicmath} % disable protrusion for tt fonts
}{}
\IfFileExists{parskip.sty}{%
\usepackage{parskip}
}{% else
\setlength{\parindent}{0pt}
\setlength{\parskip}{6pt plus 2pt minus 1pt}
}
\usepackage{xcolor}
\usepackage{hyperref}
\hypersetup{
            colorlinks=true,
            linkcolor=Maroon,
            citecolor=Blue,
            urlcolor=blue,
            breaklinks=true}
\urlstyle{same}  % don't use monospace font for urls
\usepackage{color}
\usepackage{fancyvrb}
\newcommand{\VerbBar}{|}
\newcommand{\VERB}{\Verb[commandchars=\\\{\}]}
\DefineVerbatimEnvironment{Highlighting}{Verbatim}{commandchars=\\\{\}}
% Add ',fontsize=\small' for more characters per line
\usepackage{framed}
\definecolor{shadecolor}{RGB}{35,38,41}
\newenvironment{Shaded}{\begin{snugshade}}{\end{snugshade}}
\newcommand{\AlertTok}[1]{\textcolor[rgb]{0.58,0.85,0.30}{#1}}
\newcommand{\AnnotationTok}[1]{\textcolor[rgb]{0.25,0.50,0.35}{#1}}
\newcommand{\AttributeTok}[1]{\textcolor[rgb]{0.16,0.50,0.73}{#1}}
\newcommand{\BaseNTok}[1]{\textcolor[rgb]{0.96,0.45,0.00}{#1}}
\newcommand{\BuiltInTok}[1]{\textcolor[rgb]{0.50,0.55,0.55}{#1}}
\newcommand{\CharTok}[1]{\textcolor[rgb]{0.24,0.68,0.91}{#1}}
\newcommand{\CommentTok}[1]{\textcolor[rgb]{0.48,0.49,0.49}{#1}}
\newcommand{\CommentVarTok}[1]{\textcolor[rgb]{0.50,0.55,0.55}{#1}}
\newcommand{\ConstantTok}[1]{\textcolor[rgb]{0.15,0.68,0.68}{#1}}
\newcommand{\ControlFlowTok}[1]{\textcolor[rgb]{0.99,0.74,0.29}{#1}}
\newcommand{\DataTypeTok}[1]{\textcolor[rgb]{0.16,0.50,0.73}{#1}}
\newcommand{\DecValTok}[1]{\textcolor[rgb]{0.96,0.45,0.00}{#1}}
\newcommand{\DocumentationTok}[1]{\textcolor[rgb]{0.64,0.20,0.25}{#1}}
\newcommand{\ErrorTok}[1]{\textcolor[rgb]{0.85,0.27,0.33}{#1}}
\newcommand{\ExtensionTok}[1]{\textcolor[rgb]{0.00,0.60,1.00}{#1}}
\newcommand{\FloatTok}[1]{\textcolor[rgb]{0.96,0.45,0.00}{#1}}
\newcommand{\FunctionTok}[1]{\textcolor[rgb]{0.56,0.27,0.68}{#1}}
\newcommand{\ImportTok}[1]{\textcolor[rgb]{0.15,0.68,0.38}{#1}}
\newcommand{\InformationTok}[1]{\textcolor[rgb]{0.77,0.36,0.00}{#1}}
\newcommand{\KeywordTok}[1]{\textcolor[rgb]{0.81,0.81,0.76}{#1}}
\newcommand{\NormalTok}[1]{\textcolor[rgb]{0.81,0.81,0.76}{#1}}
\newcommand{\OperatorTok}[1]{\textcolor[rgb]{0.81,0.81,0.76}{#1}}
\newcommand{\OtherTok}[1]{\textcolor[rgb]{0.15,0.68,0.38}{#1}}
\newcommand{\PreprocessorTok}[1]{\textcolor[rgb]{0.15,0.68,0.38}{#1}}
\newcommand{\RegionMarkerTok}[1]{\textcolor[rgb]{0.16,0.50,0.73}{#1}}
\newcommand{\SpecialCharTok}[1]{\textcolor[rgb]{0.24,0.68,0.91}{#1}}
\newcommand{\SpecialStringTok}[1]{\textcolor[rgb]{0.85,0.27,0.33}{#1}}
\newcommand{\StringTok}[1]{\textcolor[rgb]{0.96,0.31,0.31}{#1}}
\newcommand{\VariableTok}[1]{\textcolor[rgb]{0.15,0.68,0.68}{#1}}
\newcommand{\VerbatimStringTok}[1]{\textcolor[rgb]{0.85,0.27,0.33}{#1}}
\newcommand{\WarningTok}[1]{\textcolor[rgb]{0.85,0.27,0.33}{#1}}
\usepackage{longtable,booktabs}
% Fix footnotes in tables (requires footnote package)
\IfFileExists{footnote.sty}{\usepackage{footnote}\makesavenoteenv{longtable}}{}
\setlength{\emergencystretch}{3em}  % prevent overfull lines
\providecommand{\tightlist}{%
  \setlength{\itemsep}{0pt}\setlength{\parskip}{0pt}}
\setcounter{secnumdepth}{0}
% Redefines (sub)paragraphs to behave more like sections
\ifx\paragraph\undefined\else
\let\oldparagraph\paragraph
\renewcommand{\paragraph}[1]{\oldparagraph{#1}\mbox{}}
\fi
\ifx\subparagraph\undefined\else
\let\oldsubparagraph\subparagraph
\renewcommand{\subparagraph}[1]{\oldsubparagraph{#1}\mbox{}}
\fi

% set default figure placement to htbp
\makeatletter
\def\fps@figure{h}
\makeatother


\date{}

%%%%%%%%%%%%%%%%%%%%%%%%%%%%%%%%%%%%%%%%%%%%%%%%%%%%%%%%%%%%%%%%%%%%%%%%%%%%%%%%
%%%%%%%%%%%%%%%%%%%%%%%%%%%%%%%% Added packages %%%%%%%%%%%%%%%%%%%%%%%%%%%%%%%%
%%%%%%%%%%%%%%%%%%%%%%%%%%%%%%%%%%%%%%%%%%%%%%%%%%%%%%%%%%%%%%%%%%%%%%%%%%%%%%%%

\setcounter{MaxMatrixCols}{20}
\usepackage{cancel}
\usepackage{calc}
\usepackage{eso-pic}
\newlength{\PageFrameTopMargin}
\newlength{\PageFrameBottomMargin}
\newlength{\PageFrameLeftMargin}
\newlength{\PageFrameRightMargin}

\setlength{\PageFrameTopMargin}{1.5cm}
\setlength{\PageFrameBottomMargin}{1cm}
\setlength{\PageFrameLeftMargin}{1cm}
\setlength{\PageFrameRightMargin}{1cm}

\makeatletter

\newlength{\Page@FrameHeight}
\newlength{\Page@FrameWidth}

\AddToShipoutPicture{
  \thinlines
  \setlength{\Page@FrameHeight}{\paperheight-\PageFrameTopMargin-\PageFrameBottomMargin}
  \setlength{\Page@FrameWidth}{\paperwidth-\PageFrameLeftMargin-\PageFrameRightMargin}
  \put(\strip@pt\PageFrameLeftMargin,\strip@pt\PageFrameTopMargin){
    \framebox(\strip@pt\Page@FrameWidth, \strip@pt\Page@FrameHeight){}}}

\makeatother

\usepackage{fvextra}
\DefineVerbatimEnvironment{Highlighting}{Verbatim}{breaklines,breakanywhere,commandchars=\\\{\}}

\usepackage{graphicx}

\usepackage[a4paper, total={6in, 8in}]{geometry}
\geometry{hmargin=2cm,vmargin=2cm}

\usepackage{sectsty}

\sectionfont{\centering\Huge}
\subsectionfont{\Large}
\subsubsectionfont{\large}

\usepackage{titlesec}
\titlespacing*{\section}
{0pt}{5.5ex plus 1ex minus .2ex}{4.3ex plus .2ex}

\tolerance=1
\emergencystretch=\maxdimen
\hyphenpenalty=10000
\hbadness=10000

%%%%%%%%%%%%%%%%%%%%%%%%%%%%%%%%%%%%%%%%%%%%%%%%%%%%%%%%%%%%%%%%%%%%%%%%%%%%%%%%
%%%%%%%%%%%%%%%%%%%%%%%%%%%%%%%%%%%%%%%%%%%%%%%%%%%%%%%%%%%%%%%%%%%%%%%%%%%%%%%%

\begin{document}

%%%%%%%%%%%%%%%%%%%%%%%%%%%%%%%%%%%%%%%%%%%%%%%%%%%%%%%%%%%%%%%%%%%%%%%%%%%%%%%%
%%%%%%%%%%%%%%%%%%%%%%%%%%%%%%%% Added lines %%%%%%%%%%%%%%%%%%%%%%%%%%%%%%%%%%%
%%%%%%%%%%%%%%%%%%%%%%%%%%%%%%%%%%%%%%%%%%%%%%%%%%%%%%%%%%%%%%%%%%%%%%%%%%%%%%%%

\vspace*{2cm}
\begin{center}
    \textsc{\fontsize{40}{48} \bfseries Bootcamp}\\[0.6cm]
    \textsc{\fontsize{39}{48} \bfseries { %bootcamp_title
Python
    }}\\[0.3cm]
\end{center}
\vspace{3cm}

\begin{center}
\includegraphics[width=200pt]{assets/logo-42-ai.png}{\centering}
\end{center}

\vspace*{2cm}
\begin{center}
    \textsc{\fontsize{32}{48} \bfseries %day_number
Day01    
    }\\[0.6cm]
    \textsc{\fontsize{32}{48} \bfseries %day_title
Basics 2    
    }\\[0.3cm]
\end{center}
\vspace{3cm}

\pagenumbering{gobble}
\newpage

%%% >>>>> Page de garde
\setcounter{page}{1}
\pagenumbering{arabic}

%%%%%%%%%%%%%%%%%%%%%%%%%%%%%%%%%%%%%%%%%%%%%%%%%%%%%%%%%%%%%%%%%%%%%%%%%%%%%%%%
%%%%%%%%%%%%%%%%%%%%%%%%%%%%%%%%%%%%%%%%%%%%%%%%%%%%%%%%%%%%%%%%%%%%%%%%%%%%%%%%


\hypertarget{bootcamp-python}{%
\section{Bootcamp Python}\label{bootcamp-python}}

\hypertarget{day01---basics-2}{%
\section{Day01 - Basics 2}\label{day01---basics-2}}

The goal of the day is to get familiar with object-oriented programming
and much more.

\hypertarget{notions-of-the-day}{%
\subsection{Notions of the day}\label{notions-of-the-day}}

Objects, cast, class, inheritance, built-in functions, magic methods,
generator, constructor, iterator, \ldots{}

\hypertarget{general-rules}{%
\subsection{General rules}\label{general-rules}}

\begin{itemize}
\item
  The version of Python to use is 3.7, you can check the version of
  Python with the following command: \texttt{python\ -V}
\item
  The norm: during this bootcamp you will follow the
  \href{https://www.python.org/dev/peps/pep-0008/}{PEP 8 standards}. You
  can install \href{https://pypi.org/project/pycodestyle}{pycodestyle}
  which is a tool to check your Python code.
\item
  The function eval is never allowed.
\item
  The exercises are ordered from the easiest to the hardest.
\item
  Your exercises are going to be evaluated by someone else, so make sure
  that your variable names and function names are appropriate and civil.
\item
  Your manual is the internet.
\item
  You can also ask questions in the dedicated channel in the 42 AI
  Slack: 42-ai.slack.com.
\item
  If you find any issue or mistakes in the subject please create an
  issue on our
  \href{https://github.com/42-AI/bootcamp_python/issues}{dedicated
  repository on Github}.
\end{itemize}

\hypertarget{helper}{%
\subsection{Helper}\label{helper}}

Ensure that you have the right Python version.

\begin{Shaded}
\begin{Highlighting}[]
\NormalTok{> which python}
\NormalTok{/goinfre/miniconda/bin/python}
\NormalTok{> python -V}
\NormalTok{Python 3.7.*}
\NormalTok{> which pip}
\NormalTok{/goinfre/miniconda/bin/pip}
\end{Highlighting}
\end{Shaded}

\hypertarget{exercise-00---the-book}{%
\subsubsection{Exercise 00 - The Book}\label{exercise-00---the-book}}

\hypertarget{exercise-01---family-tree}{%
\subsubsection{Exercise 01 - Family
tree}\label{exercise-01---family-tree}}

\hypertarget{exercise-02---the-vector}{%
\subsubsection{Exercise 02 - The
Vector}\label{exercise-02---the-vector}}

\hypertarget{exercise-03---the-matrix}{%
\subsubsection{Exercise 03 - The
Matrix}\label{exercise-03---the-matrix}}

\hypertarget{exercise-04---generator}{%
\subsubsection{Exercise 04 - Generator!}\label{exercise-04---generator}}

\hypertarget{exercise-05---working-with-lists}{%
\subsubsection{Exercise 05 - Working with
lists}\label{exercise-05---working-with-lists}}

\hypertarget{exercise-06---bank-account}{%
\subsubsection{Exercise 06 - Bank
account}\label{exercise-06---bank-account}}

\clearpage

\hypertarget{exercise-00---the-book-1}{%
\section{Exercise 00 - The Book}\label{exercise-00---the-book-1}}

\begin{longtable}[]{@{}rl@{}}
\toprule
\endhead
Turn-in directory : & ex00\tabularnewline
Files to turn in : & book.py, recipe.py, test.py\tabularnewline
Forbidden functions : & None\tabularnewline
Remarks : & n/a\tabularnewline
\bottomrule
\end{longtable}

You will provide a test.py file to test your classes and prove that they
are working the right way.\\
You can import all the classes into your test.py file by adding these
lines at the top of the test.py file:

\begin{Shaded}
\begin{Highlighting}[]
\ImportTok{from}\NormalTok{ book }\ImportTok{import}\NormalTok{ Book}
\ImportTok{from}\NormalTok{ recipe }\ImportTok{import}\NormalTok{ Recipe}
\end{Highlighting}
\end{Shaded}

You will have to make a class \texttt{Book} and a class \texttt{Recipe}

Let's describe the \texttt{Recipe} class. It has some attributes:

\begin{itemize}
\item
  \texttt{name} (str)
\item
  \texttt{cooking\_lvl} (int) : range 1 to 5
\item
  \texttt{cooking\_time} (int) : in minutes (no negative numbers)
\item
  \texttt{ingredients} (list) : list of all ingredients each represented
  by a string
\item
  \texttt{description} (str) : description of the recipe
\item
  \texttt{recipe\_type} (str) : can be ``starter'', ``lunch'' or
  ``dessert''.
\end{itemize}

You have to initialize the object \texttt{Recipe} and check all its
values, only the description can be empty.\\
In case of input errors, you should print what they are and exit
properly.

You will have to implement the built-in method \texttt{\_\_str\_\_}.\\
It's the method called when you execute this code:

\begin{Shaded}
\begin{Highlighting}[]
\NormalTok{tourte }\OperatorTok{=}\NormalTok{ Recipe(...)}
\NormalTok{to_print }\OperatorTok{=} \BuiltInTok{str}\NormalTok{(tourte)}
\BuiltInTok{print}\NormalTok{(to_print)}
\end{Highlighting}
\end{Shaded}

It's implemented this way:

\begin{Shaded}
\begin{Highlighting}[]
\KeywordTok{def} \FunctionTok{__str__}\NormalTok{(}\VariableTok{self}\NormalTok{):}
    \CommentTok{"""Return the string to print with the recipe info"""}
\NormalTok{    txt }\OperatorTok{=} \StringTok{""}
    \CommentTok{"""Your code goes here"""}
    \ControlFlowTok{return}\NormalTok{ txt}
\end{Highlighting}
\end{Shaded}

The \texttt{Book} class also has some attributes:

\begin{itemize}
\item
  \texttt{name} (str)
\item
  \texttt{last\_update}
  (\href{https://docs.python.org/3/library/datetime.html}{datetime})
\item
  \texttt{creation\_date}
  (\href{https://docs.python.org/3/library/datetime.html}{datetime})
\item
  \texttt{recipes\_list} (dict) : a dictionnary why 3 keys: ``starter'',
  ``lunch'', ``dessert''.
\end{itemize}

You will have to implement some methods in \texttt{Book}:

\begin{Shaded}
\begin{Highlighting}[]
\KeywordTok{def}\NormalTok{ get_recipe_by_name(}\VariableTok{self}\NormalTok{, name):}
    \CommentTok{"""Print a recipe with the name `name` and return the instance"""}
    \ControlFlowTok{pass}

\KeywordTok{def}\NormalTok{ get_recipes_by_types(}\VariableTok{self}\NormalTok{, recipe_type):}
    \CommentTok{"""Get all recipe names for a given recipe_type """}
    \ControlFlowTok{pass}

\KeywordTok{def}\NormalTok{ add_recipe(}\VariableTok{self}\NormalTok{, recipe):}
    \CommentTok{"""Add a recipe to the book and update last_update"""}
    \ControlFlowTok{pass}
\end{Highlighting}
\end{Shaded}

You will have to handle the error if the arg passed in add\_recipe is
not a \texttt{Recipe}.

\clearpage

\hypertarget{exercise-01---family-tree-1}{%
\section{Exercise 01 - Family tree}\label{exercise-01---family-tree-1}}

\begin{longtable}[]{@{}rl@{}}
\toprule
\endhead
Turn-in directory : & ex01\tabularnewline
Files to turn in : & game.py\tabularnewline
Forbidden functions : & None\tabularnewline
Remarks : & n/a\tabularnewline
\bottomrule
\end{longtable}

You will have to make a class and its children.

Create a \texttt{GotCharacter} class and initialize it with the
following attributes:

\begin{itemize}
\item
  \texttt{first\_name}
\item
  \texttt{is\_alive} (by default is \texttt{True})
\end{itemize}

Pick up a GoT House (e.g., Stark, Lannister\ldots{}). Create a child
class that inherits from \texttt{GotCharacter} and define the following
attributes:

\begin{itemize}
\item
  \texttt{family\_name} (by default should be the same as the Class)
\item
  \texttt{house\_words} (e.g., the House words for the Stark House is:
  ``Winter is Coming'')
\end{itemize}

Example:

\begin{Shaded}
\begin{Highlighting}[]
\KeywordTok{class}\NormalTok{ Stark(GotCharacter):}
    \KeywordTok{def} \FunctionTok{__init__}\NormalTok{(}\VariableTok{self}\NormalTok{, first_name}\OperatorTok{=}\VariableTok{None}\NormalTok{, is_alive}\OperatorTok{=}\VariableTok{True}\NormalTok{):}
        \BuiltInTok{super}\NormalTok{().}\FunctionTok{__init__}\NormalTok{(first_name}\OperatorTok{=}\NormalTok{first_name, is_alive}\OperatorTok{=}\NormalTok{is_alive)}
        \VariableTok{self}\NormalTok{.family_name }\OperatorTok{=} \StringTok{"Stark"}
        \VariableTok{self}\NormalTok{.house_words }\OperatorTok{=} \StringTok{"Winter is Coming"}
\end{Highlighting}
\end{Shaded}

Add two methods to your child class:

\begin{itemize}
\item
  \texttt{print\_house\_words}: prints to screen the House words
\item
  \texttt{die}: changes the value of \texttt{is\_alive} to
  \texttt{False}
\end{itemize}

Running commands in the Python console, an example of what you should
get:

\begin{Shaded}
\begin{Highlighting}[]
\NormalTok{> python}
\NormalTok{>>> from game import GotCharacter, Stark}
\NormalTok{>>> arya = Stark("Arya")}
\NormalTok{>>> print(arya.__dict__)}
\NormalTok{\{'first_name': 'Arya', 'is_alive': True, 'family_name': 'Stark', 'house_words': 'Winter is Coming'\}}
\NormalTok{>>> arya.print_house_words()}
\NormalTok{Winter is Coming}
\NormalTok{>>> print(arya.is_alive)}
\NormalTok{True}
\NormalTok{>>> arya.die()}
\NormalTok{>>> print(arya.is_alive)}
\NormalTok{False}
\end{Highlighting}
\end{Shaded}

You can add any attribute or method you need to your class and format
the docstring the way you want to.\\
Feel free to create other children of \texttt{GotCharacter}.

\begin{Shaded}
\begin{Highlighting}[]
\NormalTok{>>> print(arya.__doc__)}
\NormalTok{A class representing the Stark family. Or when bad things happen to good people.}
\end{Highlighting}
\end{Shaded}

\clearpage

\hypertarget{exercise-02---the-vector-1}{%
\section{Exercise 02 - The Vector}\label{exercise-02---the-vector-1}}

\begin{longtable}[]{@{}rl@{}}
\toprule
\endhead
Turn-in directory : & ex02\tabularnewline
Files to turn in : & vector.pytest.py\tabularnewline
Forbidden functions : & None\tabularnewline
Forbidden libraries : & NumPy\tabularnewline
Remarks : & n/a\tabularnewline
\bottomrule
\end{longtable}

You will provide a testing file to prove that your class works as
expected.\\
You will have to create a helpful class, with more options and providing
enhanced ease of use for the user.

In this exercise, you have to create a \texttt{Vector} class. The goal
is to have vectors and be able to perform mathematical operations with
them.

\begin{Shaded}
\begin{Highlighting}[]
\OperatorTok{>>}\NormalTok{ v1 }\OperatorTok{=}\NormalTok{ Vector([}\FloatTok{0.0}\NormalTok{, }\FloatTok{1.0}\NormalTok{, }\FloatTok{2.0}\NormalTok{, }\FloatTok{3.0}\NormalTok{])}
\OperatorTok{>>}\NormalTok{ v2 }\OperatorTok{=}\NormalTok{ v1 }\OperatorTok{*} \DecValTok{5}
\OperatorTok{>>} \BuiltInTok{print}\NormalTok{(v2)}
\NormalTok{(Vector [}\FloatTok{0.0}\NormalTok{, }\FloatTok{5.0}\NormalTok{, }\FloatTok{10.0}\NormalTok{, }\FloatTok{15.0}\NormalTok{])}
\end{Highlighting}
\end{Shaded}

It has 2 attributes:

\begin{itemize}
\item
  \texttt{values} : list of float
\item
  \texttt{size} : size of the vector -\textgreater{}
  \texttt{Vector({[}0.0,\ 1.0,\ 2.0,\ 3.0{]}).size\ ==\ 4}
\end{itemize}

You should be able to initialize the object with:

\begin{itemize}
\item
  a list of floats: \texttt{Vector({[}0.0,\ 1.0,\ 2.0,\ 3.0{]})}
\item
  a size \texttt{Vector(3)} -\textgreater{} the vector will have
  \texttt{values\ =\ {[}0.0,\ 1.0,\ 2.0{]}}
\item
  a range or \texttt{Vector((10,15))} -\textgreater{} the vector will
  have \texttt{values\ =\ {[}10.0,\ 11.0,\ 12.0,\ 13.0,\ 14.0{]}}
\end{itemize}

You will implement all the following built-in functions (called `magic
methods') for your \texttt{Vector} class:

\begin{Shaded}
\begin{Highlighting}[]
    \FunctionTok{__add__}
    \FunctionTok{__radd__}
    \CommentTok{# add : scalars and vectors, can have errors with vectors.}
    \FunctionTok{__sub__}
    \FunctionTok{__rsub__}
    \CommentTok{# sub : scalars and vectors, can have errors with vectors.}
    \FunctionTok{__truediv__}
    \FunctionTok{__rtruediv__}
    \CommentTok{# div : only scalars.}
    \FunctionTok{__mul__}
    \FunctionTok{__rmul__}
    \CommentTok{# mul : scalars and vectors, can have errors with vectors, }
    \CommentTok{# return a scalar if we perform Vector * Vector (dot product)}
    \FunctionTok{__str__}
    \FunctionTok{__repr__}
\end{Highlighting}
\end{Shaded}

\hypertarget{vectors-authorized-operations-are}{%
\subsection{Vectors authorized operations
are:}\label{vectors-authorized-operations-are}}

​

\begin{itemize}
\tightlist
\item
  Addition between two vectors of same dimension (m * 1)
\end{itemize}

\large

\[
x + y = 
\begin{bmatrix} x_1 \\ \vdots \\ x_m\end{bmatrix} + 
\begin{bmatrix} y_1 \\ \vdots \\ y_m\end{bmatrix} 
= \begin{bmatrix} x_1 + y_1 \\ \vdots \\ x_m + y_m \end{bmatrix}
\] \normalsize ​

\begin{itemize}
\tightlist
\item
  Substraction between two vectors of same dimension (m * 1)
\end{itemize}

\large

\[
x - y = 
\begin{bmatrix} x_1 \\ \vdots \\ x_m\end{bmatrix} - 
\begin{bmatrix} y_1 \\ \vdots \\ y_m\end{bmatrix} 
= \begin{bmatrix} x_1 - y_1 \\ \vdots \\ x_m - y_m \end{bmatrix}
\] \normalsize ​

\begin{itemize}
\tightlist
\item
  Multiplication and division between one vector (m * 1) and one scalar
  (1 * 1)
\end{itemize}

\large

\[
x \cdot a = \begin{bmatrix} x_1 \\ \vdots \\ x_m\end{bmatrix} 
\cdot a = 
\begin{bmatrix} x_1 \cdot a \\ \vdots \\ x_m \cdot a \end{bmatrix}
\] \normalsize ​

\begin{itemize}
\tightlist
\item
  Mutiplication between two vectors of same dimensons (m * 1)
\end{itemize}

\large

\[
x \cdot y = \begin{bmatrix} x_1 \\ \vdots \\ x_m\end{bmatrix} 
\cdot 
\begin{bmatrix} y_1 \\ \vdots \\ y_m\end{bmatrix} = 
\sum_{i = 1}^{m} x_i \cdot y_i =  x_1 \cdot y_1 + \dots + x_m \cdot y_m 
\] \normalsize

Don't forget to handle all kind of errors properly!

\clearpage

\hypertarget{exercise-03---the-matrix-1}{%
\section{Exercise 03 - The Matrix}\label{exercise-03---the-matrix-1}}

\begin{longtable}[]{@{}rl@{}}
\toprule
\endhead
Turn-in directory : & ex03\tabularnewline
Files to turn in : & matrix.py, test.py\tabularnewline
Forbidden functions : & None\tabularnewline
Forbidden libraries : & NumPy\tabularnewline
Remarks : & n/a\tabularnewline
\bottomrule
\end{longtable}

You will provide a testing file to prove that your class works as
expected.\\
You will have to create a helpful class, with more options and providing
enhanced ease of use for the user.

In this exercise, you have to create a \texttt{Matrix} class. The goal
is to have matrices and be able to perform both matrix-matrix operation
and matrix-vector operations with them.

\begin{Shaded}
\begin{Highlighting}[]
\OperatorTok{>>}\NormalTok{ m1 }\OperatorTok{=}\NormalTok{ Matrix([[}\FloatTok{0.0}\NormalTok{, }\FloatTok{1.0}\NormalTok{, }\FloatTok{2.0}\NormalTok{, }\FloatTok{3.0}\NormalTok{], }
\NormalTok{                [}\FloatTok{0.0}\NormalTok{, }\FloatTok{2.0}\NormalTok{, }\FloatTok{4.0}\NormalTok{, }\FloatTok{6.0}\NormalTok{]])}

\OperatorTok{>>}\NormalTok{ m2 }\OperatorTok{=}\NormalTok{ Matrix([[}\FloatTok{0.0}\NormalTok{, }\FloatTok{1.0}\NormalTok{],}
\NormalTok{                [}\FloatTok{2.0}\NormalTok{, }\FloatTok{3.0}\NormalTok{],}
\NormalTok{                [}\FloatTok{4.0}\NormalTok{, }\FloatTok{5.0}\NormalTok{],}
\NormalTok{                [}\FloatTok{6.0}\NormalTok{, }\FloatTok{7.0}\NormalTok{]])}
\OperatorTok{>>} \BuiltInTok{print}\NormalTok{(m1 }\OperatorTok{*}\NormalTok{ m2)}
\NormalTok{(Matrix [[}\FloatTok{28.}\NormalTok{, }\FloatTok{34.}\NormalTok{], [}\FloatTok{56.}\NormalTok{, }\FloatTok{68.}\NormalTok{]])}
\end{Highlighting}
\end{Shaded}

It has 2 attributes:

\begin{itemize}
\item
  \texttt{data} : list of lists -\textgreater{} the elements stored in
  the matrix
\item
  \texttt{shape} : by shape we means the dimensions of the matrix as a
  tuple (rows, columns) -\textgreater{}
  \texttt{Matrix({[}{[}0.0,\ 1.0{]},\ {[}2.0,\ 3.0{]},\ {[}4.0,\ 5.0{]}{]}).shape\ ==\ (3,\ 2)}
\end{itemize}

You should be able to initialize the object with:

\begin{itemize}
\item
  the elements of the matrix as a list of lists:
  \texttt{Matrix({[}{[}0.0,\ 1.0,\ 2.0,\ 3.0{]},\ {[}4.0,\ 5.0,\ 6.0,\ 7.0{]}{]})}
  -\textgreater{} the dimensions of this matrix are then (2, 4)
\item
  a shape \texttt{Matrix((3,\ 3))} -\textgreater{} the matrix will be
  filled by default with zeroes
\item
  the expected elements and shape
  \texttt{Matrix({[}{[}0.0,\ 1.0,\ 2.0{]},\ {[}3.0,\ 4.0,\ 5.0{]},\ {[}6.0,\ 7.0,\ 8.0{]}{]},\ (3,\ 3))}
\end{itemize}

You will implement all the following built-in functions (called `magic
methods') for your \texttt{Matrix} class:

\begin{Shaded}
\begin{Highlighting}[]
    \FunctionTok{__add__}
    \FunctionTok{__radd__}
    \CommentTok{# add : vectors and matrices, can have errors with vectors and matrices.}
    \FunctionTok{__sub__}
    \FunctionTok{__rsub__}
    \CommentTok{# sub : vectors and matrices, can have errors with vectors and matrices.}
    \FunctionTok{__truediv__}
    \FunctionTok{__rtruediv__}
    \CommentTok{# div : only scalars.}
    \FunctionTok{__mul__}
    \FunctionTok{__rmul__}
    \CommentTok{# mul : scalars, vectors and matrices , can have errors with vectors and matrices, }
    \CommentTok{# return a Vector if we perform Matrix * Vector (dot product)}
    \FunctionTok{__str__}
    \FunctionTok{__repr__}
\end{Highlighting}
\end{Shaded}

\hypertarget{matrix---vector-authorized-operations-are}{%
\subsection{Matrix - vector authorized operations
are:}\label{matrix---vector-authorized-operations-are}}

​

\begin{itemize}
\tightlist
\item
  Multiplication between a (m * n) matrix and a (n * 1) vector
\end{itemize}

\large

\[
X \cdot y = 
\begin{bmatrix} x^{(1)}_1 & \dots& x^{(1)}_n \\ 
\vdots & \ddots & \vdots \\ 
x^{(m)}_1 & \dots & x^{(m)}_n
\end{bmatrix} 
\cdot 
\begin{bmatrix} 
y_1 \\
\vdots \\ 
y_n 
\end{bmatrix} 
= 
\begin{bmatrix} x^{(1)} \cdot y \\ \vdots  \\ x^{(m)} \cdot y \end{bmatrix}
\] \normalsize ​ In other words:

\large

\[
X \cdot y = \begin{bmatrix} \sum_{i = 1}^{n} x_{i}^{(1)} \cdot y_i \\ \vdots \\ \sum_{i = 1}^{n} x_{i}^{(m)} \cdot y_i \end{bmatrix}
\] \normalsize ​

\hypertarget{matrix---matrix-authorized-operations-are}{%
\subsection{Matrix - matrix authorized operations
are:}\label{matrix---matrix-authorized-operations-are}}

​

\begin{itemize}
\tightlist
\item
  Addition between two matrices of same dimension (m * n)
\end{itemize}

\large

\[
X + Y = 
\begin{bmatrix} 
x_{1}^{(1)} & \dots & x_{n}^{(1)}  \\ 
\vdots & \ddots & \vdots \\ 
x_{1}^{(m)} & \dots & x_{n}^{(m)} 
\end{bmatrix} +  
\begin{bmatrix} 
y_{1}^{(1)} & \dots & y_{n}^{(1)}  \\ 
\vdots & \ddots & \vdots \\ 
y_{1}^{(m)} & \dots & y_{n}^{(m)} 
\end{bmatrix} = 
\begin{bmatrix} 
x_{1}^{(1)} + y_{1}^{(1)}  & \dots & x_{n}^{(1)} + y_{n}^{(1)}  \\ 
\vdots & \ddots & \vdots \\ 
x_{1}^{(m)} + y_{1}^{(m)} & \dots & x_{n}^{(m)} + y_{n}^{(m)}
\end{bmatrix}
\] \normalsize ​

\begin{itemize}
\tightlist
\item
  Substraction between two matrices of same dimension (m * n)
\end{itemize}

\large

\[
X - Y = 
\begin{bmatrix} 
x_{1}^{(1)} & \dots & x_{n}^{(1)}  \\ 
\vdots & \ddots & \vdots \\ 
x_{1}^{(m)} & \dots & x_{n}^{(m)} 
\end{bmatrix} - 
\begin{bmatrix} 
y_{1}^{(1)} & \dots & y_{n}^{(1)}  \\ 
\vdots & \ddots & \vdots \\ 
y_{1}^{(m)} & \dots & y_{n}^{(m)} 
\end{bmatrix} = 
\begin{bmatrix} 
x_{1}^{(1)} - y_{1}^{(1)}  & \dots & x_{n}^{(1)} - y_{n}^{(1)}  \\ 
\vdots & \ddots & \vdots \\ 
x_{1}^{(m)} - y_{1}^{(m)} & \dots & x_{n}^{(m)} - y_{n}^{(m)}
\end{bmatrix}
\] \normalsize

​

\begin{itemize}
\tightlist
\item
  Multiplication or division between one matrix (m * n) and one scalar
  (1 * 1)
\end{itemize}

\large

\[
Xa = 
\begin{bmatrix} 
x_{1}^{(1)} & \dots & x_{n}^{(1)}  \\ 
\vdots & \ddots & \vdots \\ 
x_{1}^{(m)} & \dots & x_{n}^{(m)} 
\end{bmatrix} 
\cdot a
= 
\begin{bmatrix} 
x_{1}^{(1)} a  & \dots & x_{n}^{(1)} a  \\ 
\vdots & \ddots & \vdots \\ 
x_{1}^{(m)} a & \dots & x_{n}^{(m)} a
\end{bmatrix}
\] \normalsize

​

\begin{itemize}
\tightlist
\item
  Mutiplication between two matrices of compatible dimension: (m * n)
  and (n * p)
\end{itemize}

\large

\[
X  Y = 
\begin{bmatrix} 
x_{1}^{(1)} & \dots & x_{n}^{(1)}  \\ 
\vdots & \ddots & \vdots \\ 
x_{1}^{(m)} & \dots & x_{n}^{(m)} 
\end{bmatrix}  
\begin{bmatrix} 
y_{1}^{(1)} & \dots & y_{p}^{(1)}  \\ 
\vdots & \ddots & \vdots \\ 
y_{1}^{(n)} & \dots & y_{p}^{(n)} 
\end{bmatrix} = 
\begin{bmatrix} 
x^{(1)} \cdot y_1  & \dots & x^{(1)} \cdot y_{p} \\ 
\vdots & \ddots & \vdots \\ 
x^{(m)} \cdot y_1 & \dots & x^{(m)} \cdot y_{p}
\end{bmatrix}
\] \normalsize

In other words: ​ \large \[
X \cdot Y = 
\begin{bmatrix} 
\sum_{i = 1}^{n} x_{i}^{(1)} \cdot y_{1}^{(i)} & \dots & \sum_{i=1}^{n} x_{i}^{(1)} \cdot y_{p}^{(i)} \\
\vdots & \ddots & \vdots \\ 
\sum_{i = 1}^{n} x_{i}^{(m)} \cdot y_{1}^{(i)} & \dots & \sum_{i=1}^{n} x_{i}^{(m)} \cdot y_{p}^{(i)} \\
\end{bmatrix}
\] \normalsize

Don't forget to handle all kind of errors properly!

\clearpage

\hypertarget{exercise-04---generator-1}{%
\section{Exercise 04 - Generator!}\label{exercise-04---generator-1}}

\begin{longtable}[]{@{}rl@{}}
\toprule
\endhead
Turn-in directory : & ex04\tabularnewline
Files to turn in : & generator.py\tabularnewline
Forbidden functions : & random\tabularnewline
Remarks : & n/a\tabularnewline
\bottomrule
\end{longtable}

Code a function called \texttt{generator} that takes a text as input,
uses the string \texttt{sep} as a splitting parameter, and
\texttt{yield}s the resulting substrings.

The function can take an optional argument.\\
The options are:

\begin{itemize}
\item
  ``shuffle'': shuffle the list of words.
\item
  ``unique'': return a list where each word appears only once.
\item
  ``ordered'': alphabetically sort the words.
\end{itemize}

\begin{Shaded}
\begin{Highlighting}[]
\CommentTok{# function prototype}
\KeywordTok{def}\NormalTok{ generator(text, sep}\OperatorTok{=}\StringTok{" "}\NormalTok{, option}\OperatorTok{=}\VariableTok{None}\NormalTok{):}
    \CommentTok{'''Option is an optional arg, sep is mandatory'''}
\end{Highlighting}
\end{Shaded}

You can only call one option at a time.

\begin{Shaded}
\begin{Highlighting}[]
\OperatorTok{>>}\NormalTok{ text }\OperatorTok{=} \StringTok{"Le Lorem Ipsum est simplement du faux texte."}
\OperatorTok{>>} \ControlFlowTok{for}\NormalTok{ word }\KeywordTok{in}\NormalTok{ generator(text, sep}\OperatorTok{=}\StringTok{" "}\NormalTok{):}
\NormalTok{...     }\BuiltInTok{print}\NormalTok{(word)}
\NormalTok{...}
\NormalTok{Le}
\NormalTok{Lorem}
\NormalTok{Ipsum}
\NormalTok{est}
\NormalTok{simplement}
\NormalTok{du}
\NormalTok{faux}
\NormalTok{texte.}
\OperatorTok{>>} \ControlFlowTok{for}\NormalTok{ word }\KeywordTok{in}\NormalTok{ generator(text, sep}\OperatorTok{=}\StringTok{" "}\NormalTok{, option}\OperatorTok{=}\StringTok{"shuffle"}\NormalTok{):}
\NormalTok{...     }\BuiltInTok{print}\NormalTok{(word)}
\NormalTok{...}
\NormalTok{simplement}
\NormalTok{texte.}
\NormalTok{est}
\NormalTok{faux}
\NormalTok{Le}
\NormalTok{Lorem}
\NormalTok{Ipsum}
\NormalTok{du}
\OperatorTok{>>} \ControlFlowTok{for}\NormalTok{ word }\KeywordTok{in}\NormalTok{ generator(text, sep}\OperatorTok{=}\StringTok{" "}\NormalTok{, option}\OperatorTok{=}\StringTok{"ordered"}\NormalTok{):}
\NormalTok{...     }\BuiltInTok{print}\NormalTok{(word)}
\NormalTok{...}
\NormalTok{Ipsum}
\NormalTok{Le}
\NormalTok{Lorem}
\NormalTok{du}
\NormalTok{est}
\NormalTok{faux}
\NormalTok{simplement}
\NormalTok{texte.}
\end{Highlighting}
\end{Shaded}

The function should return ``ERROR'' one time if the \texttt{text}
argument is not a string, or if the \texttt{option} argument is not
valid.

\clearpage

\hypertarget{exercise-05---working-with-lists-1}{%
\section{Exercise 05 - Working with
lists}\label{exercise-05---working-with-lists-1}}

\begin{longtable}[]{@{}rl@{}}
\toprule
\endhead
Turn-in directory : & ex05\tabularnewline
Files to turn in : & eval.py\tabularnewline
Forbidden functions : & while\tabularnewline
Remarks : & use zip \& enumerate\tabularnewline
\bottomrule
\end{longtable}

Code a class \texttt{Evaluator}, that has two static functions named:
\texttt{zip\_evaluate} and \texttt{enumerate\_evaluate}.

The goal of these 2 functions is to compute the sum of the lengths of
every \texttt{words} of a given list weighted by a list a
\texttt{coefs}.

The lists \texttt{coefs} and \texttt{words} have to be the same length.
If this is not the case, the function should return -1.

You have to obtain the desired result using \texttt{zip} in the
\texttt{zip\_evaluate} function, and with \texttt{enumerate} in the
\texttt{enumerate\_evaluate} function.

\begin{Shaded}
\begin{Highlighting}[]
\OperatorTok{>>} \ImportTok{from} \BuiltInTok{eval} \ImportTok{import}\NormalTok{ Evaluator}
\OperatorTok{>>} 
\OperatorTok{>>}\NormalTok{ words }\OperatorTok{=}\NormalTok{ [}\StringTok{"Le"}\NormalTok{, }\StringTok{"Lorem"}\NormalTok{, }\StringTok{"Ipsum"}\NormalTok{, }\StringTok{"est"}\NormalTok{, }\StringTok{"simple"}\NormalTok{]}
\OperatorTok{>>}\NormalTok{ coefs }\OperatorTok{=}\NormalTok{ [}\FloatTok{1.0}\NormalTok{, }\FloatTok{2.0}\NormalTok{, }\FloatTok{1.0}\NormalTok{, }\FloatTok{4.0}\NormalTok{, }\FloatTok{0.5}\NormalTok{]}
\OperatorTok{>>}\NormalTok{ Evaluator.zip_evaluate(coefs, words)}
\FloatTok{32.0}
\OperatorTok{>>}\NormalTok{ words }\OperatorTok{=}\NormalTok{ [}\StringTok{"Le"}\NormalTok{, }\StringTok{"Lorem"}\NormalTok{, }\StringTok{"Ipsum"}\NormalTok{, }\StringTok{"n'"}\NormalTok{, }\StringTok{"est"}\NormalTok{, }\StringTok{"pas"}\NormalTok{, }\StringTok{"simple"}\NormalTok{]}
\OperatorTok{>>}\NormalTok{ coefs }\OperatorTok{=}\NormalTok{ [}\FloatTok{0.0}\NormalTok{, }\FloatTok{-1.0}\NormalTok{, }\FloatTok{1.0}\NormalTok{, }\FloatTok{-12.0}\NormalTok{, }\FloatTok{0.0}\NormalTok{, }\FloatTok{42.42}\NormalTok{]}
\OperatorTok{>>}\NormalTok{ Evaluator.enumerate_evaluate(coefs, words)}
\DecValTok{-1}
\end{Highlighting}
\end{Shaded}

\clearpage

\hypertarget{exercise-06---bank-account-1}{%
\section{Exercise 06 - Bank
Account}\label{exercise-06---bank-account-1}}

\begin{longtable}[]{@{}rl@{}}
\toprule
\endhead
Turn-in directory : & ex06\tabularnewline
Files to turn in : & the\_bank.py\tabularnewline
Forbidden functions : & None\tabularnewline
Remarks : & n/a\tabularnewline
\bottomrule
\end{longtable}

It's all about security.\\
Have a look at the class named \texttt{Account} in the snippet of code
below.

\begin{Shaded}
\begin{Highlighting}[]
\CommentTok{# in the_bank.py}
\KeywordTok{class}\NormalTok{ Account(}\BuiltInTok{object}\NormalTok{):}

\NormalTok{    ID_COUNT }\OperatorTok{=} \DecValTok{1}

    \KeywordTok{def} \FunctionTok{__init__}\NormalTok{(}\VariableTok{self}\NormalTok{, name, }\OperatorTok{**}\NormalTok{kwargs):}
        \VariableTok{self}\NormalTok{.}\BuiltInTok{id} \OperatorTok{=} \VariableTok{self}\NormalTok{.ID_COUNT}
        \VariableTok{self}\NormalTok{.name }\OperatorTok{=}\NormalTok{ name}
        \VariableTok{self}\NormalTok{.__dict__.update(kwargs)}
        \ControlFlowTok{if} \BuiltInTok{hasattr}\NormalTok{(}\VariableTok{self}\NormalTok{, }\StringTok{'value'}\NormalTok{):}
            \VariableTok{self}\NormalTok{.value }\OperatorTok{=} \DecValTok{0}
\NormalTok{        Account.ID_COUNT }\OperatorTok{+=} \DecValTok{1}
    
    \KeywordTok{def}\NormalTok{ transfer(}\VariableTok{self}\NormalTok{, amount):}
        \VariableTok{self}\NormalTok{.value }\OperatorTok{+=}\NormalTok{ amount}
\end{Highlighting}
\end{Shaded}

Now, it is your turn to code a class named \texttt{Bank}!\\
Its purpose will be to handle the security part of each transfer
attempt.\\
Security means checking if the \texttt{Account} is:

\begin{itemize}
\item
  the right object
\item
  that it is not corrupted
\item
  and that it has enough money
\end{itemize}

How do we define if a bank account is corrupted?

\begin{itemize}
\item
  It has an even number of attributes.
\item
  It has an attribute starting with \texttt{b}.
\item
  It has no attribute starting with \texttt{zip} or \texttt{addr}.
\item
  It has no attribute \texttt{name}, \texttt{id} and \texttt{value}.
\end{itemize}

A transaction is invalid if \texttt{amount\ \textless{}\ 0} or if the
amount is larger than the funds the first account has available for
transfer.

\begin{Shaded}
\begin{Highlighting}[]
\CommentTok{# in the_bank.py}
\KeywordTok{class}\NormalTok{ Bank(}\BuiltInTok{object}\NormalTok{):}
    \CommentTok{"""The bank"""}
    \KeywordTok{def} \FunctionTok{__init__}\NormalTok{(}\VariableTok{self}\NormalTok{):}
        \VariableTok{self}\NormalTok{.account }\OperatorTok{=}\NormalTok{ []}

    \KeywordTok{def}\NormalTok{ add(}\VariableTok{self}\NormalTok{, account):}
        \VariableTok{self}\NormalTok{.account.append(account)}

    \KeywordTok{def}\NormalTok{ transfer(}\VariableTok{self}\NormalTok{, origin, dest, amount):}
        \CommentTok{"""}
\CommentTok{            @origin:  int(id) or str(name) of the first account}
\CommentTok{            @dest:    int(id) or str(name) of the destination account}
\CommentTok{            @amount:  float(amount) amount to transfer}
\CommentTok{            @return         True if success, False if an error occured}
\CommentTok{        """}

    \KeywordTok{def}\NormalTok{ fix_account(}\VariableTok{self}\NormalTok{, account):}
        \CommentTok{"""}
\CommentTok{            fix the corrupted account}
\CommentTok{            @account: int(id) or str(name) of the account}
\CommentTok{            @return         True if success, False if an error occured}
\CommentTok{        """}
\end{Highlighting}
\end{Shaded}

Check out the \texttt{dir} function.

WARNING: YOU WILL HAVE TO MODIFY THE INSTANCES' ATTRIBUTES IN ORDER TO
FIX THEM.

\clearpage

\end{document}
