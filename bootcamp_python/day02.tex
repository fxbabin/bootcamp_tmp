\PassOptionsToPackage{unicode=true}{hyperref} % options for packages loaded elsewhere
\PassOptionsToPackage{hyphens}{url}
\PassOptionsToPackage{dvipsnames,svgnames*,x11names*}{xcolor}
%
\documentclass[]{article}
\usepackage{lmodern}
\usepackage{amssymb,amsmath}
\usepackage{ifxetex,ifluatex}
\usepackage{fixltx2e} % provides \textsubscript
\ifnum 0\ifxetex 1\fi\ifluatex 1\fi=0 % if pdftex
  \usepackage[T1]{fontenc}
  \usepackage[utf8x]{inputenc}
  \usepackage{textcomp} % provides euro and other symbols
\else % if luatex or xelatex
  \usepackage{unicode-math}
  \defaultfontfeatures{Ligatures=TeX,Scale=MatchLowercase}
\fi
% use upquote if available, for straight quotes in verbatim environments
\IfFileExists{upquote.sty}{\usepackage{upquote}}{}
% use microtype if available
\IfFileExists{microtype.sty}{%
\usepackage[]{microtype}
\UseMicrotypeSet[protrusion]{basicmath} % disable protrusion for tt fonts
}{}
\IfFileExists{parskip.sty}{%
\usepackage{parskip}
}{% else
\setlength{\parindent}{0pt}
\setlength{\parskip}{6pt plus 2pt minus 1pt}
}
\usepackage{xcolor}
\usepackage{hyperref}
\hypersetup{
            colorlinks=true,
            linkcolor=Maroon,
            citecolor=Blue,
            urlcolor=blue,
            breaklinks=true}
\urlstyle{same}  % don't use monospace font for urls
\usepackage{color}
\usepackage{fancyvrb}
\newcommand{\VerbBar}{|}
\newcommand{\VERB}{\Verb[commandchars=\\\{\}]}
\DefineVerbatimEnvironment{Highlighting}{Verbatim}{commandchars=\\\{\}}
% Add ',fontsize=\small' for more characters per line
\usepackage{framed}
\definecolor{shadecolor}{RGB}{35,38,41}
\newenvironment{Shaded}{\begin{snugshade}}{\end{snugshade}}
\newcommand{\AlertTok}[1]{\textcolor[rgb]{0.58,0.85,0.30}{#1}}
\newcommand{\AnnotationTok}[1]{\textcolor[rgb]{0.25,0.50,0.35}{#1}}
\newcommand{\AttributeTok}[1]{\textcolor[rgb]{0.16,0.50,0.73}{#1}}
\newcommand{\BaseNTok}[1]{\textcolor[rgb]{0.96,0.45,0.00}{#1}}
\newcommand{\BuiltInTok}[1]{\textcolor[rgb]{0.50,0.55,0.55}{#1}}
\newcommand{\CharTok}[1]{\textcolor[rgb]{0.24,0.68,0.91}{#1}}
\newcommand{\CommentTok}[1]{\textcolor[rgb]{0.48,0.49,0.49}{#1}}
\newcommand{\CommentVarTok}[1]{\textcolor[rgb]{0.50,0.55,0.55}{#1}}
\newcommand{\ConstantTok}[1]{\textcolor[rgb]{0.15,0.68,0.68}{#1}}
\newcommand{\ControlFlowTok}[1]{\textcolor[rgb]{0.99,0.74,0.29}{#1}}
\newcommand{\DataTypeTok}[1]{\textcolor[rgb]{0.16,0.50,0.73}{#1}}
\newcommand{\DecValTok}[1]{\textcolor[rgb]{0.96,0.45,0.00}{#1}}
\newcommand{\DocumentationTok}[1]{\textcolor[rgb]{0.64,0.20,0.25}{#1}}
\newcommand{\ErrorTok}[1]{\textcolor[rgb]{0.85,0.27,0.33}{#1}}
\newcommand{\ExtensionTok}[1]{\textcolor[rgb]{0.00,0.60,1.00}{#1}}
\newcommand{\FloatTok}[1]{\textcolor[rgb]{0.96,0.45,0.00}{#1}}
\newcommand{\FunctionTok}[1]{\textcolor[rgb]{0.56,0.27,0.68}{#1}}
\newcommand{\ImportTok}[1]{\textcolor[rgb]{0.15,0.68,0.38}{#1}}
\newcommand{\InformationTok}[1]{\textcolor[rgb]{0.77,0.36,0.00}{#1}}
\newcommand{\KeywordTok}[1]{\textcolor[rgb]{0.81,0.81,0.76}{#1}}
\newcommand{\NormalTok}[1]{\textcolor[rgb]{0.81,0.81,0.76}{#1}}
\newcommand{\OperatorTok}[1]{\textcolor[rgb]{0.81,0.81,0.76}{#1}}
\newcommand{\OtherTok}[1]{\textcolor[rgb]{0.15,0.68,0.38}{#1}}
\newcommand{\PreprocessorTok}[1]{\textcolor[rgb]{0.15,0.68,0.38}{#1}}
\newcommand{\RegionMarkerTok}[1]{\textcolor[rgb]{0.16,0.50,0.73}{#1}}
\newcommand{\SpecialCharTok}[1]{\textcolor[rgb]{0.24,0.68,0.91}{#1}}
\newcommand{\SpecialStringTok}[1]{\textcolor[rgb]{0.85,0.27,0.33}{#1}}
\newcommand{\StringTok}[1]{\textcolor[rgb]{0.96,0.31,0.31}{#1}}
\newcommand{\VariableTok}[1]{\textcolor[rgb]{0.15,0.68,0.68}{#1}}
\newcommand{\VerbatimStringTok}[1]{\textcolor[rgb]{0.85,0.27,0.33}{#1}}
\newcommand{\WarningTok}[1]{\textcolor[rgb]{0.85,0.27,0.33}{#1}}
\usepackage{longtable,booktabs}
% Fix footnotes in tables (requires footnote package)
\IfFileExists{footnote.sty}{\usepackage{footnote}\makesavenoteenv{longtable}}{}
\setlength{\emergencystretch}{3em}  % prevent overfull lines
\providecommand{\tightlist}{%
  \setlength{\itemsep}{0pt}\setlength{\parskip}{0pt}}
\setcounter{secnumdepth}{0}
% Redefines (sub)paragraphs to behave more like sections
\ifx\paragraph\undefined\else
\let\oldparagraph\paragraph
\renewcommand{\paragraph}[1]{\oldparagraph{#1}\mbox{}}
\fi
\ifx\subparagraph\undefined\else
\let\oldsubparagraph\subparagraph
\renewcommand{\subparagraph}[1]{\oldsubparagraph{#1}\mbox{}}
\fi

% set default figure placement to htbp
\makeatletter
\def\fps@figure{h}
\makeatother


\date{}

%%%%%%%%%%%%%%%%%%%%%%%%%%%%%%%%%%%%%%%%%%%%%%%%%%%%%%%%%%%%%%%%%%%%%%%%%%%%%%%%
%%%%%%%%%%%%%%%%%%%%%%%%%%%%%%%% Added packages %%%%%%%%%%%%%%%%%%%%%%%%%%%%%%%%
%%%%%%%%%%%%%%%%%%%%%%%%%%%%%%%%%%%%%%%%%%%%%%%%%%%%%%%%%%%%%%%%%%%%%%%%%%%%%%%%

\setcounter{MaxMatrixCols}{20}
\usepackage{cancel}
\usepackage{calc}
\usepackage{eso-pic}
\newlength{\PageFrameTopMargin}
\newlength{\PageFrameBottomMargin}
\newlength{\PageFrameLeftMargin}
\newlength{\PageFrameRightMargin}

\setlength{\PageFrameTopMargin}{1.5cm}
\setlength{\PageFrameBottomMargin}{1cm}
\setlength{\PageFrameLeftMargin}{1cm}
\setlength{\PageFrameRightMargin}{1cm}

\makeatletter

\newlength{\Page@FrameHeight}
\newlength{\Page@FrameWidth}

\AddToShipoutPicture{
  \thinlines
  \setlength{\Page@FrameHeight}{\paperheight-\PageFrameTopMargin-\PageFrameBottomMargin}
  \setlength{\Page@FrameWidth}{\paperwidth-\PageFrameLeftMargin-\PageFrameRightMargin}
  \put(\strip@pt\PageFrameLeftMargin,\strip@pt\PageFrameTopMargin){
    \framebox(\strip@pt\Page@FrameWidth, \strip@pt\Page@FrameHeight){}}}

\makeatother

\usepackage{fvextra}
\DefineVerbatimEnvironment{Highlighting}{Verbatim}{breaklines,breakanywhere,commandchars=\\\{\}}

\usepackage{graphicx}

\usepackage[a4paper, total={6in, 8in}]{geometry}
\geometry{hmargin=2cm,vmargin=2cm}

\usepackage{sectsty}

\sectionfont{\centering\Huge}
\subsectionfont{\Large}
\subsubsectionfont{\large}

\usepackage{titlesec}
\titlespacing*{\section}
{0pt}{5.5ex plus 1ex minus .2ex}{4.3ex plus .2ex}

\tolerance=1
\emergencystretch=\maxdimen
\hyphenpenalty=10000
\hbadness=10000

%%%%%%%%%%%%%%%%%%%%%%%%%%%%%%%%%%%%%%%%%%%%%%%%%%%%%%%%%%%%%%%%%%%%%%%%%%%%%%%%
%%%%%%%%%%%%%%%%%%%%%%%%%%%%%%%%%%%%%%%%%%%%%%%%%%%%%%%%%%%%%%%%%%%%%%%%%%%%%%%%

\begin{document}

%%%%%%%%%%%%%%%%%%%%%%%%%%%%%%%%%%%%%%%%%%%%%%%%%%%%%%%%%%%%%%%%%%%%%%%%%%%%%%%%
%%%%%%%%%%%%%%%%%%%%%%%%%%%%%%%% Added lines %%%%%%%%%%%%%%%%%%%%%%%%%%%%%%%%%%%
%%%%%%%%%%%%%%%%%%%%%%%%%%%%%%%%%%%%%%%%%%%%%%%%%%%%%%%%%%%%%%%%%%%%%%%%%%%%%%%%

\vspace*{2cm}
\begin{center}
    \textsc{\fontsize{40}{48} \bfseries }\\[0.6cm]
    \textsc{\fontsize{39}{48} \bfseries { %bootcamp_title
Python Machine-Learning
    }}\\[0.3cm]
\end{center}
\vspace{3cm}

\begin{center}
\includegraphics[width=200pt]{assets/logo-42-ai.png}{\centering}
\end{center}

\vspace*{2cm}
\begin{center}
    \textsc{\fontsize{32}{48} \bfseries %day_number
Module 02    
    }\\[0.6cm]
    \textsc{\fontsize{32}{48} \bfseries %day_title
Basics 3    
    }\\[0.3cm]
\end{center}
\vspace{3cm}

\pagenumbering{gobble}
\newpage

%%% >>>>> Page de garde
\setcounter{page}{1}
\pagenumbering{arabic}

%%%%%%%%%%%%%%%%%%%%%%%%%%%%%%%%%%%%%%%%%%%%%%%%%%%%%%%%%%%%%%%%%%%%%%%%%%%%%%%%
%%%%%%%%%%%%%%%%%%%%%%%%%%%%%%%%%%%%%%%%%%%%%%%%%%%%%%%%%%%%%%%%%%%%%%%%%%%%%%%%


\hypertarget{module-02---basics-3}{%
\section{Module 02 - Basics 3}\label{module-02---basics-3}}

Let's continue practicing with more advanced Python programming
exercises.

\hypertarget{notions-of-the-day}{%
\subsection{Notions of the day}\label{notions-of-the-day}}

Decorators, multiprocessing, lambda, build package, \ldots{}

\hypertarget{general-rules}{%
\subsection{General rules}\label{general-rules}}

\begin{itemize}
\item
  The version of Python to use is 3.7, you can check the version of
  Python with the following command: \texttt{python\ -V}
\item
  The norm: during this module you will follow the
  \href{https://www.python.org/dev/peps/pep-0008/}{PEP 8 standards}. You
  can install \href{https://pypi.org/project/pycodestyle}{pycodestyle}
  which is a tool to check your Python code.
\item
  The function eval is never allowed.
\item
  The exercises are ordered from the easiest to the hardest.
\item
  Your exercises are going to be evaluated by someone else, so make sure
  that your variable names and function names are appropriate and civil.
\item
  Your manual is the internet.
\item
  You can also ask questions in the dedicated channel in the 42 AI
  Slack: 42-ai.slack.com.
\item
  If you find any issue or mistakes in the subject please create an
  issue on our
  \href{https://github.com/42-AI/bootcamp_python/issues}{dedicated
  repository on Github}.
\end{itemize}

\hypertarget{helper}{%
\subsection{Helper}\label{helper}}

Ensure that you have the right Python version.

\begin{Shaded}
\begin{Highlighting}[]
\NormalTok{> which python}
\NormalTok{/goinfre/miniconda/bin/python}
\NormalTok{> python -V}
\NormalTok{Python 3.7.*}
\NormalTok{> which pip}
\NormalTok{/goinfre/miniconda/bin/pip}
\end{Highlighting}
\end{Shaded}

\hypertarget{exercise-00---map-filter-reduce}{%
\subsubsection{Exercise 00 - Map, filter,
reduce}\label{exercise-00---map-filter-reduce}}

\hypertarget{exercise-01---args-and-kwargs}{%
\subsubsection{Exercise 01 - args and
kwargs?}\label{exercise-01---args-and-kwargs}}

\hypertarget{exercise-02---the-logger}{%
\subsubsection{Exercise 02 - The
logger}\label{exercise-02---the-logger}}

\hypertarget{exercise-03---json-issues}{%
\subsubsection{Exercise 03 - Json
issues}\label{exercise-03---json-issues}}

\hypertarget{exercise-04---minipack}{%
\subsubsection{Exercise 04 - MiniPack}\label{exercise-04---minipack}}

\hypertarget{exercise-05---tinystatistician}{%
\subsubsection{Exercise 05 -
TinyStatistician}\label{exercise-05---tinystatistician}}

\clearpage

\hypertarget{exercise-00---map-filter-reduce-1}{%
\section{Exercise 00 - Map, filter,
reduce}\label{exercise-00---map-filter-reduce-1}}

\begin{longtable}[]{@{}rl@{}}
\toprule
\endhead
Turn-in directory: & ex00\tabularnewline
Files to turn in: & ft\_map.py, ft\_filter.py,
ft\_reduce.py\tabularnewline
Forbidden functions: & map, filter, reduce\tabularnewline
Remarks: & n/a\tabularnewline
\bottomrule
\end{longtable}

Implement the higher order functions \texttt{map()}, \texttt{filter()}
and \texttt{reduce()}. Take the time to understand the use cases of
these three built-in functions.

How they should be prototyped:

\begin{Shaded}
\begin{Highlighting}[]
\KeywordTok{def}\NormalTok{ ft_map(function_to_apply, list_of_inputs):}
    \ControlFlowTok{pass}

\KeywordTok{def}\NormalTok{ ft_filter(function_to_apply, list_of_inputs):}
    \ControlFlowTok{pass}

\KeywordTok{def}\NormalTok{ ft_reduce(function_to_apply, list_of_inputs):}
    \ControlFlowTok{pass}
\end{Highlighting}
\end{Shaded}

\clearpage

\hypertarget{exercise-01---args-and-kwargs-1}{%
\section{Exercise 01 - args and
kwargs?}\label{exercise-01---args-and-kwargs-1}}

\begin{longtable}[]{@{}rl@{}}
\toprule
\endhead
Turn-in directory: & ex01\tabularnewline
Files to turn in: & main.py\tabularnewline
Forbidden functions: & None\tabularnewline
Remarks: & n/a\tabularnewline
\bottomrule
\end{longtable}

Implement the \texttt{what\_are\_the\_vars} function that returns an
object with the right attributes.\\
You will have to modify the `instance' \texttt{ObjectC}, NOT the
class.\\
Have a look to \texttt{getattr}, \texttt{setattr}.

\begin{Shaded}
\begin{Highlighting}[]
\KeywordTok{def}\NormalTok{ what_are_the_vars(...):}
    \CommentTok{"""Your code"""}
    \ControlFlowTok{pass}

\KeywordTok{class}\NormalTok{ ObjectC(}\BuiltInTok{object}\NormalTok{):}
    \KeywordTok{def} \FunctionTok{__init__}\NormalTok{(}\VariableTok{self}\NormalTok{):}
        \ControlFlowTok{pass}

\KeywordTok{def}\NormalTok{ doom_printer(obj):}
    \ControlFlowTok{if}\NormalTok{ obj }\KeywordTok{is} \VariableTok{None}\NormalTok{:}
        \BuiltInTok{print}\NormalTok{(}\StringTok{"ERROR"}\NormalTok{)}
        \BuiltInTok{print}\NormalTok{(}\StringTok{"end"}\NormalTok{)}
        \ControlFlowTok{return}
    \ControlFlowTok{for}\NormalTok{ attr }\KeywordTok{in} \BuiltInTok{dir}\NormalTok{(obj):}
        \ControlFlowTok{if}\NormalTok{ attr[}\DecValTok{0}\NormalTok{] }\OperatorTok{!=} \StringTok{'_'}\NormalTok{:}
\NormalTok{            value }\OperatorTok{=} \BuiltInTok{getattr}\NormalTok{(obj, attr)}
            \BuiltInTok{print}\NormalTok{(}\StringTok{"}\SpecialCharTok{\{\}}\StringTok{: }\SpecialCharTok{\{\}}\StringTok{"}\NormalTok{.}\BuiltInTok{format}\NormalTok{(attr, value))}
    \BuiltInTok{print}\NormalTok{(}\StringTok{"end"}\NormalTok{)}

\ControlFlowTok{if} \VariableTok{__name__} \OperatorTok{==} \StringTok{"__main__"}\NormalTok{:}
\NormalTok{    obj }\OperatorTok{=}\NormalTok{ what_are_the_vars(}\DecValTok{7}\NormalTok{)}
\NormalTok{    doom_printer(obj)}
\NormalTok{    obj }\OperatorTok{=}\NormalTok{ what_are_the_vars(}\StringTok{"ft_lol"}\NormalTok{, }\StringTok{"Hi"}\NormalTok{)}
\NormalTok{    doom_printer(obj)}
\NormalTok{    obj }\OperatorTok{=}\NormalTok{ what_are_the_vars()}
\NormalTok{    doom_printer(obj)}
\NormalTok{    obj }\OperatorTok{=}\NormalTok{ what_are_the_vars(}\DecValTok{12}\NormalTok{, }\StringTok{"Yes"}\NormalTok{, [}\DecValTok{0}\NormalTok{, }\DecValTok{0}\NormalTok{, }\DecValTok{0}\NormalTok{], a}\OperatorTok{=}\DecValTok{10}\NormalTok{, hello}\OperatorTok{=}\StringTok{"world"}\NormalTok{)}
\NormalTok{    doom_printer(obj)}
\NormalTok{    obj }\OperatorTok{=}\NormalTok{ what_are_the_vars(}\DecValTok{42}\NormalTok{, a}\OperatorTok{=}\DecValTok{10}\NormalTok{, var_0}\OperatorTok{=}\StringTok{"world"}\NormalTok{)}
\NormalTok{    doom_printer(obj)}
\end{Highlighting}
\end{Shaded}

\textbf{Example:}

\begin{Shaded}
\begin{Highlighting}[]
\NormalTok{>> python main.py}
\NormalTok{var_0: 7}
\NormalTok{end}
\NormalTok{var_0: ft_lol}
\NormalTok{var_1: Hi}
\NormalTok{end}
\NormalTok{end}
\NormalTok{a: 10}
\NormalTok{hello: world}
\NormalTok{var_0: 12}
\NormalTok{var_1: Yes}
\NormalTok{var_2: [0, 0, 0]}
\NormalTok{end}
\NormalTok{ERROR}
\NormalTok{end}
\end{Highlighting}
\end{Shaded}

\clearpage

\hypertarget{exercise-02---the-logger-1}{%
\section{Exercise 02 - The logger}\label{exercise-02---the-logger-1}}

\begin{longtable}[]{@{}rl@{}}
\toprule
\endhead
Turn-in directory: & ex02\tabularnewline
Files to turn in: & logger.py\tabularnewline
Forbidden functions: & None\tabularnewline
Remarks: & n/a\tabularnewline
\bottomrule
\end{longtable}

You are going to learn some more advanced features of Python!\\
In this exercise, we want you to learn about decorators, and we are not
talking about the decoration of your room.\\
The \texttt{@log} will write info about the decorated function in a
\texttt{machine.log} file.

\begin{Shaded}
\begin{Highlighting}[]
\ImportTok{import}\NormalTok{ time}
\ImportTok{from}\NormalTok{ random }\ImportTok{import}\NormalTok{ randint}

\KeywordTok{class}\NormalTok{ CoffeeMachine():}

\NormalTok{    water_level }\OperatorTok{=} \DecValTok{100}

    \AttributeTok{@log}
    \KeywordTok{def}\NormalTok{ start_machine(}\VariableTok{self}\NormalTok{):}
      \ControlFlowTok{if} \VariableTok{self}\NormalTok{.water_level }\OperatorTok{>} \DecValTok{20}\NormalTok{:}
          \ControlFlowTok{return} \VariableTok{True}
      \ControlFlowTok{else}\NormalTok{:}
          \BuiltInTok{print}\NormalTok{(}\StringTok{"Please add water!"}\NormalTok{)}
          \ControlFlowTok{return} \VariableTok{False}
    
    \AttributeTok{@log}
    \KeywordTok{def}\NormalTok{ boil_water(}\VariableTok{self}\NormalTok{):}
        \ControlFlowTok{return} \StringTok{"boiling..."}
    
    \AttributeTok{@log}
    \KeywordTok{def}\NormalTok{ make_coffee(}\VariableTok{self}\NormalTok{):}
        \ControlFlowTok{if} \VariableTok{self}\NormalTok{.start_machine():}
            \ControlFlowTok{for}\NormalTok{ _ }\KeywordTok{in} \BuiltInTok{range}\NormalTok{(}\DecValTok{20}\NormalTok{):}
\NormalTok{                time.sleep(}\FloatTok{0.1}\NormalTok{)}
                \VariableTok{self}\NormalTok{.water_level }\OperatorTok{-=} \DecValTok{1}
            \BuiltInTok{print}\NormalTok{(}\VariableTok{self}\NormalTok{.boil_water())}
            \BuiltInTok{print}\NormalTok{(}\StringTok{"Coffee is ready!"}\NormalTok{)}
    
    \AttributeTok{@log}
    \KeywordTok{def}\NormalTok{ add_water(}\VariableTok{self}\NormalTok{, water_level):}
\NormalTok{        time.sleep(randint(}\DecValTok{1}\NormalTok{, }\DecValTok{5}\NormalTok{))}
        \VariableTok{self}\NormalTok{.water_level }\OperatorTok{+=}\NormalTok{ water_level}
        \BuiltInTok{print}\NormalTok{(}\StringTok{"Blub blub blub..."}\NormalTok{)}


\ControlFlowTok{if} \VariableTok{__name__} \OperatorTok{==} \StringTok{"__main__"}\NormalTok{:}
    
\NormalTok{    machine }\OperatorTok{=}\NormalTok{ CoffeeMachine()}
    \ControlFlowTok{for}\NormalTok{ i }\KeywordTok{in} \BuiltInTok{range}\NormalTok{(}\DecValTok{0}\NormalTok{, }\DecValTok{5}\NormalTok{):}
\NormalTok{        machine.make_coffee()}

\NormalTok{    machine.make_coffee()}
\NormalTok{    machine.add_water(}\DecValTok{70}\NormalTok{)}
\end{Highlighting}
\end{Shaded}

\textbf{Example:}

\begin{Shaded}
\begin{Highlighting}[]
\NormalTok{boiling...}
\NormalTok{Coffee is ready!}
\NormalTok{boiling...}
\NormalTok{Coffee is ready!}
\NormalTok{boiling...}
\NormalTok{Coffee is ready!}
\NormalTok{boiling...}
\NormalTok{Coffee is ready!}
\NormalTok{Please add water!}
\NormalTok{Please add water!}
\NormalTok{Blub blub blub...}
\end{Highlighting}
\end{Shaded}

\begin{Shaded}
\begin{Highlighting}[]
\NormalTok{> cat machine.log}
\NormalTok{(cmaxime)Running: Start Machine     [ exec-time = 0.001 ms ]}
\NormalTok{(cmaxime)Running: Boil Water        [ exec-time = 0.005 ms ]}
\NormalTok{(cmaxime)Running: Make Coffee       [ exec-time = 2.499 s  ]}
\NormalTok{(cmaxime)Running: Start Machine     [ exec-time = 0.002 ms ]}
\NormalTok{(cmaxime)Running: Boil Water        [ exec-time = 0.005 ms ]}
\NormalTok{(cmaxime)Running: Make Coffee       [ exec-time = 2.618 s  ]}
\NormalTok{(cmaxime)Running: Start Machine     [ exec-time = 0.003 ms ]}
\NormalTok{(cmaxime)Running: Boil Water        [ exec-time = 0.004 ms ]}
\NormalTok{(cmaxime)Running: Make Coffee       [ exec-time = 2.676 s  ]}
\NormalTok{(cmaxime)Running: Start Machine     [ exec-time = 0.003 ms ]}
\NormalTok{(cmaxime)Running: Boil Water        [ exec-time = 0.004 ms ]}
\NormalTok{(cmaxime)Running: Make Coffee       [ exec-time = 2.648 s  ]}
\NormalTok{(cmaxime)Running: Start Machine     [ exec-time = 0.011 ms ]}
\NormalTok{(cmaxime)Running: Make Coffee       [ exec-time = 0.029 ms ]}
\NormalTok{(cmaxime)Running: Start Machine     [ exec-time = 0.009 ms ]}
\NormalTok{(cmaxime)Running: Make Coffee       [ exec-time = 0.024 ms ]}
\NormalTok{(cmaxime)Running: Add Water         [ exec-time = 5.026 s  ]}
\NormalTok{>}
\end{Highlighting}
\end{Shaded}

\clearpage

\hypertarget{exercise-03---json-issues-1}{%
\section{Exercise 03 - Json issues}\label{exercise-03---json-issues-1}}

\begin{longtable}[]{@{}rl@{}}
\toprule
\endhead
Turn-in directory: & ex03\tabularnewline
Files to turn in: & csvreader.py\tabularnewline
Forbidden functions: & None\tabularnewline
Remarks: & Context Manager\tabularnewline
\bottomrule
\end{longtable}

The context manager will help you handle this task.

Implement a \texttt{CsvReader} class that opens, reads, and parses a CSV
file.\\
In order to create a context manager, your class will require a few
built-in methods: \texttt{\_\_init\_\_}, \texttt{\_\_enter\_\_} and
\texttt{\_\_exit\_\_}.\\
It's mandatory to close the file once the process has completed.

\begin{Shaded}
\begin{Highlighting}[]
\KeywordTok{class}\NormalTok{ CsvReader():}
    \KeywordTok{def} \FunctionTok{__init__}\NormalTok{(}\VariableTok{self}\NormalTok{, filename}\OperatorTok{=}\VariableTok{None}\NormalTok{, sep}\OperatorTok{=}\StringTok{','}\NormalTok{, header}\OperatorTok{=}\VariableTok{False}\NormalTok{, skip_top}\OperatorTok{=}\DecValTok{0}\NormalTok{, skip_bottom}\OperatorTok{=}\DecValTok{0}\NormalTok{):}
        \ControlFlowTok{pass}
\end{Highlighting}
\end{Shaded}

Short for Comma-Separated Values, a CSV file is a delimited text file
which uses a comma to separate values. Therefore, the field separator
(or delimiter) is usually a comma \texttt{,}, but we aim to be able to
change this parameter.\\
You can make the class skip lines at the top and the bottom of the file,
and also keep the first line as a header if \texttt{header} is
\texttt{True}.

The file shouldn't be corrupted (either a line with too many values or a
line with too few values), otherwise return \texttt{None}. You have to
handle the case \texttt{file\ not\ found}.

You will have to also implement two methods: \texttt{getdata()} and
\texttt{getheader()}

\begin{Shaded}
\begin{Highlighting}[]
\ImportTok{from}\NormalTok{ csvreader }\ImportTok{import}\NormalTok{ CsvReader}

\ControlFlowTok{if} \VariableTok{__name__} \OperatorTok{==} \StringTok{"__main__"}\NormalTok{:}
    \ControlFlowTok{with}\NormalTok{ CsvReader(}\StringTok{'good.csv'}\NormalTok{) }\ImportTok{as} \BuiltInTok{file}\NormalTok{:}
\NormalTok{        data }\OperatorTok{=} \BuiltInTok{file}\NormalTok{.getdata()}
\NormalTok{        header }\OperatorTok{=} \BuiltInTok{file}\NormalTok{.getheader()}
\end{Highlighting}
\end{Shaded}

\begin{Shaded}
\begin{Highlighting}[]
\ImportTok{from}\NormalTok{ csvreader }\ImportTok{import}\NormalTok{ CsvReader}

\ControlFlowTok{if} \VariableTok{__name__} \OperatorTok{==} \StringTok{"__main__"}\NormalTok{:}
    \ControlFlowTok{with}\NormalTok{ CsvReader(}\StringTok{'bad.csv'}\NormalTok{) }\ImportTok{as} \BuiltInTok{file}\NormalTok{:}
        \ControlFlowTok{if} \BuiltInTok{file} \OperatorTok{==} \VariableTok{None}\NormalTok{:}
            \BuiltInTok{print}\NormalTok{(}\StringTok{"File is corrupted"}\NormalTok{)}
\end{Highlighting}
\end{Shaded}

\clearpage

\hypertarget{exercise-04---minipack-1}{%
\section{Exercise 04 - MiniPack}\label{exercise-04---minipack-1}}

\begin{longtable}[]{@{}rl@{}}
\toprule
\endhead
Turn-in directory: & ex04\tabularnewline
Files to turn in: & build.sh, *.py\tabularnewline
Forbidden functions: & None\tabularnewline
Remarks: & n/a\tabularnewline
\bottomrule
\end{longtable}

You have to create a package called \texttt{ai42}.

It will have 2 functionalities:

\begin{itemize}
\item
  the progress bar (day00 ex10), that can be imported via
  \texttt{import\ ai42.progressbar}
\item
  the logger (day02 ex02) \texttt{import\ ai42.logging.log}.
\end{itemize}

You may have to rename the functions and change the architecture of the
package.

The package will be installed via pip using the following command:

\begin{Shaded}
\begin{Highlighting}[]
\NormalTok{bash build.sh && pip install ./dist/ai42-1.0.0.tar.gz}
\end{Highlighting}
\end{Shaded}

The build.sh script has to create the \texttt{ai42-1.0.0.tar.gz} file.

You can ensure whether the package was properly installed by running the
command \texttt{pip\ list} that displays the list of installed packages.

\clearpage

\hypertarget{exercise-05---tinystatistician-1}{%
\section{Exercise 05 -
TinyStatistician}\label{exercise-05---tinystatistician-1}}

\begin{longtable}[]{@{}rl@{}}
\toprule
\endhead
\begin{minipage}[t]{0.54\columnwidth}\raggedleft
Turn-in directory:\strut
\end{minipage} & \begin{minipage}[t]{0.40\columnwidth}\raggedright
ex05\strut
\end{minipage}\tabularnewline
\begin{minipage}[t]{0.54\columnwidth}\raggedleft
Files to turn in:\strut
\end{minipage} & \begin{minipage}[t]{0.40\columnwidth}\raggedright
TinyStatistician.py\strut
\end{minipage}\tabularnewline
\begin{minipage}[t]{0.54\columnwidth}\raggedleft
Forbidden functions:\strut
\end{minipage} & \begin{minipage}[t]{0.40\columnwidth}\raggedright
Any function that calculates mean, median, quartiles, variance or
standard deviation for you\strut
\end{minipage}\tabularnewline
\begin{minipage}[t]{0.54\columnwidth}\raggedleft
Forbidden libraries:\strut
\end{minipage} & \begin{minipage}[t]{0.40\columnwidth}\raggedright
NumPy\strut
\end{minipage}\tabularnewline
\begin{minipage}[t]{0.54\columnwidth}\raggedleft
Remarks:\strut
\end{minipage} & \begin{minipage}[t]{0.40\columnwidth}\raggedright
n/a\strut
\end{minipage}\tabularnewline
\bottomrule
\end{longtable}

Create a class named \texttt{TinyStatistician} that implements the
following methods.\\
All methods take in an array and return a new modified one.\\
We are assuming that all inputs are correct, i.e.~you don't have to
protect your functions against input errors.

\begin{itemize}
\tightlist
\item
  \texttt{mean(x)} : computes the mean of a given non-empty array
  \texttt{x}, using a for-loop and returns the mean as a float,
  otherwise None if \texttt{x} is an empty array. This method should not
  raise any Exception.
\end{itemize}

Given a vector \(x\) of dimension m * 1, the mathematical formula of its
mean is: \large \[
\mu = \cfrac{\sum_{i = 1}^{m}{x_i}}{m}
\] \normalsize

\begin{itemize}
\item
  \texttt{median(x)} : computes the median, also called the 50th
  percentile, of a given non-empty darray \texttt{x}, using a for-loop
  and returns the median as a float, otherwise None if \texttt{x} is an
  empty array. This method should not raise any Exception.
\item
  \texttt{quartiles(x,\ percentile)} : computes the 1st and 3rd
  quartiles, also called the 25th percentile and the 75th percentile, of
  a given non-empty array \texttt{x}, using a for-loop and returns the
  quartile as a float, otherwise None if \texttt{x} is an empty array.
  The first parameter is the array and the second parameter is the
  expected percentile. This method should not raise any Exception.
\item
  \texttt{var(x)} : computes the variance of a given non-empty array
  \texttt{x}, using a for-loop and returns the variance as a float,
  otherwise None if \texttt{x} is an empty array. This method should not
  raise any Exception.
\end{itemize}

Given a vector \(x\) of dimension m * 1, the mathematical formula of its
variance is: \large \[
\sigma^2 = \cfrac{\sum_{i = 1}^{m}{(x_i - \mu)^2}}{m} = \cfrac{\sum_{i = 1}^{m}{[x_i - (\cfrac{1}{m}\sum_{j = 1}^{m}{x_j}})]^2}{m}
\] \normalsize

\begin{itemize}
\tightlist
\item
  \texttt{std(x)} : computes the standard deviation of a given non-empty
  array \texttt{x}, using a for-loop and returns the standard deviation
  as a float, otherwise None if \texttt{x} is an empty array. This
  method should not raise any Exception.
\end{itemize}

Given a vector \(x\) of dimension m * 1, the mathematical formula of its
standard deviation is: \large \[
\sigma = \sqrt{\cfrac{\sum_{i = 1}^{m}{(x_i - \mu)^2}}{m}} = \sqrt{\cfrac{\sum_{i = 1}^{m}{[x_i - (\cfrac{1}{m}\sum_{j = 1}^{m}{x_j}})]^2}{m}}
\] \normalsize

\textbf{Examples}

\begin{Shaded}
\begin{Highlighting}[]
\OperatorTok{>>>} \ImportTok{from}\NormalTok{ TinyStatistician }\ImportTok{import}\NormalTok{ TinyStatistician}
\OperatorTok{>>>}\NormalTok{ tstat }\OperatorTok{=}\NormalTok{ TinyStatistician()}
\OperatorTok{>>>}\NormalTok{ a }\OperatorTok{=}\NormalTok{ [}\DecValTok{1}\NormalTok{, }\DecValTok{42}\NormalTok{, }\DecValTok{300}\NormalTok{, }\DecValTok{10}\NormalTok{, }\DecValTok{59}\NormalTok{]}

\OperatorTok{>>>}\NormalTok{ tstat.mean(a)}
\DecValTok{82}\NormalTok{,}\DecValTok{4}

\OperatorTok{>>>}\NormalTok{ tstat.median(a)}
\FloatTok{42.0}

\OperatorTok{>>>}\NormalTok{ tstat.quartile(a, }\DecValTok{25}\NormalTok{)}
\FloatTok{10.0}

\OperatorTok{>>>}\NormalTok{ tstat.quartile(a, }\DecValTok{75}\NormalTok{)}
\FloatTok{59.0}

\OperatorTok{>>>}\NormalTok{ tstat.var(a)}
\FloatTok{12279.439999999999}

\OperatorTok{>>>}\NormalTok{ tstat.std(a)}
\FloatTok{110.81263465868862}
\end{Highlighting}
\end{Shaded}

\clearpage

\end{document}
