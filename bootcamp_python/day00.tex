\PassOptionsToPackage{unicode=true}{hyperref} % options for packages loaded elsewhere
\PassOptionsToPackage{hyphens}{url}
\PassOptionsToPackage{dvipsnames,svgnames*,x11names*}{xcolor}
%
\documentclass[]{article}
\usepackage{lmodern}
\usepackage{amssymb,amsmath}
\usepackage{ifxetex,ifluatex}
\usepackage{fixltx2e} % provides \textsubscript
\ifnum 0\ifxetex 1\fi\ifluatex 1\fi=0 % if pdftex
  \usepackage[T1]{fontenc}
  \usepackage[utf8x]{inputenc}
  \usepackage{textcomp} % provides euro and other symbols
\else % if luatex or xelatex
  \usepackage{unicode-math}
  \defaultfontfeatures{Ligatures=TeX,Scale=MatchLowercase}
\fi
% use upquote if available, for straight quotes in verbatim environments
\IfFileExists{upquote.sty}{\usepackage{upquote}}{}
% use microtype if available
\IfFileExists{microtype.sty}{%
\usepackage[]{microtype}
\UseMicrotypeSet[protrusion]{basicmath} % disable protrusion for tt fonts
}{}
\IfFileExists{parskip.sty}{%
\usepackage{parskip}
}{% else
\setlength{\parindent}{0pt}
\setlength{\parskip}{6pt plus 2pt minus 1pt}
}
\usepackage{xcolor}
\usepackage{hyperref}
\hypersetup{
            colorlinks=true,
            linkcolor=Maroon,
            citecolor=Blue,
            urlcolor=blue,
            breaklinks=true}
\urlstyle{same}  % don't use monospace font for urls
\usepackage{color}
\usepackage{fancyvrb}
\newcommand{\VerbBar}{|}
\newcommand{\VERB}{\Verb[commandchars=\\\{\}]}
\DefineVerbatimEnvironment{Highlighting}{Verbatim}{commandchars=\\\{\}}
% Add ',fontsize=\small' for more characters per line
\usepackage{framed}
\definecolor{shadecolor}{RGB}{35,38,41}
\newenvironment{Shaded}{\begin{snugshade}}{\end{snugshade}}
\newcommand{\AlertTok}[1]{\textcolor[rgb]{0.58,0.85,0.30}{#1}}
\newcommand{\AnnotationTok}[1]{\textcolor[rgb]{0.25,0.50,0.35}{#1}}
\newcommand{\AttributeTok}[1]{\textcolor[rgb]{0.16,0.50,0.73}{#1}}
\newcommand{\BaseNTok}[1]{\textcolor[rgb]{0.96,0.45,0.00}{#1}}
\newcommand{\BuiltInTok}[1]{\textcolor[rgb]{0.50,0.55,0.55}{#1}}
\newcommand{\CharTok}[1]{\textcolor[rgb]{0.24,0.68,0.91}{#1}}
\newcommand{\CommentTok}[1]{\textcolor[rgb]{0.48,0.49,0.49}{#1}}
\newcommand{\CommentVarTok}[1]{\textcolor[rgb]{0.50,0.55,0.55}{#1}}
\newcommand{\ConstantTok}[1]{\textcolor[rgb]{0.15,0.68,0.68}{#1}}
\newcommand{\ControlFlowTok}[1]{\textcolor[rgb]{0.99,0.74,0.29}{#1}}
\newcommand{\DataTypeTok}[1]{\textcolor[rgb]{0.16,0.50,0.73}{#1}}
\newcommand{\DecValTok}[1]{\textcolor[rgb]{0.96,0.45,0.00}{#1}}
\newcommand{\DocumentationTok}[1]{\textcolor[rgb]{0.64,0.20,0.25}{#1}}
\newcommand{\ErrorTok}[1]{\textcolor[rgb]{0.85,0.27,0.33}{#1}}
\newcommand{\ExtensionTok}[1]{\textcolor[rgb]{0.00,0.60,1.00}{#1}}
\newcommand{\FloatTok}[1]{\textcolor[rgb]{0.96,0.45,0.00}{#1}}
\newcommand{\FunctionTok}[1]{\textcolor[rgb]{0.56,0.27,0.68}{#1}}
\newcommand{\ImportTok}[1]{\textcolor[rgb]{0.15,0.68,0.38}{#1}}
\newcommand{\InformationTok}[1]{\textcolor[rgb]{0.77,0.36,0.00}{#1}}
\newcommand{\KeywordTok}[1]{\textcolor[rgb]{0.81,0.81,0.76}{#1}}
\newcommand{\NormalTok}[1]{\textcolor[rgb]{0.81,0.81,0.76}{#1}}
\newcommand{\OperatorTok}[1]{\textcolor[rgb]{0.81,0.81,0.76}{#1}}
\newcommand{\OtherTok}[1]{\textcolor[rgb]{0.15,0.68,0.38}{#1}}
\newcommand{\PreprocessorTok}[1]{\textcolor[rgb]{0.15,0.68,0.38}{#1}}
\newcommand{\RegionMarkerTok}[1]{\textcolor[rgb]{0.16,0.50,0.73}{#1}}
\newcommand{\SpecialCharTok}[1]{\textcolor[rgb]{0.24,0.68,0.91}{#1}}
\newcommand{\SpecialStringTok}[1]{\textcolor[rgb]{0.85,0.27,0.33}{#1}}
\newcommand{\StringTok}[1]{\textcolor[rgb]{0.96,0.31,0.31}{#1}}
\newcommand{\VariableTok}[1]{\textcolor[rgb]{0.15,0.68,0.68}{#1}}
\newcommand{\VerbatimStringTok}[1]{\textcolor[rgb]{0.85,0.27,0.33}{#1}}
\newcommand{\WarningTok}[1]{\textcolor[rgb]{0.85,0.27,0.33}{#1}}
\usepackage{longtable,booktabs}
% Fix footnotes in tables (requires footnote package)
\IfFileExists{footnote.sty}{\usepackage{footnote}\makesavenoteenv{longtable}}{}
\setlength{\emergencystretch}{3em}  % prevent overfull lines
\providecommand{\tightlist}{%
  \setlength{\itemsep}{0pt}\setlength{\parskip}{0pt}}
\setcounter{secnumdepth}{0}
% Redefines (sub)paragraphs to behave more like sections
\ifx\paragraph\undefined\else
\let\oldparagraph\paragraph
\renewcommand{\paragraph}[1]{\oldparagraph{#1}\mbox{}}
\fi
\ifx\subparagraph\undefined\else
\let\oldsubparagraph\subparagraph
\renewcommand{\subparagraph}[1]{\oldsubparagraph{#1}\mbox{}}
\fi

% set default figure placement to htbp
\makeatletter
\def\fps@figure{h}
\makeatother


\date{}

%%%%%%%%%%%%%%%%%%%%%%%%%%%%%%%%%%%%%%%%%%%%%%%%%%%%%%%%%%%%%%%%%%%%%%%%%%%%%%%%
%%%%%%%%%%%%%%%%%%%%%%%%%%%%%%%% Added packages %%%%%%%%%%%%%%%%%%%%%%%%%%%%%%%%
%%%%%%%%%%%%%%%%%%%%%%%%%%%%%%%%%%%%%%%%%%%%%%%%%%%%%%%%%%%%%%%%%%%%%%%%%%%%%%%%

\setcounter{MaxMatrixCols}{20}
\usepackage{cancel}
\usepackage{calc}
\usepackage{eso-pic}
\newlength{\PageFrameTopMargin}
\newlength{\PageFrameBottomMargin}
\newlength{\PageFrameLeftMargin}
\newlength{\PageFrameRightMargin}

\setlength{\PageFrameTopMargin}{1.5cm}
\setlength{\PageFrameBottomMargin}{1cm}
\setlength{\PageFrameLeftMargin}{1cm}
\setlength{\PageFrameRightMargin}{1cm}

\makeatletter

\newlength{\Page@FrameHeight}
\newlength{\Page@FrameWidth}

\AddToShipoutPicture{
  \thinlines
  \setlength{\Page@FrameHeight}{\paperheight-\PageFrameTopMargin-\PageFrameBottomMargin}
  \setlength{\Page@FrameWidth}{\paperwidth-\PageFrameLeftMargin-\PageFrameRightMargin}
  \put(\strip@pt\PageFrameLeftMargin,\strip@pt\PageFrameTopMargin){
    \framebox(\strip@pt\Page@FrameWidth, \strip@pt\Page@FrameHeight){}}}

\makeatother

\usepackage{fvextra}
\DefineVerbatimEnvironment{Highlighting}{Verbatim}{breaklines,breakanywhere,commandchars=\\\{\}}

\usepackage{graphicx}

\usepackage[a4paper, total={6in, 8in}]{geometry}
\geometry{hmargin=2cm,vmargin=2cm}

\usepackage{sectsty}

\sectionfont{\centering\Huge}
\subsectionfont{\Large}
\subsubsectionfont{\large}

\usepackage{titlesec}
\titlespacing*{\section}
{0pt}{5.5ex plus 1ex minus .2ex}{4.3ex plus .2ex}

\tolerance=1
\emergencystretch=\maxdimen
\hyphenpenalty=10000
\hbadness=10000

%%%%%%%%%%%%%%%%%%%%%%%%%%%%%%%%%%%%%%%%%%%%%%%%%%%%%%%%%%%%%%%%%%%%%%%%%%%%%%%%
%%%%%%%%%%%%%%%%%%%%%%%%%%%%%%%%%%%%%%%%%%%%%%%%%%%%%%%%%%%%%%%%%%%%%%%%%%%%%%%%

\begin{document}

%%%%%%%%%%%%%%%%%%%%%%%%%%%%%%%%%%%%%%%%%%%%%%%%%%%%%%%%%%%%%%%%%%%%%%%%%%%%%%%%
%%%%%%%%%%%%%%%%%%%%%%%%%%%%%%%% Added lines %%%%%%%%%%%%%%%%%%%%%%%%%%%%%%%%%%%
%%%%%%%%%%%%%%%%%%%%%%%%%%%%%%%%%%%%%%%%%%%%%%%%%%%%%%%%%%%%%%%%%%%%%%%%%%%%%%%%

\vspace*{2cm}
\begin{center}
    \textsc{\fontsize{40}{48} \bfseries }\\[0.6cm]
    \textsc{\fontsize{39}{48} \bfseries { %bootcamp_title
Python Machine-Learning
    }}\\[0.3cm]
\end{center}
\vspace{3cm}

\begin{center}
\includegraphics[width=200pt]{assets/logo-42-ai.png}{\centering}
\end{center}

\vspace*{2cm}
\begin{center}
    \textsc{\fontsize{32}{48} \bfseries %day_number
Module 00    
    }\\[0.6cm]
    \textsc{\fontsize{32}{48} \bfseries %day_title
Basic Stuff    
    }\\[0.3cm]
\end{center}
\vspace{3cm}

\pagenumbering{gobble}
\newpage

%%% >>>>> Page de garde
\setcounter{page}{1}
\pagenumbering{arabic}

%%%%%%%%%%%%%%%%%%%%%%%%%%%%%%%%%%%%%%%%%%%%%%%%%%%%%%%%%%%%%%%%%%%%%%%%%%%%%%%%
%%%%%%%%%%%%%%%%%%%%%%%%%%%%%%%%%%%%%%%%%%%%%%%%%%%%%%%%%%%%%%%%%%%%%%%%%%%%%%%%


\hypertarget{module-00---basic-stuff---eleven-commandments}{%
\section{Module 00 - Basic stuff - Eleven
Commandments}\label{module-00---basic-stuff---eleven-commandments}}

The goal of the day is to get started with the Python language.

\hypertarget{notions-of-the-day}{%
\subsection{Notions of the day}\label{notions-of-the-day}}

Basic setup, variables, types, functions, \ldots{}

\hypertarget{general-rules}{%
\subsection{General rules}\label{general-rules}}

\begin{itemize}
\item
  The version of Python to use is 3.7, you can check the version of
  Python with the following command: \texttt{python\ -V}
\item
  The norm: during this module you will follow the
  \href{https://www.python.org/dev/peps/pep-0008/}{PEP 8 standards}. You
  can install \href{https://pypi.org/project/pycodestyle}{pycodestyle}
  which is a tool to check your Python code.
\item
  The function eval is never allowed.
\item
  The exercises are ordered from the easiest to the hardest.
\item
  Your exercises are going to be evaluated by someone else, so make sure
  that your variable names and function names are appropriate and civil.
\item
  Your manual is the internet.
\item
  You can also ask questions in the dedicated channel in the 42 AI
  Slack: 42-ai.slack.com.
\item
  If you find any issue or mistakes in the subject please create an
  issue on our
  \href{https://github.com/42-AI/bootcamp_python/issues}{dedicated
  repository on Github}.
\end{itemize}

\hypertarget{helper}{%
\subsection{Helper}\label{helper}}

How do you install and link Python in the \$PATH? That's the first
exercise!

Ensure that you have the right Python version.

\begin{Shaded}
\begin{Highlighting}[]
\NormalTok{> which python}
\NormalTok{/goinfre/miniconda/bin/python}
\NormalTok{> python -V}
\NormalTok{Python 3.7.*}
\NormalTok{> which pip}
\NormalTok{/goinfre/miniconda/bin/pip}
\end{Highlighting}
\end{Shaded}

\hypertarget{exercise-00---path}{%
\subsubsection{Exercise 00 - \$PATH}\label{exercise-00---path}}

\hypertarget{exercise-01---rev-alpha}{%
\subsubsection{Exercise 01 - Rev Alpha}\label{exercise-01---rev-alpha}}

\hypertarget{exercise-02---the-odd-the-even-and-the-zero}{%
\subsubsection{Exercise 02 - The Odd, the Even and the
Zero}\label{exercise-02---the-odd-the-even-and-the-zero}}

\hypertarget{exercise-03---functional-file}{%
\subsubsection{Exercise 03 - Functional
file}\label{exercise-03---functional-file}}

\hypertarget{exercise-04---elementary}{%
\subsubsection{Exercise 04 -
Elementary}\label{exercise-04---elementary}}

\hypertarget{exercise-05---the-right-format}{%
\subsubsection{Exercise 05 - The right
format}\label{exercise-05---the-right-format}}

\hypertarget{exercise-06---a-recipe}{%
\subsubsection{Exercise 06 - A recipe}\label{exercise-06---a-recipe}}

\hypertarget{exercise-07---shorter-faster-pythonest}{%
\subsubsection{Exercise 07 - Shorter, faster,
pythonest}\label{exercise-07---shorter-faster-pythonest}}

\hypertarget{exercise-08---s.o.s}{%
\subsubsection{Exercise 08 - S.O.S}\label{exercise-08---s.o.s}}

\hypertarget{exercise-09---secret-number}{%
\subsubsection{Exercise 09 - Secret
number}\label{exercise-09---secret-number}}

\hypertarget{exercise-10---loading-bar}{%
\subsubsection{Exercise 10 - Loading
bar!}\label{exercise-10---loading-bar}}

\clearpage

\hypertarget{exercise-00---path-1}{%
\section{Exercise 00 - \$PATH}\label{exercise-00---path-1}}

\begin{longtable}[]{@{}rl@{}}
\toprule
\endhead
Turn-in directory: & ex00\tabularnewline
Files to turn in: & answers.txt, requirements.txt\tabularnewline
Forbidden functions: & None\tabularnewline
Remarks: & n/a\tabularnewline
\bottomrule
\end{longtable}

The first thing you need to do is install Python.

\hypertarget{conda-manual-install}{%
\subsection{Conda manual install}\label{conda-manual-install}}

If you want a fully automated install go to Automated install part. The
automated part will allow you to reinstall everything more easily if you
change of Desktop. Here, we are going to see a step by step install.

\begin{enumerate}
\def\labelenumi{\arabic{enumi}.}
\tightlist
\item
  Download conda install with the following command (MacOS version):
\end{enumerate}

\begin{Shaded}
\begin{Highlighting}[]
\NormalTok{curl -LO "https://repo.anaconda.com/miniconda/Miniconda3-latest-MacOSX-x86_64.sh"}
\end{Highlighting}
\end{Shaded}

\begin{enumerate}
\def\labelenumi{\arabic{enumi}.}
\setcounter{enumi}{1}
\tightlist
\item
  Install conda using the script (we advise you to install it with this
  path \texttt{/goinfre/miniconda3}).
\end{enumerate}

\begin{Shaded}
\begin{Highlighting}[]
\NormalTok{sh Miniconda3-latest-MacOSX-x86_64.sh -b -p <path>}
\end{Highlighting}
\end{Shaded}

The goinfre will change depending on your desktop location in cluster,
so you will need to reinstall everything.

\begin{enumerate}
\def\labelenumi{\arabic{enumi}.}
\setcounter{enumi}{2}
\tightlist
\item
  Add export to your \texttt{.zshrc} file.
\end{enumerate}

\begin{Shaded}
\begin{Highlighting}[]
\NormalTok{export PATH=$MINICONDA_PATH:$PATH}
\end{Highlighting}
\end{Shaded}

\begin{enumerate}
\def\labelenumi{\arabic{enumi}.}
\setcounter{enumi}{3}
\tightlist
\item
  Source your \texttt{.zshrc} file.
\end{enumerate}

\begin{Shaded}
\begin{Highlighting}[]
\NormalTok{source ~/.zshrc}
\end{Highlighting}
\end{Shaded}

\begin{enumerate}
\def\labelenumi{\arabic{enumi}.}
\setcounter{enumi}{4}
\tightlist
\item
  Check your Python environment.
\end{enumerate}

\begin{Shaded}
\begin{Highlighting}[]
\NormalTok{which python}
\end{Highlighting}
\end{Shaded}

\begin{enumerate}
\def\labelenumi{\arabic{enumi}.}
\setcounter{enumi}{5}
\tightlist
\item
  Install needed requirements.
\end{enumerate}

\begin{Shaded}
\begin{Highlighting}[]
\NormalTok{conda install -y "jupyter" "numpy" "pandas"}
\end{Highlighting}
\end{Shaded}

Your Python should now be the one corresponding to the miniconda
environment!

\hypertarget{conda-automated-install}{%
\subsection{Conda automated install}\label{conda-automated-install}}

A way to install the entire environment is to define a bash function in
your \texttt{\textasciitilde{}/.zshrc}.

\begin{enumerate}
\def\labelenumi{\arabic{enumi}.}
\tightlist
\item
  Copy paste the following code into your
  \texttt{\textasciitilde{}/.zshrc}.
\end{enumerate}

\begin{Shaded}
\begin{Highlighting}[]
\NormalTok{function set_conda \{}
\NormalTok{    HOME=$(echo ~)}
\NormalTok{    INSTALL_PATH="/goinfre"}
\NormalTok{    MINICONDA_PATH=$INSTALL_PATH"/miniconda3/bin"}
\NormalTok{    PYTHON_PATH=$(which python)}
\NormalTok{    SCRIPT="Miniconda3-latest-MacOSX-x86_64.sh"}
\NormalTok{    REQUIREMENTS="jupyter numpy pandas"}
\NormalTok{    DL_LINK="https://repo.anaconda.com/miniconda/Miniconda3-latest-MacOSX-x86_64.sh"}

\NormalTok{    if echo $PYTHON_PATH | grep -q $INSTALL_PATH; then}
\NormalTok{        echo "good python version :)"}
\NormalTok{    else}
\NormalTok{    cd}
\NormalTok{    if [ ! -f $SCRIPT ]; then}
\NormalTok{        curl -LO $DL_LINK}
\NormalTok{        fi}
\NormalTok{        if [ ! -d $MINICONDA_PATH ]; then}
\NormalTok{            sh $SCRIPT -b -p $INSTALL_PATH"/miniconda3"}
\NormalTok{    fi}
\NormalTok{    conda install -y $(echo $REQUIREMENTS)}
\NormalTok{    clear}
\NormalTok{    echo "Which python:"}
\NormalTok{    which python}
\NormalTok{    if grep -q "^export PATH=$MINICONDA_PATH" ~/.zshrc}
\NormalTok{    then}
\NormalTok{        echo "export already in .zshrc";}
\NormalTok{    else}
\NormalTok{        echo "adding export to .zshrc ...";}
\NormalTok{        echo "export PATH=$MINICONDA_PATH:\textbackslash{}$PATH" >> ~/.zshrc}
\NormalTok{    fi}
\NormalTok{    source ~/.zshrc}
\NormalTok{    fi}
\NormalTok{\}}
\end{Highlighting}
\end{Shaded}

By default, conda will be installed in the goinfre (look at the
\texttt{INSTALL\_PATH} variable). Feel free to change that path if you
want to.

The function can be used whenever we want and will carry out the
installation of miniconda and all needed librairies for the day. It will
also add a line to export miniconda environment.

\begin{enumerate}
\def\labelenumi{\arabic{enumi}.}
\setcounter{enumi}{1}
\tightlist
\item
  Source your \texttt{.zshrc} with the following command:
\end{enumerate}

\begin{Shaded}
\begin{Highlighting}[]
\NormalTok{source ~/.zshrc}
\end{Highlighting}
\end{Shaded}

\begin{enumerate}
\def\labelenumi{\arabic{enumi}.}
\setcounter{enumi}{2}
\tightlist
\item
  Use the function \texttt{set\_conda}:
\end{enumerate}

\begin{Shaded}
\begin{Highlighting}[]
\NormalTok{set_conda}
\end{Highlighting}
\end{Shaded}

When the installation is done rerun the \texttt{set\_conda} function.

\begin{enumerate}
\def\labelenumi{\arabic{enumi}.}
\setcounter{enumi}{3}
\tightlist
\item
  Check your Python path.
\end{enumerate}

\begin{Shaded}
\begin{Highlighting}[]
\NormalTok{which python}
\end{Highlighting}
\end{Shaded}

Your Python should now be the one corresponding to the miniconda
environment!

\hypertarget{getting-started}{%
\subsection{Getting started}\label{getting-started}}

As an introduction, complete the following questionnaire using Python
and \texttt{pip}, save your answers in a file \texttt{answers.txt}
(write an answer per line in the text file), and check them with your
peers.

Find the commands to:

\begin{enumerate}
\def\labelenumi{\arabic{enumi}.}
\item
  Output a list of installed packages.
\item
  Output a list of installed packages and their versions.
\item
  Show the package metadata of \texttt{numpy}.
\item
  Search for PyPI packages whose name or summary contains ``tesseract''.
\item
  Freeze the packages and their current versions in a
  \texttt{requirements.txt} file you have to turn-in.
\end{enumerate}

\clearpage

\hypertarget{exercise-01---rev-alpha-1}{%
\section{Exercise 01 - Rev Alpha}\label{exercise-01---rev-alpha-1}}

\begin{longtable}[]{@{}rl@{}}
\toprule
\endhead
Turn-in directory: & ex01\tabularnewline
Files to turn in: & exec.py\tabularnewline
Forbidden functions: & None\tabularnewline
Remarks: & n/a\tabularnewline
\bottomrule
\end{longtable}

You will have to make a program that reverses the order of a string and
the case of its words.\\
If we have more than one argument we have to merge them into a single
string and separate each arg by a ' ' (space char).

\textbf{Example:}

\begin{Shaded}
\begin{Highlighting}[]
\NormalTok{> python exec.py "Hello World\textbackslash{}!" | cat -e}
\NormalTok{!DLROw OLLEh$}
\NormalTok{> python exec.py "Hello" "my Friend" | cat -e}
\NormalTok{DNEIRf YM OLLEh$}
\NormalTok{> python exec.py}
\NormalTok{>}
\end{Highlighting}
\end{Shaded}

\clearpage

\hypertarget{exercise-02---the-odd-the-even-and-the-zero-1}{%
\section{Exercise 02 - The Odd, the Even and the
Zero}\label{exercise-02---the-odd-the-even-and-the-zero-1}}

\begin{longtable}[]{@{}rl@{}}
\toprule
\endhead
Turn-in directory: & ex02\tabularnewline
Files to turn in: & whois.py\tabularnewline
Forbidden functions: & None\tabularnewline
Remarks: & n/a\tabularnewline
\bottomrule
\end{longtable}

You will have to make a program that checks if a number is odd, even or
zero.\\
The program will accept only one parameter, an integer.

\textbf{Example:}

\begin{Shaded}
\begin{Highlighting}[]
\NormalTok{> python whois.py 12}
\NormalTok{I'm Even.}
\NormalTok{> python whois.py 3}
\NormalTok{I'm Odd.}
\NormalTok{> python whois.py}
\NormalTok{> python whois.py 0}
\NormalTok{I'm Zero.}
\NormalTok{> python whois.py Hello}
\NormalTok{ERROR}
\NormalTok{> python whois.py 12 3}
\NormalTok{ERROR}
\end{Highlighting}
\end{Shaded}

\clearpage

\hypertarget{exercise-03---functional-file-1}{%
\section{Exercise 03 - Functional
file}\label{exercise-03---functional-file-1}}

\begin{longtable}[]{@{}rl@{}}
\toprule
\endhead
Turn-in directory: & ex03\tabularnewline
Files to turn in: & count.py\tabularnewline
Forbidden functions: & None\tabularnewline
Remarks: & n/a\tabularnewline
\bottomrule
\end{longtable}

Create a function called \texttt{text\_analyzer} that displays the sums
of upper-case characters, lower-case characters, punctuation characters
and spaces in a given text.

\texttt{text\_analyzer} will take only one parameter: the text to
analyze. You have to handle the case where the text is empty (maybe by
setting a default value). If there is no text passed to the function,
the user is prompted to give one.

Test it in the Python console.

\textbf{Example:}

\begin{Shaded}
\begin{Highlighting}[]
\NormalTok{> python}
\NormalTok{>>> from count import text_analyzer}
\NormalTok{>>> text_analyzer("Python 2.0, released 2000, introduced }
\NormalTok{features like List comprehensions and a garbage collection}
\NormalTok{system capable of collecting reference cycles.")}
\NormalTok{The text contains 143 characters:}
\NormalTok{- 2 upper letters}
\NormalTok{- 113 lower letters}
\NormalTok{- 4 punctuation marks}
\NormalTok{- 18 spaces}
\NormalTok{>>> text_analyzer("Python is an interpreted, high-level,}
\NormalTok{general-purpose programming language. Created by Guido van}
\NormalTok{Rossum and first released in 1991, Python's design philosophy}
\NormalTok{emphasizes code readability with its notable use of significant}
\NormalTok{whitespace.")}
\NormalTok{The text contains 234 characters:}
\NormalTok{- 5 upper letters}
\NormalTok{- 187 lower letters}
\NormalTok{- 8 punctuation marks}
\NormalTok{- 30 spaces}
\NormalTok{>>> text_analyzer()}
\NormalTok{What is the text to analyse?}
\NormalTok{>> Python is an interpreted, high-level, general-purpose}
\NormalTok{programming language. Created by Guido van Rossum and first}
\NormalTok{released in 1991, Python's design philosophy emphasizes code}
\NormalTok{readability with its notable use of significant whitespace.}
\NormalTok{The text contains 234 characters:}
\NormalTok{- 5 upper letters}
\NormalTok{- 187 lower letters}
\NormalTok{- 8 punctuation marks}
\NormalTok{- 30 spaces}
\end{Highlighting}
\end{Shaded}

Handle the case when more than one parameter is given to
\texttt{text\_analyzer}:

\begin{Shaded}
\begin{Highlighting}[]
\NormalTok{>>> from count import text_analyzer}
\NormalTok{>>> text_analyzer("Python", "2.0")}
\NormalTok{ERROR}
\end{Highlighting}
\end{Shaded}

You're free to write your docstring and format it the way you want.

\begin{Shaded}
\begin{Highlighting}[]
\NormalTok{>>> print(text_analyzer.__doc__)}

\NormalTok{    This function counts the number of upper characters, lower characters,}
\NormalTok{    punctuation and spaces in a given text.}
\end{Highlighting}
\end{Shaded}

\clearpage

\hypertarget{exercise-04---elementary-1}{%
\section{Exercise 04 - Elementary}\label{exercise-04---elementary-1}}

\begin{longtable}[]{@{}rl@{}}
\toprule
\endhead
Turn-in directory: & ex04\tabularnewline
Files to turn in: & operations.py\tabularnewline
Forbidden functions: & None\tabularnewline
Remarks: & n/a\tabularnewline
\bottomrule
\end{longtable}

You will have to make a program that prints the results of the four
elementary mathematical operations of arithmetic (addition, subtraction,
multiplication, division) and the modulo operation. This should be
accomplished by writing a function that takes 2 numbers as parameters
and returns 5 values, as formatted in the console output below.

\textbf{Example:}

\begin{Shaded}
\begin{Highlighting}[]
\NormalTok{> python operations.py 10 3}
\NormalTok{Sum:         13}
\NormalTok{Difference:  7}
\NormalTok{Product:     30}
\NormalTok{Quotient:    3.3333333333333335}
\NormalTok{Remainder:   1}
\NormalTok{>}
\NormalTok{> python operations.py 42 10}
\NormalTok{Sum:         52}
\NormalTok{Difference:  32}
\NormalTok{Product:     420}
\NormalTok{Quotient:    4.2}
\NormalTok{Remainder:   2}
\NormalTok{>}
\NormalTok{> python operations.py 1 0}
\NormalTok{Sum:         1}
\NormalTok{Difference:  1}
\NormalTok{Product:     0}
\NormalTok{Quotient:    ERROR (div by zero)}
\NormalTok{Remainder:   ERROR (modulo by zero)}
\NormalTok{>}
\NormalTok{> python operations.py}
\NormalTok{Usage: python operations.py <number1> <number2>}
\NormalTok{Example:}
\NormalTok{    python operations.py 10 3}
\NormalTok{>}
\NormalTok{> python operations.py 12 10 5}
\NormalTok{InputError: too many arguments}

\NormalTok{Usage: python operations.py <number1> <number2>}
\NormalTok{Example:}
\NormalTok{    python operations.py 10 3}
\NormalTok{>}
\NormalTok{> python operations.py "one" "two"}
\NormalTok{InputError: only numbers}

\NormalTok{Usage: python operations.py <number1> <number2>}
\NormalTok{Example:}
\NormalTok{    python operations.py 10 3}
\NormalTok{>}
\NormalTok{> python operations.py "512" "63.1"}
\NormalTok{InputError: only numbers}

\NormalTok{Usage: python operations.py <number1> <number2>}
\NormalTok{Example:}
\NormalTok{    python operations.py 10 3}
\end{Highlighting}
\end{Shaded}

\clearpage

\hypertarget{exercise-05---the-right-format-1}{%
\section{Exercise 05 - The right
format}\label{exercise-05---the-right-format-1}}

\begin{longtable}[]{@{}rl@{}}
\toprule
\endhead
Turn-in directory: & ex05\tabularnewline
Files to turn in: & kata00.py, kata01.py, kata02.py, kata03.py,
kata04.py\tabularnewline
Forbidden functions: & None\tabularnewline
Remarks: & n/a\tabularnewline
\bottomrule
\end{longtable}

Let's get familiar with the useful concept of \textbf{string formatting}
through a kata series.

\hypertarget{kata00}{%
\subsubsection{kata00}\label{kata00}}

\begin{Shaded}
\begin{Highlighting}[]
\NormalTok{t = (19,42,21)}
\end{Highlighting}
\end{Shaded}

Including the tuple above in your file, write a program that dynamically
builds up a formatted string like the following:

\begin{Shaded}
\begin{Highlighting}[]
\NormalTok{> python kata00.py}
\NormalTok{The 3 numbers are: 19, 42, 21}
\end{Highlighting}
\end{Shaded}

\hypertarget{kata01}{%
\subsubsection{kata01}\label{kata01}}

\begin{Shaded}
\begin{Highlighting}[]
\NormalTok{languages = \{}
\NormalTok{    'Python': 'Guido van Rossum',}
\NormalTok{    'Ruby': 'Yukihiro Matsumoto',}
\NormalTok{    'PHP': 'Rasmus Lerdorf',}
\NormalTok{    \}}
\end{Highlighting}
\end{Shaded}

Using the \texttt{languages} dictionary above, a similar exercise:

\begin{Shaded}
\begin{Highlighting}[]
\NormalTok{> python kata01.py}
\NormalTok{Python was created by Guido van Rossum}
\NormalTok{Ruby was created by Yukihiro Matsumoto}
\NormalTok{PHP was created by Rasmus Lerdorf}
\end{Highlighting}
\end{Shaded}

\hypertarget{kata02}{%
\subsubsection{kata02}\label{kata02}}

\begin{Shaded}
\begin{Highlighting}[]
\NormalTok{(3,30,2019,9,25)}
\end{Highlighting}
\end{Shaded}

Given the tuple above, whose values stand for:
\texttt{(hour,\ minutes,\ year,\ month,\ day)}, write a program that
displays it in the following format:

\begin{Shaded}
\begin{Highlighting}[]
\NormalTok{> python kata02.py}
\NormalTok{09/25/2019 03:30}
\end{Highlighting}
\end{Shaded}

\hypertarget{kata03}
\NormalTok{> python kata03.py | wc -c}
\NormalTok{    42}
\end{Highlighting}
\end{Shaded}

\hypertarget{kata04}{%
\subsubsection{kata04}\label{kata04}}

\begin{Shaded}
\begin{Highlighting}[]
\NormalTok{( 0, 4, 132.42222, 10000, 12345.67)}
\end{Highlighting}
\end{Shaded}

Given the tuple above, return the following result:

\begin{Shaded}
\begin{Highlighting}[]
\NormalTok{> python kata04.py}
\NormalTok{day_00, ex_04 : 132.42, 1.00e+04, 1.23e+04}
\end{Highlighting}
\end{Shaded}

\clearpage

\hypertarget{exercise-06---a-recipe-1}{%
\section{Exercise 06 - A recipe}\label{exercise-06---a-recipe-1}}

\begin{longtable}[]{@{}rl@{}}
\toprule
\endhead
Turn-in directory: & ex06\tabularnewline
Files to turn in: & recipe.py\tabularnewline
Forbidden functions: & None\tabularnewline
Remarks: & n/a\tabularnewline
\bottomrule
\end{longtable}

It is time to discover Python dictionaries. Dictionaries are collections
that contain mappings of unique keys to values.

\textbf{\emph{Hints:}} check what is a nested dictionary in Python.

First, you will have to create a cookbook dictionary called
\texttt{cookbook}.

\texttt{cookbook} will store 3 recipes:

\begin{itemize}
\item
  sandwich
\item
  cake
\item
  salad
\end{itemize}

Each recipe will store 3 values:

\begin{itemize}
\item
  ingredients: a \textbf{list} of ingredients
\item
  meal: type of meal
\item
  prep\_time: preparation time in minutes
\end{itemize}

Sandwich's ingredients are \emph{ham}, \emph{bread}, \emph{cheese} and
\emph{tomatoes}. It is a \emph{lunch} and it takes \emph{10} minutes of
preparation.\\
Cake's ingredients are \emph{flour}, \emph{sugar} and \emph{eggs}. It is
a \emph{dessert} and it takes \emph{60} minutes of preparation.\\
Salad's ingredients are \emph{avocado}, \emph{arugula}, \emph{tomatoes}
and \emph{spinach}. It is a \emph{lunch} and it takes \emph{15} minutes
of preparation.

\begin{enumerate}
\def\labelenumi{\arabic{enumi}.}
\item
  Get to know dictionaries. In the first place, try to print only the
  \texttt{keys} of the dictionary. Then only the \texttt{values}. And to
  conclude, all the \texttt{items}.
\item
  Write a function to print a recipe from \texttt{cookbook}. The
  function parameter will be: name of the recipe.
\item
  Write a function to delete a recipe from the dictionary. The function
  parameter will be: name of the recipe.
\item
  Write a function to add a new recipe to \texttt{cookbook} with its
  ingredients, its meal type and its preparation time. The function
  parameters will be: name of recipe, ingredients, meal and prep\_time.
\item
  Write a function to print all recipe names from \texttt{cookbook}.
  Think about formatting the output.
\item
  Last but not least, make a program using the four functions you just
  created. The program will prompt the user to make a choice between
  printing the cookbook, printing only one recipe, adding a recipe,
  deleting a recipe or quitting the cookbook.
\end{enumerate}

It could look like the example below but feel free to organize it the
way you want to:

\begin{Shaded}
\begin{Highlighting}[]
\NormalTok{> python recipe.py}
\NormalTok{Please select an option by typing the corresponding number:}
\NormalTok{1: Add a recipe}
\NormalTok{2: Delete a recipe}
\NormalTok{3: Print a recipe}
\NormalTok{4: Print the cookbook}
\NormalTok{5: Quit}
\NormalTok{>> 3}

\NormalTok{Please enter the recipe's name to get its details:}
\NormalTok{>> cake}

\NormalTok{Recipe for cake:}
\NormalTok{Ingredients list: ['flour', 'sugar', 'eggs']}
\NormalTok{To be eaten for dessert.}
\NormalTok{Takes 60 minutes of cooking.}
\end{Highlighting}
\end{Shaded}

Your program must continue running until the user exits it (option 5):

\begin{Shaded}
\begin{Highlighting}[]
\NormalTok{> python recipe.py}
\NormalTok{Please select an option by typing the corresponding number:}
\NormalTok{1: Add a recipe}
\NormalTok{2: Delete a recipe}
\NormalTok{3: Print a recipe}
\NormalTok{4: Print the cookbook}
\NormalTok{5: Quit}
\NormalTok{>> 5}

\NormalTok{Cookbook closed.}
\end{Highlighting}
\end{Shaded}

The program will also continue running if the user enters a wrong value.
It will prompt the user again until the value is correct:

\begin{Shaded}
\begin{Highlighting}[]
\NormalTok{> python recipe.py}
\NormalTok{Please select an option by typing the corresponding number:}
\NormalTok{1: Add a recipe}
\NormalTok{2: Delete a recipe}
\NormalTok{3: Print a recipe}
\NormalTok{4: Print the cookbook}
\NormalTok{5: Quit}
\NormalTok{>> test}

\NormalTok{This option does not exist, please type the corresponding number.}
\NormalTok{To exit, enter 5.}
\NormalTok{>> }
\end{Highlighting}
\end{Shaded}

\clearpage

\hypertarget{exercise-07---shorter-faster-pythonest-1}{%
\section{Exercise 07 - Shorter, faster,
pythonest}\label{exercise-07---shorter-faster-pythonest-1}}

\begin{longtable}[]{@{}rl@{}}
\toprule
\endhead
Turn-in directory: & ex07\tabularnewline
Files to turn in: & filterwords.py\tabularnewline
Forbidden functions: & filter\tabularnewline
Remarks: & n/a\tabularnewline
\bottomrule
\end{longtable}

Using list comprehensions, you will have to make a program that removes
all the words in a string that are shorter than or equal to n letters,
and returns the filtered list with no punctuation.\\
The program will accept only two parameters: a string, and an integer n.

\textbf{Example:}

\begin{Shaded}
\begin{Highlighting}[]
\NormalTok{> python filterwords.py "Hello, my friend" 3}
\NormalTok{['Hello', 'friend']}
\NormalTok{>  python filterwords.py "A robot must protect its own existence as long as such protection does not conflict with the First or Second Law" 6}
\NormalTok{['protect', 'existence', 'protection', 'conflict']}
\NormalTok{> python filterwords.py Hello World}
\NormalTok{ERROR}
\NormalTok{> python filterwords.py 300 3}
\NormalTok{ERROR}
\end{Highlighting}
\end{Shaded}

\clearpage

\hypertarget{exercise-08---s.o.s-1}{%
\section{Exercise 08 - S.O.S}\label{exercise-08---s.o.s-1}}

\begin{longtable}[]{@{}rl@{}}
\toprule
\endhead
Turn-in directory: & ex08\tabularnewline
Files to turn in: & sos.py\tabularnewline
Forbidden functions: & None\tabularnewline
Remarks: & n/a\tabularnewline
\bottomrule
\end{longtable}

You will have to make a function which encodes strings into Morse
code.\\
The input will accept all alphanumeric characters.

\textbf{Example:}

\begin{Shaded}
\begin{Highlighting}[]
\NormalTok{> python sos.py "SOS"}
\NormalTok{... --- ...}
\NormalTok{> python sos.py}
\NormalTok{> python sos.py "HELLO / WORLD"}
\NormalTok{ERROR}
\NormalTok{> python sos.py "96 BOULEVARD" "Bessiere"}
\NormalTok{----. -.... / -... --- ..- .-.. . ...- .- .-. -.. / -... . ... ... .. . .-. .}
\end{Highlighting}
\end{Shaded}

@ref: https://morsecode.scphillips.com/morse.html

\clearpage

\hypertarget{exercise-09---secret-number-1}{%
\section{Exercise 09 - Secret
number}\label{exercise-09---secret-number-1}}

\begin{longtable}[]{@{}rl@{}}
\toprule
\endhead
Turn-in directory: & ex09\tabularnewline
Files to turn in: & guess.py\tabularnewline
Forbidden functions: & None\tabularnewline
Remarks: & n/a\tabularnewline
\bottomrule
\end{longtable}

You will have to make a program that will be an interactive guessing
game. It will ask the user to guess a number between 1 and 99. The
program will tell the user if their input is too high or too low. The
game ends when the user finds out the secret number or types
\texttt{exit}.\\
You will have to import the \texttt{random} module with the
\texttt{randint} function to get a random number. You have to count the
number of trials and print that number when the user wins.

\textbf{Example:}

\begin{Shaded}
\begin{Highlighting}[]
\NormalTok{> python guess.py}
\NormalTok{This is an interactive guessing game!}
\NormalTok{You have to enter a number between 1 and 99 to find out the secret number.}
\NormalTok{Type 'exit' to end the game.}
\NormalTok{Good luck!}

\NormalTok{What's your guess between 1 and 99?}
\NormalTok{>> 54}
\NormalTok{Too high!}
\NormalTok{What's your guess between 1 and 99?}
\NormalTok{>> 34}
\NormalTok{Too low!}
\NormalTok{What's your guess between 1 and 99?}
\NormalTok{>> 45}
\NormalTok{Too high!}
\NormalTok{What's your guess between 1 and 99?}
\NormalTok{>> A}
\NormalTok{That's not a number.}
\NormalTok{What's your guess between 1 and 99?}
\NormalTok{>> 43}
\NormalTok{Congratulations, you've got it!}
\NormalTok{You won in 5 attempts!}
\end{Highlighting}
\end{Shaded}

If the user discovers the secret number on the first try, tell them.
Bonus: if the secret number is 42, make a reference to Douglas Adams.

\begin{Shaded}
\begin{Highlighting}[]
\NormalTok{> python guess.py}
\NormalTok{What's your guess between 1 and 99?}
\NormalTok{>> 42}
\NormalTok{The answer to the ultimate question of life, the universe and everything is 42.}
\NormalTok{Congratulations! You got it on your first try!}
\end{Highlighting}
\end{Shaded}

Other example:

\begin{Shaded}
\begin{Highlighting}[]
\NormalTok{> python guess.py}
\NormalTok{This is an interactive guessing game!}
\NormalTok{You have to enter a number between 1 and 99 to find out the secret number.}
\NormalTok{Type 'exit' to end the game.}
\NormalTok{Good luck!}

\NormalTok{What's your guess between 1 and 99?}
\NormalTok{>> exit}
\NormalTok{Goodbye!}
\end{Highlighting}
\end{Shaded}

\clearpage

\hypertarget{exercise-10---loading-bar-1}{%
\section{Exercise 10 - Loading bar!}\label{exercise-10---loading-bar-1}}

\begin{longtable}[]{@{}rl@{}}
\toprule
\endhead
Turn-in directory: & ex10\tabularnewline
Files to turn in: & loading.py\tabularnewline
Forbidden functions: & None\tabularnewline
Remarks: & n/a\tabularnewline
\bottomrule
\end{longtable}

This is a bonus exercise! You are about to discover the \texttt{yield}
operator!\\
So let's create a function called \texttt{ft\_progress(lst)}.

The function will display the progress of a \texttt{for} loop.

\begin{Shaded}
\begin{Highlighting}[]
\NormalTok{listy }\OperatorTok{=} \BuiltInTok{range}\NormalTok{(}\DecValTok{1000}\NormalTok{)}
\NormalTok{ret }\OperatorTok{=} \DecValTok{0}
\ControlFlowTok{for}\NormalTok{ elem }\KeywordTok{in}\NormalTok{ ft_progress(listy):}
\NormalTok{    ret }\OperatorTok{+=}\NormalTok{ (elem }\OperatorTok{+} \DecValTok{3}\NormalTok{) }\OperatorTok{%} \DecValTok{5}
\NormalTok{    sleep(}\FloatTok{0.01}\NormalTok{)}
\BuiltInTok{print}\NormalTok{()}
\BuiltInTok{print}\NormalTok{(ret)}
\end{Highlighting}
\end{Shaded}

\begin{Shaded}
\begin{Highlighting}[]
\NormalTok{> python loading.py}
\NormalTok{ETA: 8.67s [ 23%][=====>                  ] 233/1000 | elapsed time 2.33s}
\NormalTok{...}
\NormalTok{2000}
\end{Highlighting}
\end{Shaded}

\begin{Shaded}
\begin{Highlighting}[]
\NormalTok{listy }\OperatorTok{=} \BuiltInTok{range}\NormalTok{(}\DecValTok{3333}\NormalTok{)}
\NormalTok{ret }\OperatorTok{=} \DecValTok{0}
\ControlFlowTok{for}\NormalTok{ elem }\KeywordTok{in}\NormalTok{ ft_progress(listy):}
\NormalTok{    ret }\OperatorTok{+=}\NormalTok{ elem}
\NormalTok{    sleep(}\FloatTok{0.005}\NormalTok{)}
\BuiltInTok{print}\NormalTok{()}
\BuiltInTok{print}\NormalTok{(ret)}
\end{Highlighting}
\end{Shaded}

\begin{Shaded}
\begin{Highlighting}[]
\NormalTok{> python loading.py}
\NormalTok{ETA: 14.67s [  9%][=>                      ] 327/3333 | elapsed time 1.33s}
\NormalTok{...}
\NormalTok{5552778}
\end{Highlighting}
\end{Shaded}

\clearpage

\end{document}
