\PassOptionsToPackage{unicode=true}{hyperref} % options for packages loaded elsewhere
\PassOptionsToPackage{hyphens}{url}
\PassOptionsToPackage{dvipsnames,svgnames*,x11names*}{xcolor}
%
\documentclass[]{article}
\usepackage{lmodern}
\usepackage{amssymb,amsmath}
\usepackage{ifxetex,ifluatex}
\usepackage{fixltx2e} % provides \textsubscript
\ifnum 0\ifxetex 1\fi\ifluatex 1\fi=0 % if pdftex
  \usepackage[T1]{fontenc}
  \usepackage[utf8x]{inputenc}
  \usepackage{textcomp} % provides euro and other symbols
\else % if luatex or xelatex
  \usepackage{unicode-math}
  \defaultfontfeatures{Ligatures=TeX,Scale=MatchLowercase}
\fi
% use upquote if available, for straight quotes in verbatim environments
\IfFileExists{upquote.sty}{\usepackage{upquote}}{}
% use microtype if available
\IfFileExists{microtype.sty}{%
\usepackage[]{microtype}
\UseMicrotypeSet[protrusion]{basicmath} % disable protrusion for tt fonts
}{}
\IfFileExists{parskip.sty}{%
\usepackage{parskip}
}{% else
\setlength{\parindent}{0pt}
\setlength{\parskip}{6pt plus 2pt minus 1pt}
}
\usepackage{xcolor}
\usepackage{hyperref}
\hypersetup{
            colorlinks=true,
            linkcolor=Maroon,
            citecolor=Blue,
            urlcolor=blue,
            breaklinks=true}
\urlstyle{same}  % don't use monospace font for urls
\usepackage{color}
\usepackage{fancyvrb}
\newcommand{\VerbBar}{|}
\newcommand{\VERB}{\Verb[commandchars=\\\{\}]}
\DefineVerbatimEnvironment{Highlighting}{Verbatim}{commandchars=\\\{\}}
% Add ',fontsize=\small' for more characters per line
\usepackage{framed}
\definecolor{shadecolor}{RGB}{35,38,41}
\newenvironment{Shaded}{\begin{snugshade}}{\end{snugshade}}
\newcommand{\AlertTok}[1]{\textcolor[rgb]{0.58,0.85,0.30}{#1}}
\newcommand{\AnnotationTok}[1]{\textcolor[rgb]{0.25,0.50,0.35}{#1}}
\newcommand{\AttributeTok}[1]{\textcolor[rgb]{0.16,0.50,0.73}{#1}}
\newcommand{\BaseNTok}[1]{\textcolor[rgb]{0.96,0.45,0.00}{#1}}
\newcommand{\BuiltInTok}[1]{\textcolor[rgb]{0.50,0.55,0.55}{#1}}
\newcommand{\CharTok}[1]{\textcolor[rgb]{0.24,0.68,0.91}{#1}}
\newcommand{\CommentTok}[1]{\textcolor[rgb]{0.48,0.49,0.49}{#1}}
\newcommand{\CommentVarTok}[1]{\textcolor[rgb]{0.50,0.55,0.55}{#1}}
\newcommand{\ConstantTok}[1]{\textcolor[rgb]{0.15,0.68,0.68}{#1}}
\newcommand{\ControlFlowTok}[1]{\textcolor[rgb]{0.99,0.74,0.29}{#1}}
\newcommand{\DataTypeTok}[1]{\textcolor[rgb]{0.16,0.50,0.73}{#1}}
\newcommand{\DecValTok}[1]{\textcolor[rgb]{0.96,0.45,0.00}{#1}}
\newcommand{\DocumentationTok}[1]{\textcolor[rgb]{0.64,0.20,0.25}{#1}}
\newcommand{\ErrorTok}[1]{\textcolor[rgb]{0.85,0.27,0.33}{#1}}
\newcommand{\ExtensionTok}[1]{\textcolor[rgb]{0.00,0.60,1.00}{#1}}
\newcommand{\FloatTok}[1]{\textcolor[rgb]{0.96,0.45,0.00}{#1}}
\newcommand{\FunctionTok}[1]{\textcolor[rgb]{0.56,0.27,0.68}{#1}}
\newcommand{\ImportTok}[1]{\textcolor[rgb]{0.15,0.68,0.38}{#1}}
\newcommand{\InformationTok}[1]{\textcolor[rgb]{0.77,0.36,0.00}{#1}}
\newcommand{\KeywordTok}[1]{\textcolor[rgb]{0.81,0.81,0.76}{#1}}
\newcommand{\NormalTok}[1]{\textcolor[rgb]{0.81,0.81,0.76}{#1}}
\newcommand{\OperatorTok}[1]{\textcolor[rgb]{0.81,0.81,0.76}{#1}}
\newcommand{\OtherTok}[1]{\textcolor[rgb]{0.15,0.68,0.38}{#1}}
\newcommand{\PreprocessorTok}[1]{\textcolor[rgb]{0.15,0.68,0.38}{#1}}
\newcommand{\RegionMarkerTok}[1]{\textcolor[rgb]{0.16,0.50,0.73}{#1}}
\newcommand{\SpecialCharTok}[1]{\textcolor[rgb]{0.24,0.68,0.91}{#1}}
\newcommand{\SpecialStringTok}[1]{\textcolor[rgb]{0.85,0.27,0.33}{#1}}
\newcommand{\StringTok}[1]{\textcolor[rgb]{0.96,0.31,0.31}{#1}}
\newcommand{\VariableTok}[1]{\textcolor[rgb]{0.15,0.68,0.68}{#1}}
\newcommand{\VerbatimStringTok}[1]{\textcolor[rgb]{0.85,0.27,0.33}{#1}}
\newcommand{\WarningTok}[1]{\textcolor[rgb]{0.85,0.27,0.33}{#1}}
\usepackage{longtable,booktabs}
% Fix footnotes in tables (requires footnote package)
\IfFileExists{footnote.sty}{\usepackage{footnote}\makesavenoteenv{longtable}}{}
\usepackage{graphicx,grffile}
\makeatletter
\def\maxwidth{\ifdim\Gin@nat@width>\linewidth\linewidth\else\Gin@nat@width\fi}
\def\maxheight{\ifdim\Gin@nat@height>\textheight\textheight\else\Gin@nat@height\fi}
\makeatother
% Scale images if necessary, so that they will not overflow the page
% margins by default, and it is still possible to overwrite the defaults
% using explicit options in \includegraphics[width, height, ...]{}
\setkeys{Gin}{width=\maxwidth,height=\maxheight,keepaspectratio}
\setlength{\emergencystretch}{3em}  % prevent overfull lines
\providecommand{\tightlist}{%
  \setlength{\itemsep}{0pt}\setlength{\parskip}{0pt}}
\setcounter{secnumdepth}{0}
% Redefines (sub)paragraphs to behave more like sections
\ifx\paragraph\undefined\else
\let\oldparagraph\paragraph
\renewcommand{\paragraph}[1]{\oldparagraph{#1}\mbox{}}
\fi
\ifx\subparagraph\undefined\else
\let\oldsubparagraph\subparagraph
\renewcommand{\subparagraph}[1]{\oldsubparagraph{#1}\mbox{}}
\fi

% set default figure placement to htbp
\makeatletter
\def\fps@figure{h}
\makeatother


\date{}

%%%%%%%%%%%%%%%%%%%%%%%%%%%%%%%%%%%%%%%%%%%%%%%%%%%%%%%%%%%%%%%%%%%%%%%%%%%%%%%%
%%%%%%%%%%%%%%%%%%%%%%%%%%%%%%%% Added packages %%%%%%%%%%%%%%%%%%%%%%%%%%%%%%%%
%%%%%%%%%%%%%%%%%%%%%%%%%%%%%%%%%%%%%%%%%%%%%%%%%%%%%%%%%%%%%%%%%%%%%%%%%%%%%%%%

\setcounter{MaxMatrixCols}{20}
\usepackage{cancel}
\usepackage{calc}
\usepackage{eso-pic}
\newlength{\PageFrameTopMargin}
\newlength{\PageFrameBottomMargin}
\newlength{\PageFrameLeftMargin}
\newlength{\PageFrameRightMargin}

\setlength{\PageFrameTopMargin}{1.5cm}
\setlength{\PageFrameBottomMargin}{1cm}
\setlength{\PageFrameLeftMargin}{1cm}
\setlength{\PageFrameRightMargin}{1cm}

\makeatletter

\newlength{\Page@FrameHeight}
\newlength{\Page@FrameWidth}

\AddToShipoutPicture{
  \thinlines
  \setlength{\Page@FrameHeight}{\paperheight-\PageFrameTopMargin-\PageFrameBottomMargin}
  \setlength{\Page@FrameWidth}{\paperwidth-\PageFrameLeftMargin-\PageFrameRightMargin}
  \put(\strip@pt\PageFrameLeftMargin,\strip@pt\PageFrameTopMargin){
    \framebox(\strip@pt\Page@FrameWidth, \strip@pt\Page@FrameHeight){}}}

\makeatother

\usepackage{fvextra}
\DefineVerbatimEnvironment{Highlighting}{Verbatim}{breaklines,breakanywhere,commandchars=\\\{\}}

\usepackage{graphicx}

\usepackage[a4paper, total={6in, 8in}]{geometry}
\geometry{hmargin=2cm,vmargin=2cm}

\usepackage{sectsty}

\sectionfont{\centering\Huge}
\subsectionfont{\Large}
\subsubsectionfont{\large}

\usepackage{titlesec}
\titlespacing*{\section}
{0pt}{5.5ex plus 1ex minus .2ex}{4.3ex plus .2ex}

\tolerance=1
\emergencystretch=\maxdimen
\hyphenpenalty=10000
\hbadness=10000

%%%%%%%%%%%%%%%%%%%%%%%%%%%%%%%%%%%%%%%%%%%%%%%%%%%%%%%%%%%%%%%%%%%%%%%%%%%%%%%%
%%%%%%%%%%%%%%%%%%%%%%%%%%%%%%%%%%%%%%%%%%%%%%%%%%%%%%%%%%%%%%%%%%%%%%%%%%%%%%%%

\begin{document}

%%%%%%%%%%%%%%%%%%%%%%%%%%%%%%%%%%%%%%%%%%%%%%%%%%%%%%%%%%%%%%%%%%%%%%%%%%%%%%%%
%%%%%%%%%%%%%%%%%%%%%%%%%%%%%%%% Added lines %%%%%%%%%%%%%%%%%%%%%%%%%%%%%%%%%%%
%%%%%%%%%%%%%%%%%%%%%%%%%%%%%%%%%%%%%%%%%%%%%%%%%%%%%%%%%%%%%%%%%%%%%%%%%%%%%%%%

\vspace*{2cm}
\begin{center}
    \textsc{\fontsize{40}{48} \bfseries }\\[0.6cm]
    \textsc{\fontsize{39}{48} \bfseries { %bootcamp_title
Python Machine-Learning
    }}\\[0.3cm]
\end{center}
\vspace{3cm}

\begin{center}
\includegraphics[width=200pt]{assets/logo-42-ai.png}{\centering}
\end{center}

\vspace*{2cm}
\begin{center}
    \textsc{\fontsize{32}{48} \bfseries %day_number
Module 04    
    }\\[0.6cm]
    \textsc{\fontsize{32}{48} \bfseries %day_title
Pandas    
    }\\[0.3cm]
\end{center}
\vspace{3cm}

\pagenumbering{gobble}
\newpage

%%% >>>>> Page de garde
\setcounter{page}{1}
\pagenumbering{arabic}

%%%%%%%%%%%%%%%%%%%%%%%%%%%%%%%%%%%%%%%%%%%%%%%%%%%%%%%%%%%%%%%%%%%%%%%%%%%%%%%%
%%%%%%%%%%%%%%%%%%%%%%%%%%%%%%%%%%%%%%%%%%%%%%%%%%%%%%%%%%%%%%%%%%%%%%%%%%%%%%%%


\hypertarget{module-04---pandas}{%
\section{Module 04 - Pandas}\label{module-04---pandas}}

Today you will learn how to use a Python library that will allow you to
manipulate dataframes.

\hypertarget{notions-of-the-day}{%
\subsection{Notions of the day}\label{notions-of-the-day}}

Pandas! And Bamboos!

\hypertarget{general-rules}{%
\subsection{General rules}\label{general-rules}}

\begin{itemize}
\item
  Use the Pandas Library.
\item
  The version of Python to use is 3.7, you can check the version of
  Python with the following command: \texttt{python\ -V}
\item
  The norm: during this module you will follow the
  \href{https://www.python.org/dev/peps/pep-0008/}{PEP 8 standards}. You
  can install \href{https://pypi.org/project/pycodestyle}{pycodestyle}
  which is a tool to check your Python code.
\item
  The function eval is never allowed.
\item
  The exercises are ordered from the easiest to the hardest.
\item
  Your exercises are going to be evaluated by someone else, so make sure
  that your variable names and function names are appropriate and civil.
\item
  Your manual is the internet.
\item
  You can also ask questions in the dedicated channel in the 42 AI
  Slack: 42-ai.slack.com.
\item
  If you find any issue or mistakes in the subject please create an
  issue on our
  \href{https://github.com/42-AI/bootcamp_python/issues}{dedicated
  repository on Github}.
\end{itemize}

\hypertarget{helper}{%
\subsection{Helper}\label{helper}}

For this day you will use the dataset \texttt{athlete\_events.csv}
provided in the \texttt{resources} folder.

\begin{Shaded}
\begin{Highlighting}[]
\NormalTok{pip install pandas}
\end{Highlighting}
\end{Shaded}

Ensure that you have the right Python version.

\begin{Shaded}
\begin{Highlighting}[]
\NormalTok{> which python}
\NormalTok{/goinfre/miniconda/bin/python}
\NormalTok{> python -V}
\NormalTok{Python 3.7.*}
\NormalTok{> which pip}
\NormalTok{/goinfre/miniconda/bin/pip}
\end{Highlighting}
\end{Shaded}

\hypertarget{exercise-00---fileloader}{%
\subsubsection{Exercise 00 -
FileLoader}\label{exercise-00---fileloader}}

\hypertarget{exercise-01---youngestfellah}{%
\subsubsection{Exercise 01 -
YoungestFellah}\label{exercise-01---youngestfellah}}

\hypertarget{exercise-02---proportionbysport}{%
\subsubsection{Exercise 02 -
ProportionBySport}\label{exercise-02---proportionbysport}}

\hypertarget{exercise-03---howmanymedals}{%
\subsubsection{Exercise 03 -
HowManyMedals}\label{exercise-03---howmanymedals}}

\hypertarget{exercise-04---spatiotemporaldata}{%
\subsubsection{Exercise 04 -
SpatioTemporalData}\label{exercise-04---spatiotemporaldata}}

\hypertarget{exercise-05---howmanymedalsbycountry}{%
\subsubsection{Exercise 05 -
HowManyMedalsByCountry}\label{exercise-05---howmanymedalsbycountry}}

\hypertarget{exercise-06---myplotlib}{%
\subsubsection{Exercise 06 - MyPlotLib}\label{exercise-06---myplotlib}}

\hypertarget{exercise-07---komparator}{%
\subsubsection{Exercise 07 -
Komparator}\label{exercise-07---komparator}}

\clearpage

\hypertarget{exercise-00---fileloader-1}{%
\section{Exercise 00 - FileLoader}\label{exercise-00---fileloader-1}}

\begin{longtable}[]{@{}rl@{}}
\toprule
\endhead
Turn-in directory: & ex00\tabularnewline
Files to turn in: & FileLoader.py\tabularnewline
Allowed libraries: & Pandas\tabularnewline
Remarks: & Be as lazy as possible\ldots{}\tabularnewline
\bottomrule
\end{longtable}

Write a class named \texttt{FileLoader} which implements the following
methods:

\begin{itemize}
\item
  \texttt{load(path)} : takes as an argument the file path of the
  dataset to load, displays a message specifying the dimensions of the
  dataset (e.g.~340 x 500) and returns the dataset loaded as a
  pandas.DataFrame.
\item
  \texttt{display(df,\ n)} : takes a pandas.DataFrame and an integer as
  arguments, displays the first n rows of the dataset if n is positive,
  or the last n rows if n is negative.
\end{itemize}

\begin{Shaded}
\begin{Highlighting}[]
\OperatorTok{>>>} \ImportTok{from}\NormalTok{ FileLoader }\ImportTok{import}\NormalTok{ FileLoader}
\OperatorTok{>>>}\NormalTok{ loader }\OperatorTok{=}\NormalTok{ FileLoader()}
\OperatorTok{>>>}\NormalTok{ data }\OperatorTok{=}\NormalTok{ loader.load(}\StringTok{"../data/adult_data.csv"}\NormalTok{)}
\NormalTok{Loading dataset of dimensions }\DecValTok{32561}\NormalTok{ x }\DecValTok{15}
\OperatorTok{>>>}\NormalTok{ loader.display(data, }\DecValTok{12}\NormalTok{)}
\NormalTok{age         workclass  fnlwgt  ... hours}\OperatorTok{-}\NormalTok{per}\OperatorTok{-}\NormalTok{week  native}\OperatorTok{-}\NormalTok{country salary}
\DecValTok{0}    \DecValTok{39}\NormalTok{         State}\OperatorTok{-}\NormalTok{gov   }\DecValTok{77516}\NormalTok{  ...             }\DecValTok{40}\NormalTok{   United}\OperatorTok{-}\NormalTok{States  }\OperatorTok{<=}\NormalTok{50K}
\DecValTok{1}    \DecValTok{50}\NormalTok{  Self}\OperatorTok{-}\NormalTok{emp}\OperatorTok{-}\KeywordTok{not}\OperatorTok{-}\NormalTok{inc   }\DecValTok{83311}\NormalTok{  ...             }\DecValTok{13}\NormalTok{   United}\OperatorTok{-}\NormalTok{States  }\OperatorTok{<=}\NormalTok{50K}
\DecValTok{2}    \DecValTok{38}\NormalTok{           Private  }\DecValTok{215646}\NormalTok{  ...             }\DecValTok{40}\NormalTok{   United}\OperatorTok{-}\NormalTok{States  }\OperatorTok{<=}\NormalTok{50K}
\DecValTok{3}    \DecValTok{53}\NormalTok{           Private  }\DecValTok{234721}\NormalTok{  ...             }\DecValTok{40}\NormalTok{   United}\OperatorTok{-}\NormalTok{States  }\OperatorTok{<=}\NormalTok{50K}
\DecValTok{4}    \DecValTok{28}\NormalTok{           Private  }\DecValTok{338409}\NormalTok{  ...             }\DecValTok{40}\NormalTok{            Cuba  }\OperatorTok{<=}\NormalTok{50K}
\DecValTok{5}    \DecValTok{37}\NormalTok{           Private  }\DecValTok{284582}\NormalTok{  ...             }\DecValTok{40}\NormalTok{   United}\OperatorTok{-}\NormalTok{States  }\OperatorTok{<=}\NormalTok{50K}
\DecValTok{6}    \DecValTok{49}\NormalTok{           Private  }\DecValTok{160187}\NormalTok{  ...             }\DecValTok{16}\NormalTok{         Jamaica  }\OperatorTok{<=}\NormalTok{50K}
\DecValTok{7}    \DecValTok{52}\NormalTok{  Self}\OperatorTok{-}\NormalTok{emp}\OperatorTok{-}\KeywordTok{not}\OperatorTok{-}\NormalTok{inc  }\DecValTok{209642}\NormalTok{  ...             }\DecValTok{45}\NormalTok{   United}\OperatorTok{-}\NormalTok{States   }\OperatorTok{>}\NormalTok{50K}
\DecValTok{8}    \DecValTok{31}\NormalTok{           Private   }\DecValTok{45781}\NormalTok{  ...             }\DecValTok{50}\NormalTok{   United}\OperatorTok{-}\NormalTok{States   }\OperatorTok{>}\NormalTok{50K}
\DecValTok{9}    \DecValTok{42}\NormalTok{           Private  }\DecValTok{159449}\NormalTok{  ...             }\DecValTok{40}\NormalTok{   United}\OperatorTok{-}\NormalTok{States   }\OperatorTok{>}\NormalTok{50K}
\DecValTok{10}   \DecValTok{37}\NormalTok{           Private  }\DecValTok{280464}\NormalTok{  ...             }\DecValTok{80}\NormalTok{   United}\OperatorTok{-}\NormalTok{States   }\OperatorTok{>}\NormalTok{50K}
\DecValTok{11}   \DecValTok{30}\NormalTok{         State}\OperatorTok{-}\NormalTok{gov  }\DecValTok{141297}\NormalTok{  ...             }\DecValTok{40}\NormalTok{           India   }\OperatorTok{>}\NormalTok{50K}

\NormalTok{[}\DecValTok{12}\NormalTok{ rows x }\DecValTok{15}\NormalTok{ columns]}
\end{Highlighting}
\end{Shaded}

NB: Your terminal may display more columns if the window is wider.

\clearpage

\hypertarget{exercise-01---youngestfellah-1}{%
\section{Exercise 01 -
YoungestFellah}\label{exercise-01---youngestfellah-1}}

\begin{longtable}[]{@{}rl@{}}
\toprule
\endhead
Turn-in directory: & ex01\tabularnewline
Files to turn in: & FileLoader.py, YoungestFellah.py\tabularnewline
Allowed libraries: & Pandas\tabularnewline
Remarks: & n/a\tabularnewline
\bottomrule
\end{longtable}

This exercise uses the following dataset: \texttt{athlete\_events.csv}

Write a function \texttt{youngestFellah} that takes two arguments:

\begin{itemize}
\item
  a pandas.DataFrame which contains the dataset
\item
  an Olympic year.
\end{itemize}

The function returns a dictionary containing the age of the youngest
woman and man who took part in the Olympics on that year. The name of
the dictionary's keys is up to you, but it must be self-explanatory.

\textbf{Example:}

\begin{Shaded}
\begin{Highlighting}[]
\OperatorTok{>>>} \ImportTok{from}\NormalTok{ FileLoader }\ImportTok{import}\NormalTok{ FileLoader}
\OperatorTok{>>>}\NormalTok{ loader }\OperatorTok{=}\NormalTok{ FileLoader()}
\OperatorTok{>>>}\NormalTok{ data }\OperatorTok{=}\NormalTok{ loader.load(}\StringTok{'../data/athlete_events.csv'}\NormalTok{)}
\NormalTok{Loading dataset of dimensions }\DecValTok{271116}\NormalTok{ x }\DecValTok{15}
\OperatorTok{>>>} \ImportTok{from}\NormalTok{ YoungestFellah }\ImportTok{import}\NormalTok{ youngestFellah}
\OperatorTok{>>>}\NormalTok{ youngestFellah(data, }\DecValTok{2004}\NormalTok{)}
\NormalTok{\{}\StringTok{'f'}\NormalTok{: }\FloatTok{13.0}\NormalTok{, }\StringTok{'m'}\NormalTok{: }\FloatTok{14.0}\NormalTok{\}}
\end{Highlighting}
\end{Shaded}

\clearpage

\hypertarget{exercise-02---proportionbysport-1}{%
\section{Exercise 02 -
ProportionBySport}\label{exercise-02---proportionbysport-1}}

\begin{longtable}[]{@{}rl@{}}
\toprule
\endhead
Turn-in directory: & ex02\tabularnewline
Files to turn in: & FileLoader.py, ProportionBySport.py\tabularnewline
Allowed libraries: & Pandas\tabularnewline
Remarks: & n/a\tabularnewline
\bottomrule
\end{longtable}

This exercise uses the dataset \texttt{athlete\_events.csv}

Write a function \textbf{proportionBySport} that takes four arguments:

\begin{itemize}
\item
  a pandas.DataFrame of the dataset
\item
  an olympic year
\item
  a sport
\item
  a gender.
\end{itemize}

The function returns a float corresponding to the proportion
(percentage) of participants who played the given sport among the
participants of the given gender.

The function answers questions like the following : ``What was the
percentage of female basketball players among all the female
participants of the 2016 Olympics?''

Hint: here and further, if needed, drop duplicated sportspeople to count
only unique ones. Beware to call the dropping function at the right
moment and with the right parameters, in order not to omit any
individuals.

\textbf{Example:}

\begin{Shaded}
\begin{Highlighting}[]
\OperatorTok{>>>} \ImportTok{from}\NormalTok{ FileLoader }\ImportTok{import}\NormalTok{ FileLoader}
\OperatorTok{>>>}\NormalTok{ loader }\OperatorTok{=}\NormalTok{ FileLoader()}
\OperatorTok{>>>}\NormalTok{ data }\OperatorTok{=}\NormalTok{ loader.load(}\StringTok{'../data/athlete_events.csv'}\NormalTok{)}
\NormalTok{Loading dataset of dimensions }\DecValTok{271116}\NormalTok{ x }\DecValTok{15}
\OperatorTok{>>>} \ImportTok{from}\NormalTok{ ProportionBySport }\ImportTok{import}\NormalTok{ proportionBySport}
\OperatorTok{>>>}\NormalTok{ proportionBySport(data, }\DecValTok{2004}\NormalTok{, }\StringTok{'Tennis'}\NormalTok{, }\StringTok{'F'}\NormalTok{)}
\FloatTok{0.01935634328358209}
\end{Highlighting}
\end{Shaded}

We assume that we are always using appropriate arguments as input, and
thus do not need to handle input errors.

\clearpage

\hypertarget{exercise-3---howmanymedals}{%
\section{Exercise 3 - HowManyMedals}\label{exercise-3---howmanymedals}}

\begin{longtable}[]{@{}rl@{}}
\toprule
\endhead
Turn-in directory: & ex03\tabularnewline
Files to turn in: & FileLoader.py, HowManyMedals.py\tabularnewline
Allowed libraries: & Pandas\tabularnewline
Remarks: & n/a\tabularnewline
\bottomrule
\end{longtable}

This exercise uses the following dataset: \texttt{athlete\_events.csv}

Write a function \texttt{howManyMedals} that takes two arguments:

\begin{itemize}
\item
  a pandas.DataFrame which contains the dataset
\item
  a participant name.
\end{itemize}

The function returns a dictionary of dictionaries giving the number and
type of medals for each year during which the participant won medals.\\
The keys of the main dictionary are the Olympic games years. In each
year's dictionary, the keys are `G', `S', `B' corresponding to the type
of medals won (gold, silver, bronze). The innermost values correspond to
the number of medals of a given type won for a given year.

\textbf{Example:}

\begin{Shaded}
\begin{Highlighting}[]
\OperatorTok{>>>} \ImportTok{from}\NormalTok{ FileLoader }\ImportTok{import}\NormalTok{ FileLoader}
\OperatorTok{>>>}\NormalTok{ loader }\OperatorTok{=}\NormalTok{ FileLoader()}
\OperatorTok{>>>}\NormalTok{ data }\OperatorTok{=}\NormalTok{ loader.load(}\StringTok{'../data/athlete_events.csv'}\NormalTok{)}
\NormalTok{Loading dataset of dimensions }\DecValTok{271116}\NormalTok{ x }\DecValTok{15}
\OperatorTok{>>>} \ImportTok{from}\NormalTok{ HowManyMedals }\ImportTok{import}\NormalTok{ howManyMedals}
\OperatorTok{>>>}\NormalTok{ howManyMedals(data, }\StringTok{'Kjetil Andr Aamodt'}\NormalTok{)}
\NormalTok{\{}\DecValTok{1992}\NormalTok{: \{}\StringTok{'G'}\NormalTok{: }\DecValTok{1}\NormalTok{, }\StringTok{'S'}\NormalTok{: }\DecValTok{0}\NormalTok{, }\StringTok{'B'}\NormalTok{: }\DecValTok{1}\NormalTok{\}, }\DecValTok{1994}\NormalTok{: \{}\StringTok{'G'}\NormalTok{: }\DecValTok{0}\NormalTok{, }\StringTok{'S'}\NormalTok{: }\DecValTok{2}\NormalTok{, }\StringTok{'B'}\NormalTok{: }\DecValTok{1}\NormalTok{\}, }\DecValTok{1998}\NormalTok{: \{}\StringTok{'G'}\NormalTok{: }\DecValTok{0}\NormalTok{, }\StringTok{'S'}\NormalTok{: }\DecValTok{0}\NormalTok{, }\StringTok{'B'}\NormalTok{: }\DecValTok{0}\NormalTok{\}, }\DecValTok{2002}\NormalTok{: \{}\StringTok{'G'}\NormalTok{: }\DecValTok{2}\NormalTok{, }\StringTok{'S'}\NormalTok{: }\DecValTok{0}\NormalTok{, }\StringTok{'B'}\NormalTok{: }\DecValTok{0}\NormalTok{\}, }\DecValTok{2006}\NormalTok{: \{}\StringTok{'G'}\NormalTok{: }\DecValTok{1}\NormalTok{, }\StringTok{'S'}\NormalTok{: }\DecValTok{0}\NormalTok{, }\StringTok{'B'}\NormalTok{: }\DecValTok{0}\NormalTok{\}\}}
\end{Highlighting}
\end{Shaded}

\clearpage

\hypertarget{exercise-04---spatiotemporaldata-1}{%
\section{Exercise 04 -
SpatioTemporalData}\label{exercise-04---spatiotemporaldata-1}}

\begin{longtable}[]{@{}rl@{}}
\toprule
\endhead
Turn-in directory: & ex04\tabularnewline
Files to turn in: & FileLoader.py, SpatioTemporalData.py\tabularnewline
Allowed libraries: & Pandas\tabularnewline
Remarks: & n/a\tabularnewline
\bottomrule
\end{longtable}

This exercise uses the dataset \texttt{athlete\_events.csv}

Write a class called \texttt{SpatioTemporalData} that takes a dataset
(pandas.DataFrame) as argument in its constructor and implements the
following methods:

\begin{itemize}
\item
  \texttt{when(location)} : takes a location as an argument and returns
  a list containing the years where games were held in the given
  location.
\item
  \texttt{where(date)} : takes a date as an argument and returns the
  location where the Olympics took place in the given year.
\end{itemize}

\textbf{Example:}

\begin{Shaded}
\begin{Highlighting}[]
\OperatorTok{>>>} \ImportTok{from}\NormalTok{ FileLoader }\ImportTok{import}\NormalTok{ FileLoader}
\OperatorTok{>>>}\NormalTok{ loader }\OperatorTok{=}\NormalTok{ FileLoader()}
\OperatorTok{>>>}\NormalTok{ data }\OperatorTok{=}\NormalTok{ loader.load(}\StringTok{'../data/athlete_events.csv'}\NormalTok{)}
\NormalTok{Loading dataset of dimensions }\DecValTok{271116}\NormalTok{ x }\DecValTok{15}
\OperatorTok{>>>} \ImportTok{from}\NormalTok{ SpatioTemporalData }\ImportTok{import}\NormalTok{ SpatioTemporalData}
\OperatorTok{>>>}\NormalTok{ sp }\OperatorTok{=}\NormalTok{ SpatioTemporalData(data)}
\OperatorTok{>>>}\NormalTok{ sp.where(}\DecValTok{1896}\NormalTok{)}
\NormalTok{[}\StringTok{'Athina'}\NormalTok{]}
\OperatorTok{>>>}\NormalTok{ sp.where(}\DecValTok{2016}\NormalTok{)}
\NormalTok{[}\StringTok{'Rio de Janeiro'}\NormalTok{]}
\OperatorTok{>>>}\NormalTok{ sp.when(}\StringTok{'Athina'}\NormalTok{)}
\NormalTok{[}\DecValTok{2004}\NormalTok{, }\DecValTok{1906}\NormalTok{, }\DecValTok{1896}\NormalTok{]}
\OperatorTok{>>>}\NormalTok{ sp.when(}\StringTok{'Paris'}\NormalTok{)}
\NormalTok{[}\DecValTok{1900}\NormalTok{, }\DecValTok{1924}\NormalTok{]}
\end{Highlighting}
\end{Shaded}

\clearpage

\hypertarget{exercise-05---howmanymedalsbycountry-1}{%
\section{Exercise 05 -
HowManyMedalsByCountry}\label{exercise-05---howmanymedalsbycountry-1}}

\begin{longtable}[]{@{}rl@{}}
\toprule
\endhead
Turn-in directory: & ex05\tabularnewline
Files to turn in: & FileLoader.py,
HowManyMedalsByCountry.py\tabularnewline
Allowed libraries: & Pandas\tabularnewline
Remarks: & n/a\tabularnewline
\bottomrule
\end{longtable}

This exercise uses the following dataset: \texttt{athlete\_events.csv}

Write a function \texttt{howManyMedalsByCountry} that takes two
arguments:

\begin{itemize}
\item
  a pandas.DataFrame which contains the dataset
\item
  a country name.
\end{itemize}

The function returns a dictionary of dictionaries giving the number and
type of medal for each competition where the country team earned
medals.\\
The keys of the main dictionary are the Olympic games' years. In each
year's dictionary, the key are `G', `S', `B' corresponding to the type
of medals won.

Duplicated medals per team games should be handled and not counted
twice.

\textbf{Example:}

\begin{Shaded}
\begin{Highlighting}[]
\OperatorTok{>>>} \ImportTok{from}\NormalTok{ FileLoader }\ImportTok{import}\NormalTok{ FileLoader}
\OperatorTok{>>>}\NormalTok{ loader }\OperatorTok{=}\NormalTok{ FileLoader()}
\OperatorTok{>>>}\NormalTok{ data }\OperatorTok{=}\NormalTok{ loader.load(}\StringTok{'../data/athlete_events.csv'}\NormalTok{)}
\NormalTok{Loading dataset of dimensions }\DecValTok{271116}\NormalTok{ x }\DecValTok{15}
\OperatorTok{>>>} \ImportTok{from}\NormalTok{ HowManyMedalsByCountry }\ImportTok{import}\NormalTok{ howManyMedalsByCountry}
\OperatorTok{>>>}\NormalTok{ howManyMedalsByCountry(data, }\StringTok{'Martian Federation'}\NormalTok{)}
\NormalTok{\{}\DecValTok{2192}\NormalTok{: \{}\StringTok{'G'}\NormalTok{: }\DecValTok{17}\NormalTok{, }\StringTok{'S'}\NormalTok{: }\DecValTok{14}\NormalTok{, }\StringTok{'B'}\NormalTok{: }\DecValTok{23}\NormalTok{\}, }\DecValTok{2196}\NormalTok{: \{}\StringTok{'G'}\NormalTok{: }\DecValTok{8}\NormalTok{, }\StringTok{'S'}\NormalTok{: }\DecValTok{21}\NormalTok{, }\StringTok{'B'}\NormalTok{: }\DecValTok{19}\NormalTok{\}, }\DecValTok{2200}\NormalTok{: \{}\StringTok{'G'}\NormalTok{: }\DecValTok{26}\NormalTok{, }\StringTok{'S'}\NormalTok{: }\DecValTok{19}\NormalTok{, }\StringTok{'B'}\NormalTok{: }\DecValTok{7}\NormalTok{\}\}}
\end{Highlighting}
\end{Shaded}

You probably guessed by now that we gave up providing real
examples\ldots{}

If you want real examples, you can easily look online. Do beware that
some medals might be awarded or removed years after the games are over,
for example if a previous medallist was found to have cheated and is
sanctioned. The \texttt{athlete\_events.csv} dataset might not always
take these posterior changes into account.

\clearpage

\hypertarget{exercise-06---myplotlib-1}{%
\section{Exercise 06 - MyPlotLib}\label{exercise-06---myplotlib-1}}

\begin{longtable}[]{@{}rl@{}}
\toprule
\endhead
\begin{minipage}[t]{0.54\columnwidth}\raggedleft
Turn-in directory:\strut
\end{minipage} & \begin{minipage}[t]{0.40\columnwidth}\raggedright
ex06\strut
\end{minipage}\tabularnewline
\begin{minipage}[t]{0.54\columnwidth}\raggedleft
Files to turn in:\strut
\end{minipage} & \begin{minipage}[t]{0.40\columnwidth}\raggedright
MyPlotLib.py\strut
\end{minipage}\tabularnewline
\begin{minipage}[t]{0.54\columnwidth}\raggedleft
Allowed libraries:\strut
\end{minipage} & \begin{minipage}[t]{0.40\columnwidth}\raggedright
Pandas, Matplotlib, Seaborn, Scipy\strut
\end{minipage}\tabularnewline
\begin{minipage}[t]{0.54\columnwidth}\raggedleft
Remarks:\strut
\end{minipage} & \begin{minipage}[t]{0.40\columnwidth}\raggedright
The less work you do, the better! You don't necessarily need all those
libraries to complete the exercise.\strut
\end{minipage}\tabularnewline
\bottomrule
\end{longtable}

This exercise uses the following dataset: \texttt{athlete\_events.csv}

Write a class called \texttt{MyPlotLib}. This class implements different
plotting methods, each of which take two arguments:

\begin{itemize}
\item
  a pandas.DataFrame which contains the dataset
\item
  a list of feature names.
\end{itemize}

Hint: What is a feature?
https://towardsdatascience.com/feature-engineering-for-machine-learning-3a5e293a5114

\begin{itemize}
\item
  \texttt{histogram(data,\ features)} : plots one histogram for each
  numerical feature in the list
\item
  \texttt{density(data,\ features)} : plots the density curve of each
  numerical feature in the list
\item
  \texttt{pair\_plot(data,\ features)} : plots a matrix of subplots
  (also called scatter plot matrix). On each subplot shows a scatter
  plot of one numerical variable against another one. The main diagonal
  of this matrix shows simple histograms.
\item
  \texttt{box\_plot(data,\ features)} : displays a box plot for each
  numerical variable in the dataset.
\end{itemize}

\textbf{Examples}

\begin{figure}
\centering
\includegraphics[width=4.16667in,height=\textheight]{tmp/assets/ex06_histogram.png}
\caption{histogram}
\end{figure}

\begin{figure}
\centering
\includegraphics[width=4.16667in,height=\textheight]{tmp/assets/ex06_density.png}
\caption{density}
\end{figure}

\begin{figure}
\centering
\includegraphics[width=4.16667in,height=\textheight]{tmp/assets/ex06_pair_plot.png}
\caption{pair\_plot}
\end{figure}

\begin{figure}
\centering
\includegraphics[width=4.16667in,height=\textheight]{tmp/assets/ex06_box_plot.png}
\caption{box\_plot}
\end{figure}

\clearpage

\hypertarget{exercise-07---komparator-1}{%
\section{Exercise 07 - Komparator}\label{exercise-07---komparator-1}}

\begin{longtable}[]{@{}rl@{}}
\toprule
\endhead
\begin{minipage}[t]{0.54\columnwidth}\raggedleft
Turn-in directory:\strut
\end{minipage} & \begin{minipage}[t]{0.40\columnwidth}\raggedright
ex07\strut
\end{minipage}\tabularnewline
\begin{minipage}[t]{0.54\columnwidth}\raggedleft
Files to turn in:\strut
\end{minipage} & \begin{minipage}[t]{0.40\columnwidth}\raggedright
Komparator.py, MyPlotLib.py (optional)\strut
\end{minipage}\tabularnewline
\begin{minipage}[t]{0.54\columnwidth}\raggedleft
Allowed libraries:\strut
\end{minipage} & \begin{minipage}[t]{0.40\columnwidth}\raggedright
Pandas, Matplotlib, Seaborn, Scipy\strut
\end{minipage}\tabularnewline
\begin{minipage}[t]{0.54\columnwidth}\raggedleft
Remarks:\strut
\end{minipage} & \begin{minipage}[t]{0.40\columnwidth}\raggedright
The less work you do, the better! You don't necessarily need all those
libraries to complete the exercise.\strut
\end{minipage}\tabularnewline
\bottomrule
\end{longtable}

This exercise uses the following dataset: \texttt{athlete\_events.csv}

Write a class called \texttt{Komparator} whose constructor takes as an
argument a pandas.DataFrame which contains the dataset. The class must
implement the following methods, which take as input two variable names:

\begin{itemize}
\item
  \texttt{compare\_box\_plots(categorical\_var,\ numerical\_var)} :
  displays a series of box plots to compare how the distribution of the
  numerical variable changes if we only consider the subpopulation which
  belongs to each category. There should be as many box plots as
  categories. For example, with Sex and Height, we would compare the
  height distributions of men vs.~women with two box plots.
\item
  \texttt{density(categorical\_var,\ numerical\_var)} : displays the
  density of the numerical variable. Each subpopulation should be
  represented by a separate curve on the graph.
\item
  \texttt{compare\_histograms(categorical\_var,\ numerical\_var)} :
  plots the numerical variable in a separate histogram for each
  category. As a bonus, you can use overlapping histograms with a color
  code.
\end{itemize}

BONUS: Your functions can also accept a list of numerical variables
(instead of just one), and output a comparison plot for each variable in
the list.

\clearpage

\end{document}
