\PassOptionsToPackage{unicode=true}{hyperref} % options for packages loaded elsewhere
\PassOptionsToPackage{hyphens}{url}
\PassOptionsToPackage{dvipsnames,svgnames*,x11names*}{xcolor}
%
\documentclass[]{article}
\usepackage{lmodern}
\usepackage{amssymb,amsmath}
\usepackage{ifxetex,ifluatex}
\usepackage{fixltx2e} % provides \textsubscript
\ifnum 0\ifxetex 1\fi\ifluatex 1\fi=0 % if pdftex
  \usepackage[T1]{fontenc}
  \usepackage[utf8x]{inputenc}
  \usepackage{textcomp} % provides euro and other symbols
\else % if luatex or xelatex
  \usepackage{unicode-math}
  \defaultfontfeatures{Ligatures=TeX,Scale=MatchLowercase}
\fi
% use upquote if available, for straight quotes in verbatim environments
\IfFileExists{upquote.sty}{\usepackage{upquote}}{}
% use microtype if available
\IfFileExists{microtype.sty}{%
\usepackage[]{microtype}
\UseMicrotypeSet[protrusion]{basicmath} % disable protrusion for tt fonts
}{}
\IfFileExists{parskip.sty}{%
\usepackage{parskip}
}{% else
\setlength{\parindent}{0pt}
\setlength{\parskip}{6pt plus 2pt minus 1pt}
}
\usepackage{xcolor}
\usepackage{hyperref}
\hypersetup{
            colorlinks=true,
            linkcolor=Maroon,
            citecolor=Blue,
            urlcolor=blue,
            breaklinks=true}
\urlstyle{same}  % don't use monospace font for urls
\usepackage{color}
\usepackage{fancyvrb}
\newcommand{\VerbBar}{|}
\newcommand{\VERB}{\Verb[commandchars=\\\{\}]}
\DefineVerbatimEnvironment{Highlighting}{Verbatim}{commandchars=\\\{\}}
% Add ',fontsize=\small' for more characters per line
\usepackage{framed}
\definecolor{shadecolor}{RGB}{35,38,41}
\newenvironment{Shaded}{\begin{snugshade}}{\end{snugshade}}
\newcommand{\AlertTok}[1]{\textcolor[rgb]{0.58,0.85,0.30}{#1}}
\newcommand{\AnnotationTok}[1]{\textcolor[rgb]{0.25,0.50,0.35}{#1}}
\newcommand{\AttributeTok}[1]{\textcolor[rgb]{0.16,0.50,0.73}{#1}}
\newcommand{\BaseNTok}[1]{\textcolor[rgb]{0.96,0.45,0.00}{#1}}
\newcommand{\BuiltInTok}[1]{\textcolor[rgb]{0.50,0.55,0.55}{#1}}
\newcommand{\CharTok}[1]{\textcolor[rgb]{0.24,0.68,0.91}{#1}}
\newcommand{\CommentTok}[1]{\textcolor[rgb]{0.48,0.49,0.49}{#1}}
\newcommand{\CommentVarTok}[1]{\textcolor[rgb]{0.50,0.55,0.55}{#1}}
\newcommand{\ConstantTok}[1]{\textcolor[rgb]{0.15,0.68,0.68}{#1}}
\newcommand{\ControlFlowTok}[1]{\textcolor[rgb]{0.99,0.74,0.29}{#1}}
\newcommand{\DataTypeTok}[1]{\textcolor[rgb]{0.16,0.50,0.73}{#1}}
\newcommand{\DecValTok}[1]{\textcolor[rgb]{0.96,0.45,0.00}{#1}}
\newcommand{\DocumentationTok}[1]{\textcolor[rgb]{0.64,0.20,0.25}{#1}}
\newcommand{\ErrorTok}[1]{\textcolor[rgb]{0.85,0.27,0.33}{#1}}
\newcommand{\ExtensionTok}[1]{\textcolor[rgb]{0.00,0.60,1.00}{#1}}
\newcommand{\FloatTok}[1]{\textcolor[rgb]{0.96,0.45,0.00}{#1}}
\newcommand{\FunctionTok}[1]{\textcolor[rgb]{0.56,0.27,0.68}{#1}}
\newcommand{\ImportTok}[1]{\textcolor[rgb]{0.15,0.68,0.38}{#1}}
\newcommand{\InformationTok}[1]{\textcolor[rgb]{0.77,0.36,0.00}{#1}}
\newcommand{\KeywordTok}[1]{\textcolor[rgb]{0.81,0.81,0.76}{#1}}
\newcommand{\NormalTok}[1]{\textcolor[rgb]{0.81,0.81,0.76}{#1}}
\newcommand{\OperatorTok}[1]{\textcolor[rgb]{0.81,0.81,0.76}{#1}}
\newcommand{\OtherTok}[1]{\textcolor[rgb]{0.15,0.68,0.38}{#1}}
\newcommand{\PreprocessorTok}[1]{\textcolor[rgb]{0.15,0.68,0.38}{#1}}
\newcommand{\RegionMarkerTok}[1]{\textcolor[rgb]{0.16,0.50,0.73}{#1}}
\newcommand{\SpecialCharTok}[1]{\textcolor[rgb]{0.24,0.68,0.91}{#1}}
\newcommand{\SpecialStringTok}[1]{\textcolor[rgb]{0.85,0.27,0.33}{#1}}
\newcommand{\StringTok}[1]{\textcolor[rgb]{0.96,0.31,0.31}{#1}}
\newcommand{\VariableTok}[1]{\textcolor[rgb]{0.15,0.68,0.68}{#1}}
\newcommand{\VerbatimStringTok}[1]{\textcolor[rgb]{0.85,0.27,0.33}{#1}}
\newcommand{\WarningTok}[1]{\textcolor[rgb]{0.85,0.27,0.33}{#1}}
\usepackage{longtable,booktabs}
% Fix footnotes in tables (requires footnote package)
\IfFileExists{footnote.sty}{\usepackage{footnote}\makesavenoteenv{longtable}}{}
\usepackage{graphicx,grffile}
\makeatletter
\def\maxwidth{\ifdim\Gin@nat@width>\linewidth\linewidth\else\Gin@nat@width\fi}
\def\maxheight{\ifdim\Gin@nat@height>\textheight\textheight\else\Gin@nat@height\fi}
\makeatother
% Scale images if necessary, so that they will not overflow the page
% margins by default, and it is still possible to overwrite the defaults
% using explicit options in \includegraphics[width, height, ...]{}
\setkeys{Gin}{width=\maxwidth,height=\maxheight,keepaspectratio}
\setlength{\emergencystretch}{3em}  % prevent overfull lines
\providecommand{\tightlist}{%
  \setlength{\itemsep}{0pt}\setlength{\parskip}{0pt}}
\setcounter{secnumdepth}{0}
% Redefines (sub)paragraphs to behave more like sections
\ifx\paragraph\undefined\else
\let\oldparagraph\paragraph
\renewcommand{\paragraph}[1]{\oldparagraph{#1}\mbox{}}
\fi
\ifx\subparagraph\undefined\else
\let\oldsubparagraph\subparagraph
\renewcommand{\subparagraph}[1]{\oldsubparagraph{#1}\mbox{}}
\fi

% set default figure placement to htbp
\makeatletter
\def\fps@figure{h}
\makeatother


\date{}

%%%%%%%%%%%%%%%%%%%%%%%%%%%%%%%%%%%%%%%%%%%%%%%%%%%%%%%%%%%%%%%%%%%%%%%%%%%%%%%%
%%%%%%%%%%%%%%%%%%%%%%%%%%%%%%%% Added packages %%%%%%%%%%%%%%%%%%%%%%%%%%%%%%%%
%%%%%%%%%%%%%%%%%%%%%%%%%%%%%%%%%%%%%%%%%%%%%%%%%%%%%%%%%%%%%%%%%%%%%%%%%%%%%%%%

\setcounter{MaxMatrixCols}{20}
\usepackage{cancel}
\usepackage{calc}
\usepackage{eso-pic}
\newlength{\PageFrameTopMargin}
\newlength{\PageFrameBottomMargin}
\newlength{\PageFrameLeftMargin}
\newlength{\PageFrameRightMargin}

\setlength{\PageFrameTopMargin}{1.5cm}
\setlength{\PageFrameBottomMargin}{1cm}
\setlength{\PageFrameLeftMargin}{1cm}
\setlength{\PageFrameRightMargin}{1cm}

\makeatletter

\newlength{\Page@FrameHeight}
\newlength{\Page@FrameWidth}

\AddToShipoutPicture{
  \thinlines
  \setlength{\Page@FrameHeight}{\paperheight-\PageFrameTopMargin-\PageFrameBottomMargin}
  \setlength{\Page@FrameWidth}{\paperwidth-\PageFrameLeftMargin-\PageFrameRightMargin}
  \put(\strip@pt\PageFrameLeftMargin,\strip@pt\PageFrameTopMargin){
    \framebox(\strip@pt\Page@FrameWidth, \strip@pt\Page@FrameHeight){}}}

\makeatother

\usepackage{fvextra}
\DefineVerbatimEnvironment{Highlighting}{Verbatim}{breaklines,breakanywhere,commandchars=\\\{\}}

\usepackage{graphicx}

\usepackage[a4paper, total={6in, 8in}]{geometry}
\geometry{hmargin=2cm,vmargin=2cm}

\usepackage{sectsty}

\sectionfont{\centering\Huge}
\subsectionfont{\Large}
\subsubsectionfont{\large}

\usepackage{titlesec}
\titlespacing*{\section}
{0pt}{5.5ex plus 1ex minus .2ex}{4.3ex plus .2ex}

\tolerance=1
\emergencystretch=\maxdimen
\hyphenpenalty=10000
\hbadness=10000

%%%%%%%%%%%%%%%%%%%%%%%%%%%%%%%%%%%%%%%%%%%%%%%%%%%%%%%%%%%%%%%%%%%%%%%%%%%%%%%%
%%%%%%%%%%%%%%%%%%%%%%%%%%%%%%%%%%%%%%%%%%%%%%%%%%%%%%%%%%%%%%%%%%%%%%%%%%%%%%%%

\begin{document}

%%%%%%%%%%%%%%%%%%%%%%%%%%%%%%%%%%%%%%%%%%%%%%%%%%%%%%%%%%%%%%%%%%%%%%%%%%%%%%%%
%%%%%%%%%%%%%%%%%%%%%%%%%%%%%%%% Added lines %%%%%%%%%%%%%%%%%%%%%%%%%%%%%%%%%%%
%%%%%%%%%%%%%%%%%%%%%%%%%%%%%%%%%%%%%%%%%%%%%%%%%%%%%%%%%%%%%%%%%%%%%%%%%%%%%%%%

\vspace*{2cm}
\begin{center}
    \textsc{\fontsize{40}{48} \bfseries Bootcamp}\\[0.6cm]
    \textsc{\fontsize{39}{48} \bfseries { %bootcamp_title
Python
    }}\\[0.3cm]
\end{center}
\vspace{3cm}

\begin{center}
\includegraphics[width=200pt]{assets/logo-42-ai.png}{\centering}
\end{center}

\vspace*{2cm}
\begin{center}
    \textsc{\fontsize{32}{48} \bfseries %day_number
Day03    
    }\\[0.6cm]
    \textsc{\fontsize{32}{48} \bfseries %day_title
NumPy    
    }\\[0.3cm]
\end{center}
\vspace{3cm}

\pagenumbering{gobble}
\newpage

%%% >>>>> Page de garde
\setcounter{page}{1}
\pagenumbering{arabic}

%%%%%%%%%%%%%%%%%%%%%%%%%%%%%%%%%%%%%%%%%%%%%%%%%%%%%%%%%%%%%%%%%%%%%%%%%%%%%%%%
%%%%%%%%%%%%%%%%%%%%%%%%%%%%%%%%%%%%%%%%%%%%%%%%%%%%%%%%%%%%%%%%%%%%%%%%%%%%%%%%


\hypertarget{bootcamp-python}{%
\section{Bootcamp Python}\label{bootcamp-python}}

\hypertarget{day03---numpy}{%
\section{Day03 - NumPy}\label{day03---numpy}}

Today you will learn how to use the Python library that will allow you
to manipulate multidimensional arrays (vectors, matrices,
tensors\ldots{}) and perform complex mathematical operations on them.

\hypertarget{notions-of-the-day}{%
\subsection{Notions of the day}\label{notions-of-the-day}}

NumPy array, slicing, stacking, dimensions, broadcasting, normalization,
etc\ldots{}

\hypertarget{general-rules}{%
\subsection{General rules}\label{general-rules}}

\begin{itemize}
\item
  Use the NumPy Library: use NumPy's built-in functions as much as
  possible. Here you will be given no credit for reinventing the wheel.
\item
  The version of Python to use is 3.7, you can check the version of
  Python with the following command: \texttt{python\ -V}
\item
  The norm: during this bootcamp you will follow the
  \href{https://www.python.org/dev/peps/pep-0008/}{PEP 8 standards}. You
  can install \href{https://pypi.org/project/pycodestyle}{pycodestyle}
  which is a tool to check your Python code.
\item
  The function eval is never allowed.
\item
  The exercises are ordered from the easiest to the hardest.
\item
  Your exercises are going to be evaluated by someone else, so make sure
  that your variable names and function names are appropriate and civil.
\item
  Your manual is the internet.
\item
  You can also ask questions in the dedicated channel in the 42 AI
  Slack: 42-ai.slack.com.
\item
  If you find any issue or mistakes in the subject please create an
  issue on our
  \href{https://github.com/42-AI/bootcamp_python/issues}{dedicated
  repository on Github}.
\end{itemize}

\hypertarget{helper}{%
\subsection{Helper}\label{helper}}

For this day you will use the image provided in the \texttt{resources}
folder

Ensure that you have the right Python version.

\begin{Shaded}
\begin{Highlighting}[]
\NormalTok{> which python}
\NormalTok{/goinfre/miniconda/bin/python}
\NormalTok{> python -V}
\NormalTok{Python 3.7.*}
\NormalTok{> which pip}
\NormalTok{/goinfre/miniconda/bin/pip}
\end{Highlighting}
\end{Shaded}

\hypertarget{exercise-00---numpycreator}{%
\subsubsection{Exercise 00 -
NumPyCreator}\label{exercise-00---numpycreator}}

\hypertarget{exercise-01---imageprocessor}{%
\subsubsection{Exercise 01 -
ImageProcessor}\label{exercise-01---imageprocessor}}

\hypertarget{exercise-02---scrapbooker}{%
\subsubsection{Exercise 02 -
ScrapBooker}\label{exercise-02---scrapbooker}}

\hypertarget{exercise-03---colorfilter}{%
\subsubsection{Exercise 03 -
ColorFilter}\label{exercise-03---colorfilter}}

\hypertarget{exercise-04---k-means-clustering}{%
\subsubsection{Exercise 04 - K-means
Clustering}\label{exercise-04---k-means-clustering}}

\clearpage

\hypertarget{exercise-00---numpycreator-1}{%
\section{Exercise 00 -
NumPyCreator}\label{exercise-00---numpycreator-1}}

\begin{longtable}[]{@{}rl@{}}
\toprule
\endhead
Turn-in directory : & ex00\tabularnewline
Files to turn in : & NumPyCreator.py\tabularnewline
Allowed libraries : & NumPy\tabularnewline
Remarks : & n/a\tabularnewline
\bottomrule
\end{longtable}

Write a class named NumPyCreator, which will implement all of the
following methods.\\
Each method receives as an argument a different type of data structure
and transforms it into a NumPy array:

\begin{itemize}
\item
  \texttt{from\_list(lst)} : takes in a list and returns its
  corresponding NumPy array.
\item
  \texttt{from\_tuple(tpl)} : takes in a tuple and returns its
  corresponding NumPy array.
\item
  \texttt{from\_iterable(itr)} : takes in an iterable and returns an
  array which contains all of its elements.
\item
  \texttt{from\_shape(shape,\ value)} : returns an array filled with the
  same value.\\
  The first argument is a tuple which specifies the shape of the array,
  and the second argument specifies the value of all the elements. This
  value must be 0 by default.
\item
  \texttt{random(shape)} : returns an array filled with random values.\\
  It takes as an argument a tuple which specifies the shape of the
  array.
\item
  \texttt{identity(n)} : returns an array representing the identity
  matrix of size n.
\end{itemize}

\texttt{BONUS} : Add to those methods an optional argument which
specifies the datatype (dtype) of the array (e.g.~if you want its
elements to be represented as integers, floats, \ldots{})

\texttt{NOTE} : All those methods can be implemented in one line. You
only need to find the right NumPy functions.

\begin{Shaded}
\begin{Highlighting}[]
\OperatorTok{>>>} \ImportTok{from}\NormalTok{ NumPyCreator }\ImportTok{import}\NormalTok{ NumPyCreator}
\OperatorTok{>>>}\NormalTok{ npc }\OperatorTok{=}\NormalTok{ NumPyCreator()}

\OperatorTok{>>>}\NormalTok{ npc.from_list([[}\DecValTok{1}\NormalTok{,}\DecValTok{2}\NormalTok{,}\DecValTok{3}\NormalTok{],[}\DecValTok{6}\NormalTok{,}\DecValTok{3}\NormalTok{,}\DecValTok{4}\NormalTok{]])}
\NormalTok{array([[}\DecValTok{1}\NormalTok{, }\DecValTok{2}\NormalTok{, }\DecValTok{3}\NormalTok{],}
\NormalTok{       [}\DecValTok{6}\NormalTok{, }\DecValTok{3}\NormalTok{, }\DecValTok{4}\NormalTok{]])}

\OperatorTok{>>>}\NormalTok{ npc.from_tuple((}\StringTok{"a"}\NormalTok{, }\StringTok{"b"}\NormalTok{, }\StringTok{"c"}\NormalTok{))}
\NormalTok{array([}\StringTok{'a'}\NormalTok{, }\StringTok{'b'}\NormalTok{, }\StringTok{'c'}\NormalTok{])}

\OperatorTok{>>>}\NormalTok{ npc.from_iterable(}\BuiltInTok{range}\NormalTok{(}\DecValTok{5}\NormalTok{))}
\NormalTok{array([}\DecValTok{0}\NormalTok{, }\DecValTok{1}\NormalTok{, }\DecValTok{2}\NormalTok{, }\DecValTok{3}\NormalTok{, }\DecValTok{4}\NormalTok{])}

\OperatorTok{>>>}\NormalTok{ shape}\OperatorTok{=}\NormalTok{(}\DecValTok{3}\NormalTok{,}\DecValTok{5}\NormalTok{)}
\OperatorTok{>>>}\NormalTok{ npc.from_shape(shape)}
\NormalTok{array([[}\DecValTok{0}\NormalTok{, }\DecValTok{0}\NormalTok{, }\DecValTok{0}\NormalTok{, }\DecValTok{0}\NormalTok{, }\DecValTok{0}\NormalTok{],}
\NormalTok{       [}\DecValTok{0}\NormalTok{, }\DecValTok{0}\NormalTok{, }\DecValTok{0}\NormalTok{, }\DecValTok{0}\NormalTok{, }\DecValTok{0}\NormalTok{],}
\NormalTok{       [}\DecValTok{0}\NormalTok{, }\DecValTok{0}\NormalTok{, }\DecValTok{0}\NormalTok{, }\DecValTok{0}\NormalTok{, }\DecValTok{0}\NormalTok{]])}

\OperatorTok{>>>}\NormalTok{ npc.random(shape)}
\NormalTok{array([[}\FloatTok{0.57055863}\NormalTok{, }\FloatTok{0.23519999}\NormalTok{, }\FloatTok{0.56209311}\NormalTok{, }\FloatTok{0.79231567}\NormalTok{, }\FloatTok{0.213768}\NormalTok{  ],}
\NormalTok{      [}\FloatTok{0.39608366}\NormalTok{, }\FloatTok{0.18632147}\NormalTok{, }\FloatTok{0.80054602}\NormalTok{, }\FloatTok{0.44905766}\NormalTok{, }\FloatTok{0.81313615}\NormalTok{],}
\NormalTok{      [}\FloatTok{0.79585328}\NormalTok{, }\FloatTok{0.00660962}\NormalTok{, }\FloatTok{0.92910958}\NormalTok{, }\FloatTok{0.9905421}\NormalTok{ , }\FloatTok{0.05244791}\NormalTok{]])}

\OperatorTok{>>>}\NormalTok{ npc.identity(}\DecValTok{4}\NormalTok{)}
\NormalTok{array([[}\FloatTok{1.}\NormalTok{, }\FloatTok{0.}\NormalTok{, }\FloatTok{0.}\NormalTok{, }\FloatTok{0.}\NormalTok{],}
\NormalTok{       [}\FloatTok{0.}\NormalTok{, }\FloatTok{1.}\NormalTok{, }\FloatTok{0.}\NormalTok{, }\FloatTok{0.}\NormalTok{],}
\NormalTok{       [}\FloatTok{0.}\NormalTok{, }\FloatTok{0.}\NormalTok{, }\FloatTok{1.}\NormalTok{, }\FloatTok{0.}\NormalTok{],}
\NormalTok{       [}\FloatTok{0.}\NormalTok{, }\FloatTok{0.}\NormalTok{, }\FloatTok{0.}\NormalTok{, }\FloatTok{1.}\NormalTok{]])}
\end{Highlighting}
\end{Shaded}

\clearpage

\hypertarget{exercise-01---imageprocessor-1}{%
\section{Exercise 01 -
ImageProcessor}\label{exercise-01---imageprocessor-1}}

\begin{longtable}[]{@{}rl@{}}
\toprule
\endhead
Turn-in directory : & ex01\tabularnewline
Files to turn in : & ImageProcessor.py\tabularnewline
Forbidden functions : & None\tabularnewline
Helpful libraries : & Matplotlib\tabularnewline
\bottomrule
\end{longtable}

Build a tool that will be helpful to load and display images in the
upcoming exercises.

Write a class named \texttt{ImageProcessor} that implements the
following methods:

\begin{itemize}
\item
  \texttt{load(path)} : opens the .png file specified by the
  \texttt{path} argument and returns an array with the RGB values of the
  image pixels.\\
  It must display a message specifying the dimensions of the image
  (e.g.~340 x 500).
\item
  \texttt{display(array)} : takes a NumPy array as an argument and
  displays the corresponding RGB image.
\end{itemize}

\texttt{NOTE} : You can use the library of your choice for this
exercise, but converting the image to a NumPy array is mandatory. The
goal of this exercise is to dispense with the technicality of loading
and displaying images, so that you can focus on array manipulation in
the upcoming exercises.

\begin{Shaded}
\begin{Highlighting}[]
\OperatorTok{>>>} \ImportTok{from}\NormalTok{ ImageProcessor }\ImportTok{import}\NormalTok{ ImageProcessor}
\OperatorTok{>>>}\NormalTok{ imp }\OperatorTok{=}\NormalTok{ ImageProcessor()}
\OperatorTok{>>>}\NormalTok{ arr }\OperatorTok{=}\NormalTok{ imp.load(}\StringTok{"../resources/42AI.png"}\NormalTok{)}
\NormalTok{Loading image of dimensions }\DecValTok{200}\NormalTok{ x }\DecValTok{200}
\OperatorTok{>>>}\NormalTok{ arr}
\NormalTok{array([[[}\FloatTok{0.03529412}\NormalTok{, }\FloatTok{0.12156863}\NormalTok{, }\FloatTok{0.3137255}\NormalTok{ ],}
\NormalTok{        [}\FloatTok{0.03921569}\NormalTok{, }\FloatTok{0.1254902}\NormalTok{ , }\FloatTok{0.31764707}\NormalTok{],}
\NormalTok{        [}\FloatTok{0.04313726}\NormalTok{, }\FloatTok{0.12941177}\NormalTok{, }\FloatTok{0.3254902}\NormalTok{ ],}
\NormalTok{        ...,}
\NormalTok{        [}\FloatTok{0.02745098}\NormalTok{, }\FloatTok{0.07450981}\NormalTok{, }\FloatTok{0.22745098}\NormalTok{],}
\NormalTok{        [}\FloatTok{0.02745098}\NormalTok{, }\FloatTok{0.07450981}\NormalTok{, }\FloatTok{0.22745098}\NormalTok{],}
\NormalTok{        [}\FloatTok{0.02352941}\NormalTok{, }\FloatTok{0.07058824}\NormalTok{, }\FloatTok{0.22352941}\NormalTok{]],}

\NormalTok{       [[}\FloatTok{0.03921569}\NormalTok{, }\FloatTok{0.11764706}\NormalTok{, }\FloatTok{0.30588236}\NormalTok{],}
\NormalTok{        [}\FloatTok{0.03529412}\NormalTok{, }\FloatTok{0.11764706}\NormalTok{, }\FloatTok{0.30980393}\NormalTok{],}
\NormalTok{        [}\FloatTok{0.03921569}\NormalTok{, }\FloatTok{0.12156863}\NormalTok{, }\FloatTok{0.30980393}\NormalTok{],}
\NormalTok{        ...,}
\NormalTok{        [}\FloatTok{0.02352941}\NormalTok{, }\FloatTok{0.07450981}\NormalTok{, }\FloatTok{0.22745098}\NormalTok{],}
\NormalTok{        [}\FloatTok{0.02352941}\NormalTok{, }\FloatTok{0.07450981}\NormalTok{, }\FloatTok{0.22745098}\NormalTok{],}
\NormalTok{        [}\FloatTok{0.02352941}\NormalTok{, }\FloatTok{0.07450981}\NormalTok{, }\FloatTok{0.22745098}\NormalTok{]],}

\NormalTok{       [[}\FloatTok{0.03137255}\NormalTok{, }\FloatTok{0.10980392}\NormalTok{, }\FloatTok{0.2901961}\NormalTok{ ],}
\NormalTok{        [}\FloatTok{0.03137255}\NormalTok{, }\FloatTok{0.11372549}\NormalTok{, }\FloatTok{0.29803923}\NormalTok{],}
\NormalTok{        [}\FloatTok{0.03529412}\NormalTok{, }\FloatTok{0.11764706}\NormalTok{, }\FloatTok{0.30588236}\NormalTok{],}
\NormalTok{        ...,}
\NormalTok{        [}\FloatTok{0.02745098}\NormalTok{, }\FloatTok{0.07450981}\NormalTok{, }\FloatTok{0.23137255}\NormalTok{],}
\NormalTok{        [}\FloatTok{0.02352941}\NormalTok{, }\FloatTok{0.07450981}\NormalTok{, }\FloatTok{0.22745098}\NormalTok{],}
\NormalTok{        [}\FloatTok{0.02352941}\NormalTok{, }\FloatTok{0.07450981}\NormalTok{, }\FloatTok{0.22745098}\NormalTok{]],}

\NormalTok{       ...,}

\NormalTok{       [[}\FloatTok{0.03137255}\NormalTok{, }\FloatTok{0.07450981}\NormalTok{, }\FloatTok{0.21960784}\NormalTok{],}
\NormalTok{        [}\FloatTok{0.03137255}\NormalTok{, }\FloatTok{0.07058824}\NormalTok{, }\FloatTok{0.21568628}\NormalTok{],}
\NormalTok{        [}\FloatTok{0.03137255}\NormalTok{, }\FloatTok{0.07058824}\NormalTok{, }\FloatTok{0.21960784}\NormalTok{],}
\NormalTok{        ...,}
\NormalTok{        [}\FloatTok{0.03921569}\NormalTok{, }\FloatTok{0.10980392}\NormalTok{, }\FloatTok{0.2784314}\NormalTok{ ],}
\NormalTok{        [}\FloatTok{0.03921569}\NormalTok{, }\FloatTok{0.10980392}\NormalTok{, }\FloatTok{0.27450982}\NormalTok{],}
\NormalTok{        [}\FloatTok{0.03921569}\NormalTok{, }\FloatTok{0.10980392}\NormalTok{, }\FloatTok{0.27450982}\NormalTok{]],}

\NormalTok{       [[}\FloatTok{0.03137255}\NormalTok{, }\FloatTok{0.07058824}\NormalTok{, }\FloatTok{0.21960784}\NormalTok{],}
\NormalTok{        [}\FloatTok{0.03137255}\NormalTok{, }\FloatTok{0.07058824}\NormalTok{, }\FloatTok{0.21568628}\NormalTok{],}
\NormalTok{        [}\FloatTok{0.03137255}\NormalTok{, }\FloatTok{0.07058824}\NormalTok{, }\FloatTok{0.21568628}\NormalTok{],}
\NormalTok{        ...,}
\NormalTok{        [}\FloatTok{0.03921569}\NormalTok{, }\FloatTok{0.10588235}\NormalTok{, }\FloatTok{0.27058825}\NormalTok{],}
\NormalTok{        [}\FloatTok{0.03921569}\NormalTok{, }\FloatTok{0.10588235}\NormalTok{, }\FloatTok{0.27058825}\NormalTok{],}
\NormalTok{        [}\FloatTok{0.03921569}\NormalTok{, }\FloatTok{0.10588235}\NormalTok{, }\FloatTok{0.27058825}\NormalTok{]],}

\NormalTok{       [[}\FloatTok{0.03137255}\NormalTok{, }\FloatTok{0.07058824}\NormalTok{, }\FloatTok{0.21960784}\NormalTok{],}
\NormalTok{        [}\FloatTok{0.03137255}\NormalTok{, }\FloatTok{0.07058824}\NormalTok{, }\FloatTok{0.21176471}\NormalTok{],}
\NormalTok{        [}\FloatTok{0.03137255}\NormalTok{, }\FloatTok{0.07058824}\NormalTok{, }\FloatTok{0.21568628}\NormalTok{],}
\NormalTok{        ...,}
\NormalTok{        [}\FloatTok{0.03921569}\NormalTok{, }\FloatTok{0.10588235}\NormalTok{, }\FloatTok{0.26666668}\NormalTok{],}
\NormalTok{        [}\FloatTok{0.03921569}\NormalTok{, }\FloatTok{0.10588235}\NormalTok{, }\FloatTok{0.26666668}\NormalTok{],}
\NormalTok{        [}\FloatTok{0.03921569}\NormalTok{, }\FloatTok{0.10588235}\NormalTok{, }\FloatTok{0.26666668}\NormalTok{]]], dtype}\OperatorTok{=}\NormalTok{float32)}
\OperatorTok{>>>}\NormalTok{ imp.display(arr)}
\end{Highlighting}
\end{Shaded}

\clearpage

\hypertarget{exercise-02---scrapbooker-1}{%
\section{Exercise 02 - ScrapBooker}\label{exercise-02---scrapbooker-1}}

\begin{longtable}[]{@{}rl@{}}
\toprule
\endhead
Turn-in directory : & ex02\tabularnewline
Files to turn in : & ScrapBooker.py\tabularnewline
Allowed libraries : & NumPy\tabularnewline
Notions : & Slicing\tabularnewline
\bottomrule
\end{longtable}

Write a class named \texttt{ScrapBooker} which implements the following
methods.\\
All methods take in a NumPy array and return a new modified one.\\
We are assuming that all inputs are correct, i.e.~you don't have to
protect your functions against input errors.

\begin{itemize}
\item
  \texttt{crop(array,\ dimensions,\ position)} : crops the image as a
  rectangle with the given \texttt{dimensions} (meaning, the new height
  and width for the image), whose top left corner is given by the
  \texttt{position} argument. The position should be (0,0) by default.
  You have to consider it an error (and handle said error) if
  \texttt{dimensions} is larger than the current image size.
\item
  \texttt{thin(array,\ n,\ axis)} : deletes every n-th pixel row along
  the specified axis (0 vertical, 1 horizontal), example below.
\item
  \texttt{juxtapose(array,\ n,\ axis)} : juxtaposes \texttt{n} copies of
  the image along the specified axis (0 vertical, 1 horizontal).
\item
  \texttt{mosaic(array,\ dimensions)} : makes a grid with multiple
  copies of the array. The \texttt{dimensions} argument specifies the
  dimensions (meaning the height and width) of the grid (e.g.~2x3).
\end{itemize}

\texttt{NOTE} : In this exercise, when specifying positions or
dimensions, we will assume that the first coordinate is counted along
the vertical axis starting from the TOP, and that the second coordinate
is counted along the horizontal axis starting from the left. Indexing
starts from 0.

e.g.:\\
(1,3)\\
\ldots{}..\\
\ldots{}x.\\
\ldots{}..

example for thin:

\begin{Shaded}
\begin{Highlighting}[]
\NormalTok{perform thin with n=3 and axis=0:}
\NormalTok{ABCDEFGHIJQL        ABDEGHJQ}
\NormalTok{ABCDEFGHIJQL        ABDEGHJQ}
\NormalTok{ABCDEFGHIJQL        ABDEGHJQ}
\NormalTok{ABCDEFGHIJQL        ABDEGHJQ}
\NormalTok{ABCDEFGHIJQL        ABDEGHJQ}
\NormalTok{ABCDEFGHIJQL  ==>   ABDEGHJQ}
\NormalTok{ABCDEFGHIJQL        ABDEGHJQ}
\NormalTok{ABCDEFGHIJQL        ABDEGHJQ}
\NormalTok{ABCDEFGHIJQL        ABDEGHJQ}
\NormalTok{ABCDEFGHIJQL        ABDEGHJQ}
\NormalTok{ABCDEFGHIJQL        ABDEGHJQ}

\NormalTok{perform thin with n=4 and axis=1:}
\NormalTok{AAAAAAAAAAAA        }
\NormalTok{BBBBBBBBBBBB        AAAAAAAAAAAA}
\NormalTok{CCCCCCCCCCCC        BBBBBBBBBBBB}
\NormalTok{DDDDDDDDDDDD        CCCCCCCCCCCC}
\NormalTok{EEEEEEEEEEEE        EEEEEEEEEEEE}
\NormalTok{FFFFFFFFFFFF  ==>   FFFFFFFFFFFF}
\NormalTok{GGGGGGGGGGGG        GGGGGGGGGGGG}
\NormalTok{HHHHHHHHHHHH        IIIIIIIIIIII}
\NormalTok{IIIIIIIIIIII        JJJJJJJJJJJJ}
\NormalTok{JJJJJJJJJJJJ        KKKKKKKKKKKK}
\NormalTok{KKKKKKKKKKKK        }
\NormalTok{LLLLLLLLLLLL        }
\end{Highlighting}
\end{Shaded}

\clearpage

\hypertarget{exercise-03---colorfilter-1}{%
\section{Exercise 03 - ColorFilter}\label{exercise-03---colorfilter-1}}

\begin{longtable}[]{@{}rl@{}}
\toprule
\endhead
Turn-in directory : & ex03\tabularnewline
Files to turn in : & ColorFilter.py\tabularnewline
Forbidden functions : & See each method\tabularnewline
Notions : & Broadcasting\tabularnewline
\bottomrule
\end{longtable}

Now you will build a tool that can apply a variety of color filters on
images. For this exercise, the authorized functions and operators are
specified for each methods. You are not allowed to use anything else.

Write a class named \texttt{ColorFilter} which implements the following
methods:

\begin{itemize}
\item
  \texttt{invert(array)} : Takes a NumPy array of an image as an
  argument and returns an array with inverted color.\\
  Authorized functions: None\\
  Authorized operator: -
\item
  \texttt{to\_blue(array)} : Takes a NumPy array of an image as an
  argument and returns an array with a blue filter.\\
  Authorized functions: .zeros, .shape\\
  Authorized operator: None
\item
  \texttt{to\_green(array)} : Takes a NumPy array of an image as an
  argument and returns an array with a green filter.\\
  Authorized functions: None\\
  Authorized operator: *
\item
  \texttt{to\_red(array)} : Takes a NumPy array of an image as an
  argument and returns an array with a red filter.\\
  Authorized functions : green, blue\\
  Authorized operator: -, +
\item
  \texttt{celluloid(array)} : Takes a NumPy array of an image as an
  argument, and returns an array with a celluloid shade filter. The
  celluloid filter must display at least four thresholds of shades. Be
  careful! You are not asked to apply black contour on the object here
  (you will have to, but later\ldots{}), you only have to work on the
  shades of your images.\\
  Authorized functions: .arange, linspace
\end{itemize}

\texttt{Bonus}: add an argument to your method to let the user choose
the number of thresholds.\\
Authorized functions: .vectorize, .arange\\
Authorized operator: None

\begin{itemize}
\item
  \texttt{to\_grayscale(array,\ filter)} : Takes a NumPy array of an
  image as an argument and returns an array in grayscale. The method
  takes another argument to select between two possible grayscale
  filters. Each filter has specific authorized functions and operators.

  \begin{itemize}
  \item
    `mean' or `m' : Takes a NumPy array of an image as an argument and
    returns an array in grayscale created from the mean of the RBG
    channels.\\
    Authorized functions: .sum, .shape, reshape, broadcast\_to,
    as\_type\\
    Authorized operator: /
  \item
    `weighted' or `w' : Takes a NumPy array of an image as an argument
    and returns an array in weighted grayscale. This argument should be
    selected by default if not given.\\
    The usual weighted grayscale is calculated as : 0.299 * R\_channel +
    0.587 * G\_channel + 0.114 * B\_channel.\\
    Authorized functions: .sum, .shape, .tile\\
    Authorized operator: *
  \end{itemize}
\end{itemize}

\begin{Shaded}
\begin{Highlighting}[]
\OperatorTok{>>>} \ImportTok{from}\NormalTok{ ImageProcessor }\ImportTok{import}\NormalTok{ ImageProcessor}
\OperatorTok{>>>}\NormalTok{ imp }\OperatorTok{=}\NormalTok{ ImageProcessor()}
\OperatorTok{>>>}\NormalTok{ arr }\OperatorTok{=}\NormalTok{ imp.load(}\StringTok{"../42AI.png"}\NormalTok{)}
\NormalTok{Loading image of dimensions }\DecValTok{200}\NormalTok{ x }\DecValTok{200}
\OperatorTok{>>>} \ImportTok{from}\NormalTok{ ColorFilter }\ImportTok{import}\NormalTok{ ColorFilter}
\OperatorTok{>>>}\NormalTok{ cf }\OperatorTok{=}\NormalTok{ ColorFilter()}
\OperatorTok{>>>}\NormalTok{ cf.invert(arr)}
\OperatorTok{>>>}
\OperatorTok{>>>}\NormalTok{ cf.to_green(arr)}
\OperatorTok{>>>}
\OperatorTok{>>>}\NormalTok{ cf.to_red(arr)}
\OperatorTok{>>>}
\OperatorTok{>>>}\NormalTok{ cf.to_blue(arr)}
\OperatorTok{>>>}
\OperatorTok{>>>}\NormalTok{ cf.to_celluloid(arr)}
\OperatorTok{>>>}
\OperatorTok{>>>}\NormalTok{ cf.to_grayscale(arr, }\StringTok{'m'}\NormalTok{)}
\OperatorTok{>>>}
\OperatorTok{>>>}\NormalTok{ cf.to_grayscale(arr, }\StringTok{'weigthed'}\NormalTok{)}
\OperatorTok{>>>}
\end{Highlighting}
\end{Shaded}

\textbf{Examples}

From this base image:

\begin{figure}
\centering
\includegraphics[width=4.16667in,height=\textheight]{tmp/assets/img.png}
\caption{Elon Musk}
\end{figure}

\begin{figure}
\centering
\includegraphics[width=4.16667in,height=\textheight]{tmp/assets/inv.png}
\caption{invert}
\end{figure}

\begin{figure}
\centering
\includegraphics[width=4.16667in,height=\textheight]{tmp/assets/blue.png}
\caption{to\_blue}
\end{figure}

\begin{figure}
\centering
\includegraphics[width=4.16667in,height=\textheight]{tmp/assets/green.png}
\caption{to\_green}
\end{figure}

\begin{figure}
\centering
\includegraphics[width=4.16667in,height=\textheight]{tmp/assets/red.png}
\caption{to\_red}
\end{figure}

\begin{figure}
\centering
\includegraphics[width=4.16667in,height=\textheight]{tmp/assets/cell.png}
\caption{celluloid}
\end{figure}

\clearpage

\hypertarget{exercise-04---k-means-clustering-1}{%
\section{Exercise 04 - K-means
Clustering}\label{exercise-04---k-means-clustering-1}}

\begin{longtable}[]{@{}rl@{}}
\toprule
\endhead
Turn-in directory : & ex04\tabularnewline
Files to turn in : & Kmeans.py\tabularnewline
Forbidden functions : & None\tabularnewline
Remarks : & n/a\tabularnewline
\bottomrule
\end{longtable}

ALERT! DATA CORRUPTED

\textbf{Objective:}

The solar system census dataset is corrupted! The citizens' homelands
are missing!\\
You must implement the K-means clustering algorithm in order to recover
the citizens' origins.

On this web-page you can find good explanations on how K-means is
working:\\
\href{https://bigdata-madesimple.com/possibly-the-simplest-way-to-explain-k-means-algorithm/}{Possibly
the simplest way to explain K-Means algorithm}

The missing part is how to compute the distance between 2 data points
(cluster centroid or a row in the data).\\
In our case the data we have to process is composed of 3 values (height,
weight and bone\_density). Thus, each data point is a vector of 3
values.\\
Now that we have mathematically defined our data points (vector of 3
values), it is then very easy to compute the distance between two points
using vector properties. You can use L1 distance, L2 distance, cosine
similarity, and so forth\ldots{} Choosing the distance to use is called
hyperparameter tuning. I would suggest you to try with the easiest
setting (L1 distance) first.\\
What you will notice is that the final result of the
``training''/``fitting'' will depend a lot on the random initialization.
Commonly, in machine-learning libraries, K-means is run multiple times
(with different random initializations) and the best result is saved.

NB: To implement the fit function, keep in mind that a centroid can be
considered as the gravity center of a set of points.

\textbf{Instructions:}

Create the class \texttt{KmeansClustering} with the following methods:

\begin{Shaded}
\begin{Highlighting}[]
\KeywordTok{class}\NormalTok{ KmeansClustering:}
    \KeywordTok{def} \FunctionTok{__init__}\NormalTok{(}\VariableTok{self}\NormalTok{, max_iter}\OperatorTok{=}\DecValTok{20}\NormalTok{, ncentroid}\OperatorTok{=}\DecValTok{5}\NormalTok{):}
        \VariableTok{self}\NormalTok{.ncentroid }\OperatorTok{=}\NormalTok{ ncentroid }\CommentTok{# number of centroids}
        \VariableTok{self}\NormalTok{.max_iter }\OperatorTok{=}\NormalTok{ max_iter }\CommentTok{# number of max iterations to update the centroids}
        \VariableTok{self}\NormalTok{.centroids }\OperatorTok{=}\NormalTok{ [] }\CommentTok{# values of the centroids}
        
    \KeywordTok{def}\NormalTok{ fit(}\VariableTok{self}\NormalTok{, X):}
        \CommentTok{"""}
\CommentTok{        Run the K-means clustering algorithm.}
\CommentTok{        For the location of the initial centroids, random pick ncentroids from the dataset.}
\CommentTok{        Args:}
\CommentTok{          X: has to be an numpy.ndarray, a matrice of dimension m * n.}
\CommentTok{        Returns:}
\CommentTok{          None.}
\CommentTok{        Raises:}
\CommentTok{          This function should not raise any Exception.}
\CommentTok{        """}

    \KeywordTok{def}\NormalTok{ predict(}\VariableTok{self}\NormalTok{, X):}
        \CommentTok{"""}
\CommentTok{        Predict from wich cluster each datapoint belongs to.}
\CommentTok{        Args:}
\CommentTok{          X: has to be an numpy.ndarray, a matrice of dimension m * n.}
\CommentTok{        Returns:}
\CommentTok{          the prediction has a numpy.ndarray, a vector of dimension m * 1.}
\CommentTok{        Raises:}
\CommentTok{          This function should not raise any Exception.}
\CommentTok{        """}
\end{Highlighting}
\end{Shaded}

\textbf{Dataset:}

The dataset, named \textbf{solar\_system\_census} can be found in the
resources folder.\\
It is a part of the solar system census dataset, and contains biometric
informations such as the height, weight, and bone density of solar
system citizens.

As you should know solar citizens come from four registered areas: The
flying cities of Venus, United Nations of Earth, Mars Republic, and the
Asteroids' Belt colonies.

Unfortunately the data about the planets of origin was lost\ldots{}\\
Use your K-means algorithm to recover it!\\
Once your clusters are found, try to find matches between clusters and
the citizens' homelands.

\textbf{\emph{Hints:}}

\begin{itemize}
\item
  People are slender on Venus than on Earth.
\item
  People of the Martian Republic are taller than on Earth.
\item
  Citizens of the Belt are the tallest of the solar system and have the
  lowest bone density due to the lack of gravity.
\end{itemize}

\textbf{Example:}

Here is an exemple of the algorithm in action:\\
\href{https://i.ibb.co/bKFVVx2/ezgif-com-gif-maker.gif}{K-means
animation}

\clearpage

\end{document}
